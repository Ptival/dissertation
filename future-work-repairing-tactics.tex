\section{Repairing tactic scripts}

Using the same methodology we used to repair programs and terms, we have
considered repairing proof scripts written in a tactic language like \Ltac{}.
The problem with \Coq{} tactics is that their effect depends on what is in
context when the tactic is called, making them hard to reason about statically.
For instance, the \coqinline{intros} tactic inspects the current sub-goal, and
introduces as many universally quantified variables from the sub-goal as it can.
To do so, it has to come up with names for those variables, yet the user never
explicitly passes those names.  Instead, the algorithm chooses suitable, fresh
names, using the names of the variables as they appear in the sub-goal (when
their quantification is explicitly named), or picking some fresh name.

For instance, if your current goal is:

\begin{minted}[linenos=false]{coq}
nat → nat → nat → nat
\end{minted}

\noindent%
%
and you run \coqinline{intros}, the following names will be added to your
context:

\begin{minted}[linenos=false]{coq}
H, H0, H1 : nat
\end{minted}

\noindent%
%
This makes tactic scripts complicated to repair, as there is no lexical binding
position for those automatically introduced names.  In order to repair tactic
scripts, we \emph{must} be able to replay them.  This can only be done outside
the tool if the tactic language has well-defined semantics.  Unfortunately,
\Ltac{} is a very complicated tactic script, and formalizing its semantics is
still an open-ended problem.

Had we done so, we could hope to repair tactics by replaying them, and computing
diffs for the proof context, just like we did for our global environment and
local context when repairing terms.  The problem is very interesting, as a
change in the number of constructors of a data type can have drastic changes in
the structure of a proof.  We have started working on a prototype, toying with
repairing a tactic language that only provides tactics like \coqinline{intro}
and \coqinline{destruct}, but we don't have a proper prototype for those yet.
