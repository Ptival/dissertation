\documentclass[preview]{standalone}

\usepackage{fontspec}
\usepackage{gfsdidot}          % makes math stuff look neat, but also overrides text stuff
\setmainfont{TeX Gyre Pagella} % forces back text stuff to look neat too
\setmonofont{DejaVuSansMono}   % monospace stuff should look neat too

\usepackage[dvipsnames]{xcolor} % must be loaded prior to tikz, pgfplots, etc.
\let\iint\relax
\let\iiint\relax
\let\iiiint\relax
\let\idotsint\relax
\usepackage{amsmath} % loaded by mathtools?
\usepackage{amssymb}
\usepackage[english]{babel}
\usepackage{booktabs}
\usepackage{calc}
\usepackage{caption}
\usepackage{color}
\usepackage{DejaVuSansMono}
\usepackage{dsfont}
\let\textlozenge\relax
\usepackage[inline]{enumitem}
\usepackage{environ}
\usepackage{etoolbox}
\usepackage{fontawesome}
\usepackage{forest}
\usepackage{graphicx}
\usepackage{longtable}
\usepackage{ltablex} \keepXColumns % Made me sad for appendix tables, where was it used?
\usepackage{makecell}
\usepackage{mathpartir}
%\usepackage{mathtools} % \coloneqq
\usepackage{mdframed}
\usepackage{minted}
\usepackage{multicol}
\usepackage[parfill]{parskip}
\usepackage{pgfplots}
\usepackage{pifont} % \ding
\usepackage{soul}
\usepackage{syntax}
\usepackage{tabularx}
\usepackage[skins,theorems]{tcolorbox}
\usepackage{tikz}
\usepackage{tikzpeople}
\usepackage{tikz-qtree}
\usepackage{xpatch}
\usepackage[breaklinks=true,pdfborder={0 0 0}]{hyperref}

% gfsdidot makes ∀ and ∃ super ugly, this reverts them to something fine
\DeclareSymbolFont{CMsymbols}{OMS}{cmsy}{b}{n}
\SetSymbolFont{CMsymbols}{bold}{OMS}{cmsy}{ub}{n}
\DeclareMathSymbol{\forall}{\mathord}{CMsymbols}{"38}
\DeclareMathSymbol{\exists}{\mathord}{CMsymbols}{"39}

\definecolor{monokaibg}{HTML}{272822}
\definecolor{color01}{HTML}{E6194B}
\definecolor{color02}{HTML}{3CB44B}
\definecolor{color03}{HTML}{FFE119}
\definecolor{color04}{HTML}{4363D8}
\definecolor{color05}{HTML}{F58231}
\definecolor{color06}{HTML}{911EB4}
\definecolor{color07}{HTML}{46F0F0}
\definecolor{color08}{HTML}{F032E6}
\definecolor{color09}{HTML}{BCF60C}
\definecolor{color10}{HTML}{FABEBE}
\definecolor{color11}{HTML}{008080}
\definecolor{color12}{HTML}{E6BEFF}
\definecolor{color13}{HTML}{9A6324}
\definecolor{color14}{HTML}{FFFAC8}
\definecolor{color15}{HTML}{800000}
\definecolor{color16}{HTML}{AAFFC3}
\definecolor{color17}{HTML}{808000}
\definecolor{color18}{HTML}{FFD8B1}
\definecolor{color19}{HTML}{000075}
\definecolor{color20}{HTML}{808080}
\definecolor{color21}{HTML}{FFFFFF}
\definecolor{color22}{HTML}{000000}

\def\mathunderline#1#2{\color{#1}\underline{{\color{black}#2}}\color{black}}

\forestset{
  default preamble={
    for tree={
      align=center,
      draw=black,
      edge={
        ultra thick,
      },
      edge path={
        \noexpand\path[\forestoption{edge}]
        (!u.parent anchor)
        -- +(0,-10pt)
        -| (.child anchor)
        \forestoption{edge label};
      },
      font=\bfseries,
      % inner sep=10pt,
      l sep=30pt,
      % line width=2pt,
      minimum height=20pt,
      minimum width=30pt,
      parent anchor=south,
      rectangle,
      ultra thick,
    }
  }
}

\newcolumntype{Y}{>{\centering\arraybackslash}X}

% defines safecoqinline, a command like coqinline but works in tabular
% environments
\makeatletter
\def\safecoqinline#1{%
\ifx\@footnotetext\TX@trial@ftn
\detokenize{#1}%
\else
\coqinline{#1}%
\fi}
\makeatother

\captionsetup{
  justification=centering,
}

\usetikzlibrary{arrows, calc, fit, positioning, shapes, tikzmark}

\tikzset{
bicolor/.style 2 args={
  dashed,dash pattern=on 4pt off 4pt,#1,
  postaction={draw,dashed,dash pattern=on 4pt off 4pt,#2,dash phase=4pt}
  },
}

\tikzstyle{Matching}=[
black,
dashed,
line width=3pt,
]

\tikzstyle{RoundedDottedPath}=[
densely dotted,
color05,
line width=3pt,
rounded corners=10pt,
]

\tikzstyle{RoundedRectangle}=[
dashed,
draw,
line width=3pt,
rounded corners=10pt,
]

\tikzstyle{NodeLabel}=[
black,
circle,
draw,
fill=white,
font=\bfseries,
inner sep=1pt,
ultra thick,
]

\graphicspath{ {./images/} }

% Fix a mdframed bug where skipbelow is ignored
% \makeatletter
% \xpatchcmd{\endmdframed}
%   {\aftergroup\endmdf@trivlist\color@endgroup}
%   {\endmdf@trivlist\color@endgroup\@doendpe}
%   {}{}
% \makeatother

% \surroundwithmdframed[
% backgroundcolor=monokaibg,
% linecolor=white,
% skipabove=1em,
% skipbelow=1em,
% leftmargin=10pt,
% innertopmargin=1pt,
% innerbottommargin=0pt
% ]{minted}

\setminted{
  % DO NOT use bgcolor, use mdframed instead
  % bgcolor=monokaibg,
  linenos=true,
  style=lovelace,
  breaklines=true,
  encoding=utf8,
  fontsize=\large,
  baselinestretch=1,
}

% need bgcolor for the display to be nice in rules
\newmintinline{coq}{bgcolor=white}

\makeatletter
\pgfdeclareshape{document}{
\inheritsavedanchors[from=rectangle] % this is nearly a rectangle
\inheritanchorborder[from=rectangle]
\inheritanchor[from=rectangle]{center}
\inheritanchor[from=rectangle]{north}
\inheritanchor[from=rectangle]{south}
\inheritanchor[from=rectangle]{west}
\inheritanchor[from=rectangle]{east}
% ... and possibly more
\backgroundpath{% this is new
% store lower right in xa/ya and upper right in xb/yb
\southwest \pgf@xa=\pgf@x \pgf@ya=\pgf@y
\northeast \pgf@xb=\pgf@x \pgf@yb=\pgf@y
% compute corner of ‘‘flipped page’’
\pgf@xc=\pgf@xb \advance\pgf@xc by-10pt % this should be a parameter
\pgf@yc=\pgf@yb \advance\pgf@yc by-10pt
% construct main path
\pgfpathmoveto{\pgfpoint{\pgf@xa}{\pgf@ya}}
\pgfpathlineto{\pgfpoint{\pgf@xa}{\pgf@yb}}
\pgfpathlineto{\pgfpoint{\pgf@xc}{\pgf@yb}}
\pgfpathlineto{\pgfpoint{\pgf@xb}{\pgf@yc}}
\pgfpathlineto{\pgfpoint{\pgf@xb}{\pgf@ya}}
\pgfpathclose
% add little corner
\pgfpathmoveto{\pgfpoint{\pgf@xc}{\pgf@yb}}
\pgfpathlineto{\pgfpoint{\pgf@xc}{\pgf@yc}}
\pgfpathlineto{\pgfpoint{\pgf@xb}{\pgf@yc}}
\pgfpathlineto{\pgfpoint{\pgf@xc}{\pgf@yc}}
}
}
\makeatother

%%%%% General-purpose text macros %%%%%
\newcommand{\define}[1]{\emph{#1}}

\newcommand{\Language}[1]{\emph{#1}}
\newcommand{\Gallina}{\Language{Gallina}}
\newcommand{\Ltac}{\Language{Ltac}}
\newcommand{\Vernacular}{\Language{Vernacular}}

\newcommand{\Chick}{\Language{Chick}}
\newcommand{\Coq}{\Language{Coq}}
\newcommand{\Haskell}{\Language{Haskell}}
\newcommand{\JavaScript}{\Language{JavaScript}}
\newcommand{\PeaCoq}{\Language{PeaCoq}}
\newcommand{\RxJS}{\Language{RxJS}}
\newcommand{\SerAPI}{\Language{SerAPI}}
\newcommand{\Snap}{\Language{Snap}}
\newcommand{\TypeScript}{\Language{TypeScript}}

\newcommand{\OperatorColor}{purple}
\newcommand{\Operator}[1]{\textcolor{\OperatorColor}{\ #1\ }}
\newcommand{\Entails}{\Operator{\vdash}}
\newcommand{\HasType}{\Operator{:}}

\newmintinline[mycoq]{coq}{fontsize=\small}
\newcommand{\coqinline}[1]{%
  \colorbox{monokaibg}{%
    \parbox[c][0.9em]{\widthof{\mycoq{#1}}}{\mycoq{#1}}%
  }%
}

\newcommand{\modified}[1]{\color{yellow}{#1}}
\newcommand{\repaired}[1]{\color{green}{#1}}

\newcommand{\rulename}[1]{$\LeftTirNameStyle{#1}$}

\newcommand{\RmApp}{Rm-App}
\newcommand{\RmPi}{Rm-Pi}
\newcommand{\InsApp}{Ins-App}

%%%%% Math-mode macros %%%%%

\newcommand{\opcolor}{purple}
\newcommand{\out}[1]{ \boxed{ \textcolor{teal}{#1} } }
\newcommand{\MathPatches}[3]{
  #1 % \overset{#2}{\mathbin{\textcolor{\opcolor}{\rightsquigarrow}}} \out{#3}
}

\newcommand{\App}{\$}
\newcommand{\Mod}{Mod}
\newcommand{\Drop}{Drop}
\newcommand{\Ins}{Ins}
\newcommand{\Keep}{Keep}
\newcommand{\Lam}{\lambda}

\newcommand{\oMod}{\overset{\mathtt{\Mod}}}
\newcommand{\oDrop}{\overset{\mathtt{\Drop}}}
\newcommand{\oIns}{\overset{\mathtt{\Ins}}}
\newcommand{\oKeep}{\overset{\mathtt{\Keep}}}

\newcommand{\Cons}{::}
\newcommand{\permute}[2]{ \overline{#2}^{\overset{#1}{\rightleftarrows} } }
\newcommand{\permuteOp}[2]{ \overset{\overset{#1}{\rightleftarrows}}{#2} }

\newcommand{\mkMathPiRaw}[4]{#3{\Pi} #4{#1} \rightarrow #2}
\newcommand{\mkMathPi}[5]{\mkMathPiRaw{(#2 : #1)}{#3}{#4}{#5}}


\newcommand{\MathCons}[2]{#1 \Cons #2}
\newcommand{\MathDrop}[1]{\oDrop{\Cons} #1}
\newcommand{\MathDropApp}[1]{\oDrop{\App} #1}
\newcommand{\MathDropLam}[1]{\oDrop{\Lam} #1}
\newcommand{\MathDropPi}[1]{\oDrop{\Pi} #1}
\newcommand{\MathHole}{\texttt{\_}}
\newcommand{\MathIns}[2]{#1 \oIns{\Cons} #2}
\newcommand{\MathInsApp}[2]{\mkMathApp{#1}{#2}{\oIns}}
\newcommand{\MathInsAppOp}{\oIns{\App}}
\newcommand{\MathInsLam}[2]{\mkMathLam{#1}{#2}{\oIns}{}}
\newcommand{\MathInsLamOp}{\oIns{\Lam}}
\newcommand{\MathInsPi}[3]{\mkMathPi {#1}{#2}{#3}{\oIns}{}}
\newcommand{\MathInsPiOp}{\oIns{\Pi}}
\newcommand{\MathLam}[2]{\mkMathLam{#1}{#2}{}{}}
\newcommand{\MathLams}[3]{\mkMathLam{#1}{#2}{}{\BarCount{#3}}}
\newcommand{\MathKeepPiOp}{\oKeep{\Pi}}
\newcommand{\MathMod}[2]{#1 \oMod{\Cons} #2}
\newcommand{\MathModAppOp}{\oMod{\App}}
\newcommand{\MathModLamOp}{\oMod{\Lam}}
\newcommand{\MathModPiOp}{\oMod{\Pi}}
\newcommand{\MathPermute}[2]{\permuteOp{#1}{\Cons} #2}
\newcommand{\MathPermuteLams}[2]{\permuteOp{#1}{\lambda} #2}
\newcommand{\MathPermutePis}[2]{\permuteOp{#1}{\Pi} #2}
\newcommand{\MathPi}[3]{\mkMathPi{#1}{#2}{#3}{}{}}
\newcommand{\MathPis}[4]{\mkMathPi{#1}{#2}{#3}{}{\BarCount{#4}}}
\newcommand{\MathReplace}[1]{#1} % \thickul{#1}}
\newcommand{\MathSame}{\mathds{1}}

\newcommand{\mkMathApp}[3]{#1 #3{ \$ } #2}


\begin{document}
\begin{figure}[htp!]
  \resizebox{\columnwidth}{!}{

\begin{forest}
[,phantom,
  for tree={
    draw opacity=0.25,
    edge={ draw opacity=0.25 },
    fill opacity=0.25,
    text opacity=0.25,
  },
  [→
    [→
      [A,fill=color01]
      [→,name=spec n1,tikz={
        \node[RoundedRectangle,red,fit= () (!1) (!l)] (T1) {};
        \node[below=1pt of T1]{$T_{1}$};
        \node[NodeLabel] at (spec n1.south east) {$l_{1}$};
        },
  for tree={
    draw opacity=1,
    fill opacity=1,
    text opacity=1,
  },
        [C,fill=color02,edge={draw opacity=1}]
        [D,fill=color03,edge={draw opacity=1}]
      ]
    ]
    [→
      [A,fill=color01]
      [→,name=spec n2,tikz={
        \node[RoundedRectangle,red,fit= () (!1) (!l)] (T2) {};
        \node[below=1pt of T2]{$T_{2}$};
        \node[NodeLabel] at (spec n2.south east) {$l_{2}$};
      },
  for tree={
    draw opacity=1,
    fill opacity=1,
    text opacity=1,
  },
        [C,fill=color02,edge={draw opacity=1}]
        [D,fill=color03,edge={draw opacity=1}]
      ]
    ]
  ]
  [,phantom[,phantom[,phantom]]]
  [→
    [→
      [C,fill=color02]
      [→
        [B]
        [→
          [A,fill=color01]
          [D,fill=color03]
        ]
      ]
    ]
    [→
      [A,fill=color01]
      [→
        [B]
        [→,name=spec n3,tikz={
          \node[RoundedRectangle,red,fit= () (!1) (!l)] (T3) {};
          \node[below=1pt of T3]{$T_{3}$};
          \node[NodeLabel] at (spec n3.south east) {$r$};
          },
  for tree={
    draw opacity=1,
    fill opacity=1,
    text opacity=1,
  },
          [C,fill=color02,edge={draw opacity=1}]
          [D,fill=color03,edge={draw opacity=1}]
        ]
      ]
    ]
  ]
]
\end{forest}


% \begin{forest}
%   begin draw/.code={\begin{tikzpicture}[anchor=base,baseline=(current bounding box)]},
% [,phantom,tikz={
%   %\draw[Matching] (!11l) -- (!l1ll);
%   %\draw[Matching] (!11l) -- (!l1);
% },
%   [→,fill=color06,
%       [→,fill=color08,
%         [C,fill=color02,
%         ]
%         [→,fill=color07,
%           [A,fill=color01,
%           ]
%           [D,fill=color03,
%           ]
%       ]
%     ]
%     [→,fill=color09,
%       [A,fill=color12,
%       ]
%       [→,fill=color10,
%         [C,fill=color13]
%         [D,fill=color14]
%       ]
%     ]
%   ]
%   [,phantom[,phantom[,phantom]]]
%   [→,fill=color06,
%     [→,fill=color08,
%       [C,fill=color02,
%       ]
%       [→,fill=color15,
%         [B,
%         ]
%         [→,fill=color07,
%           [A,fill=color01,
%           ]
%           [D,fill=color03,
%           ]
%         ]
%       ]
%     ]
%     [→,fill=color09,
%       [A,fill=color12,
%       ]
%       [→,fill=color16,
%         [B]
%         [→,fill=color10
%           [C,fill=color13]
%           [D,fill=color14]
%         ]
%       ]
%     ]
%   ]
% ]
% \end{forest}
% }%
% %
% \centering
%   \resizebox{0.49\columnwidth}{!}{
% \scalebox{1.25}{
%   \tcbhighmath[size=fbox,boxrule=2pt,colframe=color06,colback=white]{
%     \begin{tabular}{c}
%       $
%       \MathModPi%
%       {
%         \bicolorbox{
%           \MathPermutePis%
%           {[0,1]}
%           {
%             \tcbhighmath[size=fbox,boxrule=2pt,colframe=color08,colback=white]{
%               \MathModPi%
%               {
%                 \tcbhighmath[size=fbox,boxrule=2pt,colframe=color02,colback=white]{
%                   \MathSame{}
%                 }
%               }
%               {\MathSame{}}
%               {
%                 \tcbhighmath[size=fbox,boxrule=2pt,colframe=color15,colback=white]{
%                   \MathInsPi%
%                   {B}
%                   {\_}
%                   {
%                     \tcbhighmath[size=fbox,boxrule=2pt,colframe=color07,colback=white]{
%                       \MathSame{}
%                     }
%                   }
%                 }
%               }
%             }
%           }
%         }
%       }
%       {\MathSame{}}
%       {  } % left empty so that it can be on two lines
%       $
%       \tabularnewline
%       $
%         \tcbhighmath[size=fbox,boxrule=2pt,colframe=color09,colback=white]{
%           \MathModPi%
%           {
%             \tcbhighmath[size=fbox,boxrule=2pt,colframe=color12,colback=white]{
%               \MathSame{}
%             }
%           }
%           {\MathSame{}}
%           {
%             \tcbhighmath[size=fbox,boxrule=2pt,colframe=color16,colback=white]{
%               \MathInsPi%
%               {B}
%               {\_}
%               {
%                 \tcbhighmath[size=fbox,boxrule=2pt,colframe=color10,colback=white]{
%                   \MathSame{}
%                 }
%               }
%             }
%           }
%         }
%       $
%     \end{tabular}
%   }
% }
% }


  }
\end{figure}
\end{document}

%%% Local Variables:
%%% mode: latex
%%% TeX-master: t
%%% End:
