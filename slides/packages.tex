\usepackage{fontspec}
\usepackage{gfsdidot}          % makes math stuff look neat, but also overrides text stuff
\setmainfont{TeX Gyre Pagella} % forces back text stuff to look neat too
\setmonofont{DejaVuSansMono}   % monospace stuff should look neat too

\usepackage[dvipsnames]{xcolor} % must be loaded prior to tikz, pgfplots, etc.
\let\iint\relax
\let\iiint\relax
\let\iiiint\relax
\let\idotsint\relax
\usepackage{amsmath} % loaded by mathtools?
\usepackage{amssymb}
\usepackage[english]{babel}
\usepackage{booktabs}
\usepackage{calc}
\usepackage{caption}
\usepackage{color}
\usepackage{DejaVuSansMono}
\usepackage{dsfont}
\let\textlozenge\relax
\usepackage[inline]{enumitem}
\usepackage{environ}
\usepackage{etoolbox}
\usepackage{fontawesome}
\usepackage{forest}
\usepackage{graphicx}
\usepackage{longtable}
\usepackage{ltablex} \keepXColumns % Made me sad for appendix tables, where was it used?
\usepackage{makecell}
\usepackage{mathpartir}
%\usepackage{mathtools} % \coloneqq
\usepackage{mdframed}
\usepackage{minted}
\usepackage{multicol}
\usepackage[parfill]{parskip}
\usepackage{pgfplots}
\usepackage{pifont} % \ding
\usepackage{soul}
\usepackage{syntax}
\usepackage{tabularx}
\usepackage[skins,theorems]{tcolorbox}
\usepackage{tikz}
\usepackage{tikzpeople}
\usepackage{tikz-qtree}
\usepackage{xpatch}
\usepackage[breaklinks=true,pdfborder={0 0 0}]{hyperref}

% gfsdidot makes ∀ and ∃ super ugly, this reverts them to something fine
\DeclareSymbolFont{CMsymbols}{OMS}{cmsy}{b}{n}
\SetSymbolFont{CMsymbols}{bold}{OMS}{cmsy}{ub}{n}
\DeclareMathSymbol{\forall}{\mathord}{CMsymbols}{"38}
\DeclareMathSymbol{\exists}{\mathord}{CMsymbols}{"39}

\definecolor{monokaibg}{HTML}{272822}
\definecolor{color01}{HTML}{E6194B}
\definecolor{color02}{HTML}{3CB44B}
\definecolor{color03}{HTML}{FFE119}
\definecolor{color04}{HTML}{4363D8}
\definecolor{color05}{HTML}{F58231}
\definecolor{color06}{HTML}{911EB4}
\definecolor{color07}{HTML}{46F0F0}
\definecolor{color08}{HTML}{F032E6}
\definecolor{color09}{HTML}{BCF60C}
\definecolor{color10}{HTML}{FABEBE}
\definecolor{color11}{HTML}{008080}
\definecolor{color12}{HTML}{E6BEFF}
\definecolor{color13}{HTML}{9A6324}
\definecolor{color14}{HTML}{FFFAC8}
\definecolor{color15}{HTML}{800000}
\definecolor{color16}{HTML}{AAFFC3}
\definecolor{color17}{HTML}{808000}
\definecolor{color18}{HTML}{FFD8B1}
\definecolor{color19}{HTML}{000075}
\definecolor{color20}{HTML}{808080}
\definecolor{color21}{HTML}{FFFFFF}
\definecolor{color22}{HTML}{000000}

\def\mathunderline#1#2{\color{#1}\underline{{\color{black}#2}}\color{black}}

\forestset{
  default preamble={
    for tree={
      align=center,
      draw=black,
      edge={
        ultra thick,
      },
      edge path={
        \noexpand\path[\forestoption{edge}]
        (!u.parent anchor)
        -- +(0,-10pt)
        -| (.child anchor)
        \forestoption{edge label};
      },
      font=\bfseries,
      % inner sep=10pt,
      l sep=30pt,
      % line width=2pt,
      minimum height=20pt,
      minimum width=30pt,
      parent anchor=south,
      rectangle,
      ultra thick,
    }
  }
}

\newcolumntype{Y}{>{\centering\arraybackslash}X}

% defines safecoqinline, a command like coqinline but works in tabular
% environments
\makeatletter
\def\safecoqinline#1{%
\ifx\@footnotetext\TX@trial@ftn
\detokenize{#1}%
\else
\coqinline{#1}%
\fi}
\makeatother

\captionsetup{
  justification=centering,
}

\usetikzlibrary{arrows, calc, fit, positioning, shapes, tikzmark}

\tikzset{
bicolor/.style 2 args={
  dashed,dash pattern=on 4pt off 4pt,#1,
  postaction={draw,dashed,dash pattern=on 4pt off 4pt,#2,dash phase=4pt}
  },
}

\tikzstyle{Matching}=[
black,
dashed,
line width=3pt,
]

\tikzstyle{RoundedDottedPath}=[
densely dotted,
color05,
line width=3pt,
rounded corners=10pt,
]

\tikzstyle{RoundedRectangle}=[
dashed,
draw,
line width=3pt,
rounded corners=10pt,
]

\tikzstyle{NodeLabel}=[
black,
circle,
draw,
fill=white,
font=\bfseries,
inner sep=1pt,
ultra thick,
]

\graphicspath{ {./images/} }

% Fix a mdframed bug where skipbelow is ignored
% \makeatletter
% \xpatchcmd{\endmdframed}
%   {\aftergroup\endmdf@trivlist\color@endgroup}
%   {\endmdf@trivlist\color@endgroup\@doendpe}
%   {}{}
% \makeatother

% \surroundwithmdframed[
% backgroundcolor=monokaibg,
% linecolor=white,
% skipabove=1em,
% skipbelow=1em,
% leftmargin=10pt,
% innertopmargin=1pt,
% innerbottommargin=0pt
% ]{minted}

\setminted{
  % DO NOT use bgcolor, use mdframed instead
  % bgcolor=monokaibg,
  linenos=true,
  style=lovelace,
  breaklines=true,
  encoding=utf8,
  fontsize=\large,
  baselinestretch=1,
}

% need bgcolor for the display to be nice in rules
\newmintinline{coq}{bgcolor=white}

\makeatletter
\pgfdeclareshape{document}{
\inheritsavedanchors[from=rectangle] % this is nearly a rectangle
\inheritanchorborder[from=rectangle]
\inheritanchor[from=rectangle]{center}
\inheritanchor[from=rectangle]{north}
\inheritanchor[from=rectangle]{south}
\inheritanchor[from=rectangle]{west}
\inheritanchor[from=rectangle]{east}
% ... and possibly more
\backgroundpath{% this is new
% store lower right in xa/ya and upper right in xb/yb
\southwest \pgf@xa=\pgf@x \pgf@ya=\pgf@y
\northeast \pgf@xb=\pgf@x \pgf@yb=\pgf@y
% compute corner of ‘‘flipped page’’
\pgf@xc=\pgf@xb \advance\pgf@xc by-10pt % this should be a parameter
\pgf@yc=\pgf@yb \advance\pgf@yc by-10pt
% construct main path
\pgfpathmoveto{\pgfpoint{\pgf@xa}{\pgf@ya}}
\pgfpathlineto{\pgfpoint{\pgf@xa}{\pgf@yb}}
\pgfpathlineto{\pgfpoint{\pgf@xc}{\pgf@yb}}
\pgfpathlineto{\pgfpoint{\pgf@xb}{\pgf@yc}}
\pgfpathlineto{\pgfpoint{\pgf@xb}{\pgf@ya}}
\pgfpathclose
% add little corner
\pgfpathmoveto{\pgfpoint{\pgf@xc}{\pgf@yb}}
\pgfpathlineto{\pgfpoint{\pgf@xc}{\pgf@yc}}
\pgfpathlineto{\pgfpoint{\pgf@xb}{\pgf@yc}}
\pgfpathlineto{\pgfpoint{\pgf@xc}{\pgf@yc}}
}
}
\makeatother
