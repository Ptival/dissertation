\section{Describing program modifications with diffs}\label{chick-diffs}

We define a family of data types that allows us to describe changes made to
terms from the language presented in Section~\ref{chick-syntax-terms}, as well
as programs from the language presented in Section~\ref{chick-syntax-programs}.
We will refer to these descriptions of changes as
\textit{diffs}~\footnotemark{}.  We will usually denote a \textit{diff type}
$\Delta_{\tau}$ if it corresponds to values of the diffed type $\tau$.
Informally, one can think of the type $\Delta_{\tau}$ as a descriptor for how
some value $v_{1}$ of type $\tau$ can be transformed into another value $v_{2}$.

\footnotetext{Our approach is very similar to that of~\mycite{miraldo2017type},
though it was derived independently from their work, at around the same time
they were publishing it.}

It is important to clarify how we intend to use those diff types.  We will want
to describe changes made to the data types in the abstract syntax tree
(\define{AST}) of the user's program.  An example will help us avoid some
misconception: consider the user program and its modification in
Figure~\ref{bool-modification}.  While, from the program point of view, a value
of type \coqinline{bool} was modified from the value \coqinline{True} to the
value \coqinline{False}, we will \emph{not} use the type
$\Delta_{\coqinline{bool}}$ to describe this change!  From the point of view of
the abstract syntax tree, a value of type \coqinline{term} has changed from the
original value \coqinline{Var "True"} to the value \coqinline{Var "False"},
which we will capture using the diff type $\Delta_{\coqinline{Term}}$.

To be precise, we will have to describe how a \coqinline{Definition} has
changed, and describe how its name has remained \coqinline{a}, its type has
remained \coqinline{bool}, but its definition has changed.  This will be
described as a value of type $\Delta_{\coqinline{Vernacular}}$, containing a
diff for each of those three components that may change, including the one of
diff type $\Delta_{\coqinline{Term}}$.

\begin{figure*}[!htp]

  \noindent%
  \begin{minipage}[t]{0.50\textwidth}
    \begin{minted}[escapeinside=@@,linenos=false]{coq}
Inductive bool : Set :=
| True  : bool
| False : bool.

Definition a : bool := True. @\dummystrut{ }@
\end{minted}
\end{minipage}%
\begin{minipage}[t]{0.50\textwidth}
  \begin{minted}[escapeinside=@@,linenos=false]{coq}
Inductive bool : Set :=
| True  : bool
| False : bool.

Definition a : bool := @\modified{False}@.
\end{minted}
\end{minipage}

\caption{A simple program and its modification}

\label{bool-modification}

\end{figure*}

When the user makes a change to their program, we will attempt to guess the
structure of their changes, as a value of one of the program diff type
$\Delta_{\coqinline{Program}}$.  To illustrate the kind of changes we are
interested in describing, let us look back at our motivating example in
Figure~\ref{listtovec}, where we can observe the following partial attempts at
refactoring:

\begin{enumerate}

\item they renamed \coqinline{list} into \coqinline{vec},

\item added an index of type \coqinline{nat},

\item renamed constructor \coqinline{nil} into \coqinline{vnil},

\item instantiated the index for the first constructor with \coqinline{O},

\item renamed constructor \coqinline{cons} into \coqinline{vcons},

\item added a parameter \coqinline{n} of type \coqinline{nat} to the second
  constructor,

\item updated the recursive occurrence's name,

\item updated the recursive occurrence's index,

\item and instantiated the index for the second constructor with \coqinline{(S n)}.

\end{enumerate}

Intuitively, we want to capture changes like insertion, modifications,
deletions, and permutations, at all syntactic levels of the meta-language
(within terms, within inductive declarations, etc.).  We will use the same
descriptions to capture user-provided and repair-generated modifications.

Every diff type is accompanied by a patching function.  Given an element of the
diffed type, say $\MathLocalAssum{x}{\chi}$, and an element of the diff type for
that particular type, say $\MathLocalAssum{\delta_{x}}{\Delta_{\chi}}$, the
patching function produces, when successful, a patched element (with the same
type as the original one), say $\MathLocalAssum{x'}{\chi}$.  This operation is
partial for many diff types, because they often capture modifications that only
make sense for certain constructors of the diffed type: for instance, a diff
stating that the head of a list has been modified can not be meaningfully
applied to an empty list.  We will overload the notation
$\MathPatches{x}{\delta_x}{x'}$ to indicate that $x'$ is the (optional) value
obtained when (successfully) patching $x$ according to the diff $\delta_x$,
using the relevant patching function for that diff type.  In all of our
notations, we use a black frame and a teal highlight to indicate that a value is
an output.

\subsection{Atomic diff}

There are many data types for which we will only capture changes at an atomic
granularity: either the value is the same in the new program, or it has been
replaced with a different, unrelated value.  For instance, a binder can either
have the same name, or have been renamed.  Similarly, the only possible change
to the recursive flag of a definition is to be atomically changed to a different
value.  The parameterized diff type $\DeltaAtomic{}$ captures such cases
for a given type $\tau$:

\begin{grammar}
<$\DeltaAtomic{}\ \tau$> ::= \ %trick LaTeX
\alt $\MathSame$                        \hfill (unchanged)
\alt $\MathReplace{\langle\tau\rangle}$ \hfill (replaced)
\end{grammar}
%
with the following semantics:

\begin{mathpar}
  {
    \inferrule*
    [right=Identity]
    {  }
    {\MathPatches{x}{\MathSame}{x}}
  }

  {
    \inferrule*
    [right=Replace]
    {  }
    {\MathPatches{x}{\MathReplace{y}}{y}}
  }

\end{mathpar}

\paragraph{Notation} We chose this notation to be reminiscent of the identity
function (for $\MathSame{}$), and the constant function (for
$\MathReplaceOp{}$).  However, these are not functions, but simply inert
constructors.

\subsection{List diff}~\label{list-diff}

We will now describe our diff type for lists.  \emph{Again}, the subject of
discourse is \emph{not} lists as they appear in the user's program, but lists of
abstract syntax tree constructs, as they appear in the AST of said program.  We
refer the reader to Section~\ref{chick-syntax-terms} and
Section~\ref{chick-syntax-programs}, to remind themselves of the nature of such
lists and their prevalence: the patterns of a \coqinline{match} construct, the
parameters and indices to an inductive type, as well as programs themselves, are
all instances of lists of syntactic constructs.

We provide a rich selection of diff operations for lists.  The aim is not to
have a canonical representation, but rather to capture closely the intent of the
user modifications.  We give the abstract syntax for these operators as prefix
and infix operators, following a visual intuition of what happens to a list
\coqinline{h ∷ t}, though the reader is encouraged to read the semantics,
immediately following, in order to understand the intent of each construct, as
it will probably feel opaque on first glance.  We also provide a concrete
example of a diff in Section~\ref{chick-diffs-example} that uses many of these
constructs.%
%
\footnote{Note that we omit the constructor for the atomic list diff, even
though our implementation requires it to lift values of the atomic diff type to
the corresponding list diff type.}.

\begin{grammar}
<$\DeltaList{}\ \tau \ \Delta_{\tau}$> ::= \ %trick LaTeX
\alt \synt{$\DeltaAtomic{}\ \tau$} \hfill (atomic modification of the whole list)
\alt \begin{tabular}{p{0.9cm} >{\centering}p{0.9cm} l}$\langle\tau\rangle$
       & $\MathIns{}{}$
       & \synt{$\DeltaList{}\ \tau\ \Delta_{\tau}$} \\\end{tabular} \hfill
     (insert a head)
\alt \begin{tabular}{p{0.9cm} >{\centering}p{0.9cm} l}$\langle\Delta_\tau\rangle$
       & $\MathMod{}{}$
       & \synt{$\DeltaList{}\ \tau\ \Delta_{\tau}$} \\\end{tabular}
     \hfill (modify and keep the head)
\alt \begin{tabular}{p{0.9cm} >{\centering}p{0.9cm} l}\quad
       & $\MathDrop{}$
       & \synt{$\DeltaList{}\ \tau\ \Delta_{\tau}$} \\\end{tabular}
     \hfill (drop the head)
\alt \begin{tabular}{p{0.9cm} >{\centering}p{0.9cm} l}\quad
       & $\MathPermute{p}{}$
       & \synt{$\DeltaList{}\ \tau\ \Delta_{\tau}$} \\\end{tabular}
     \hfill (permute according to a permutation $p$)
\end{grammar}
%
with the following semantics:

\begin{mathpar}
  {
    \inferrule*
    [right=Insert]
    {\MathPatches{l}{\delta_{l}}{l'}}
    {\MathPatches{l}{\MathIns{h}{\delta_{l}}}{(h :: l')}}
  }

  {
    \inferrule*
    [right=Modify]
    {\MathPatches{h}{\delta_{h}}{h'} \quad \MathPatches{t}{\delta_{t}}{t'}}
    {\MathPatches{(h :: t)}{\MathMod{\delta_{h}}{\delta_{t}}}{(h' :: t')}}
  }
\\
  % {
  %   \inferrule*
  %   [right=Keep]
  %   {\MathPatches{t}{\delta_{t}}{t'}}
  %   {\MathPatches{(h :: t)}{\MathKeep{\delta_{t}}}{(h :: t')}}
  % }
  {
    \inferrule*
    [right=Drop]
    {\MathPatches{t}{\delta_{t}}{t'}}
    {\MathPatches{(h :: t)}{\MathDrop{\delta_{t}}}{t'}}
  }

  {
    \inferrule*
    [right=Permute]
    {\MathPatches{(h_{p(1)} \Cons \ldots \Cons h_{p(|p|)} \Cons t)}{\delta}{l}}
    {\MathPatches{(h_1 \Cons \ldots \Cons h_{|p|} \Cons t)}{\MathPermute{p}{\delta}}{l}}
  }

\end{mathpar}

Note that we defined the semantics of $\MathModPiOp$ so that it both modifies
and keeps the head: the recursive occurrence in rule \rulename{Modify},
$\delta_t$, therefore applies to the tail $t$ and not the whole list after the
head has been repaired.  On the other hand, the recursive occurrence in rule
\rulename{Permute}, named $\delta$, targets the entire list after the
permutation is performed, not solely the tail: this is necessary because we will
want to perform modifications of elements after having shuffled them around.

\subsection{Term diff}\label{chick-diffs-term-diff}

The diffs for terms include atomic changes, as well as insertion, modification,
deletion, and permutation of most constructors.  We illustrate a couple of
these:

\begin{grammar}
<$\DeltaTerm{}$> ::= \ %trick LaTeX
\alt \synt{$\DeltaAtomic{}\ t$} \hfill (atomic modification)
\alt \synt{$\DeltaTerm{}$} $\oIns{ \$ }$ \synt{$\DeltaTerm{}$} \hfill
(insert application)
\alt $\oIns{\lambda}$ \synt{v}, \synt{$\DeltaTerm{}$} \hfill (insert value
abstraction)
\alt $\oIns{\Pi}$ (\synt{v} : \synt{$\DeltaTerm{}$}),
\synt{$\DeltaTerm{}$} \hfill (insert type abstraction)
\alt … \hfill (other insertions)
\alt … \hfill (removals/modifications/permutations)
\end{grammar}

\noindent Note that we use an infix dollar sign ($\$$) as a symbol for function
application in our diffs, even though we use an infix space for function
application in our terms, which is not ideal, but should help with readability.
For our purpose, we biased the diffs on binary operations in the least
surprising way: deleting a function application keeps its left child, i.e.\
removes the function call and keeps the function (Rule \rulename{\RmApp}), while
deleting a $\Pi$ keeps its right child, i.e.\ removes the value being quantified
but keeps the return type (Rule \rulename{\RmPi}).  For insertion, we allow
maximal flexibility by passing the entire old term to both recursive
occurrences: for instance, the diff $(\MathInsApp{\MathSame}{\MathSame})$ turns
any term $t$ into the self-application $(t\ t)$ (Rule \rulename{\InsApp}).
However, it is often the case that only one recursive occurrence will use the
original term, while the other ones will replace it: for instance, the diff
$(\MathInsApp{\MathSame}{\MathReplace{x}})$ turns any term $t$ into the
application $(t\ x)$.

\begin{mathpar}
  {
    \inferrule*
    [right=\RmApp]
    {\MathPatches{f}{\delta}{f'}}
    {\MathPatches{f\ x}{\MathDropApp{\delta}}{f'}}
  }

  {
    \inferrule*
    [right=\RmPi]
    {\MathPatches{\tau_{2}}{\delta}{\tau_{2}'}}
    {\MathPatches{\MathPi{\tau_{1}}{x}{\tau_{2}}}{\MathDropPi{\delta}}{\tau_{2}'}}
  }

\\

  {
    \inferrule*
    [right=\InsApp]
    {\MathPatches{t}{\delta_1}{t_1} \quad \MathPatches{t}{\delta_2}{t_2}}
    {\MathPatches{t}{\MathInsApp{\delta_1}{\delta_2}}{t_1\ t_2}}
  }

\end{mathpar}

\subsection{Other diff types}

We also need diffs for many other internal data types.  Diff types for tuples
are derived from diff types of their constituents in a straightforward way.
Inductive data type definitions, as well as constructor definitions, behave
essentially like a tuple of all their arguments, so their diff type is derived
accordingly.

\subsection{Example of a program diff}~\label{chick-diffs-example}

With all of this machinery, we can define the original diff for our running
example (again, referring to the changes seen on Figure~\ref{listtovec}).  The
diff is a value of type $\DeltaInductive{}$, as shown in
Figure~\ref{diff-list-vec}.  Because the diffs for the constructors take a lot
of space, they are abbreviated as $\delta \coqinline{nil}$ and $\delta
\coqinline{cons}$, and shown separately in Figures~\ref{diff-list-vec-nil} for
\coqinline{nil} and~\ref{diff-list-vec-cons} for \coqinline{cons}.

\begin{figure*}[!htp]

  \centering

  We demonstrate a diff between the following two programs.  The constructors
are elided, and their diff ($\delta\coqinline{nil}$ and
$\delta\coqinline{cons}$) is shown in Figures~\ref{diff-list-vec-nil}
and~\ref{diff-list-vec-cons} respectively.

\noindent%
\cprotect\fbox{%
  \begin{minipage}[t]{0.50\textwidth-2\fboxrule-2\fboxsep}%
    \begin{minted}[fontsize=\footnotesize,escapeinside=@@,linenos=false]{coq}
Inductive nat : Set :=
| O : nat
| S : ∀ (n : nat), nat.

Inductive list (A : Type) @\dummystrut{ }@
: Type := …               @\dummystrut{ }@

… (* rest of first program *)
    \end{minted}
  \end{minipage}}%
\cprotect\fbox{%
  \begin{minipage}[t]{0.50\textwidth-2\fboxrule-2\fboxsep}%
    \begin{minted}[fontsize=\footnotesize,escapeinside=@@,linenos=false]{coq}
Inductive nat : Set :=
| O : nat
| S : ∀ (n : nat), nat.

Inductive @\modified{vec}@ (A : Type)
: @\modified{nat →}@ Type := …

… (* identical *)
    \end{minted}
  \end{minipage}}

  \vspace{2em}%

  \begin{align*}
&\MathSameInductive{} \MathMod{}{} && \text{do not modify inductive \coqinline{nat}}\\
& \deltaInductive{} ( && \text{modify \coqinline{list} into \coqinline{vec}} \\
& \qquad \MathReplace{\coqinline{vec}}, && \text{modify the name} \\
& \qquad \MathSameList{},       && \text{keep the parameter} \\
& \qquad \MathIns{\coqinline{nat}}{\MathSameList{}}, && \text{add the \coqinline{nat} index} \\
& \qquad \MathSameUniverse{}, && \text{keep the universe} \\
& \qquad \MathMod{\delta\coqinline{nil}}{\MathMod{\delta\coqinline{cons}}{\MathSameList{}}}
  && \text{modify the constructors (elided)} \\
& ) \MathMod{}{} && \\
& \MathSameList{} && \text{do not modify the rest of the program}
  \end{align*}

  \caption{Diff for our running example (constructors elided)}
  \label{diff-list-vec}

\end{figure*}

While the inductive diff should be somewhat straightforward, the reader might be
surprised by $\delta\coqinline{nil}$ (in Figure~\ref{diff-list-vec-nil})
\emph{not} mentioning the renaming of \coqinline{list} (on the left) into
\coqinline{vec} on the right.  While this renaming appears syntactically in the
concrete syntax, it is only a surface-level syntactic requirement that
constructors repeat the type they inhabit.  The abstract syntax tree only
contains the list of indices, and does \emph{not} repeat the name of the inductive
type, nor the parameters of the inductive type, in every constructor.

\begin{figure*}[!htp]

  \noindent%
\cprotect\fbox{%
  \begin{minipage}[t]{0.50\textwidth-2\fboxrule-2\fboxsep}%
    \begin{minted}[fontsize=\footnotesize,escapeinside=@@,linenos=false]{coq}
…
| nil : list A @\dummystrut{ }@
  \end{minted}
\end{minipage}}%
\cprotect\fbox{%
  \begin{minipage}[t]{0.50\textwidth-2\fboxrule-2\fboxsep}%
  \begin{minted}[fontsize=\footnotesize,escapeinside=@@,linenos=false]{coq}
…
| @\modified{vnil}@ : @\modified{vec}@ A @\modified{O}@
  \end{minted}
\end{minipage}}

  \vspace{2em}%

  \begin{align*}
& \delta\coqinline{nil} \coloneqq & && \\
& \qquad \deltaConstructor{} ( && \text{modify the \coqinline{nil} constructor} \\
& \qquad \qquad \MathReplace{\coqinline{vnil}}, && \text{modify constructor name} \\
& \qquad \qquad \MathSame{},                    && \text{no change to parameters} \\
& \qquad \qquad \MathIns{\coqinline{O}}{\MathSame{}}
  && \text{instantiate the \coqinline{nat} index with value \coqinline{O}} \\
& \qquad ) && \\
  \end{align*}

  \caption{Diff for our running example (\coqinline{nil} constructor only)}
  \label{diff-list-vec-nil}

\end{figure*}

\begin{figure*}[!htp]

  \noindent%
\cprotect\fbox{%
  \begin{minipage}[t]{0.50\textwidth-2\fboxrule-2\fboxsep}%
    \begin{minted}[fontsize=\footnotesize,escapeinside=@@,linenos=false]{coq}
…
| cons : A →         @\dummystrut{ }@
    list A → list A. @\dummystrut{ }@
  \end{minted}
\end{minipage}}%
\cprotect\fbox{%
  \begin{minipage}[t]{0.50\textwidth-2\fboxrule-2\fboxsep}%
  \begin{minted}[fontsize=\footnotesize,escapeinside=@@,linenos=false]{coq}
…
| @\modified{vcons}@ : A → @\modified{∀ (n : nat),}@
    @\modified{vec}@ A @\modified{n}@ → @\modified{vec}@ A @\modified{(S n)}@.
  \end{minted}
\end{minipage}}

  \vspace{2em}%

  \begin{align*}
& \delta\coqinline{cons} \coloneqq & && \\
& \qquad \delta_{\ConstructorText} (  & && \text{modify the \coqinline{cons} constructor} \\
& \qquad \qquad \MathReplace{\coqinline{vcons}}, & && \text{modify constructor name} \\
& \qquad \qquad \MathSame{} & \MathMod{}{} && \text{keep the first parameter} \\
& \qquad \qquad ( \coqinline{n} : \coqinline{nat} ) & \MathIns{}{} && \text{insert a second parameter} \\
& \qquad \qquad ( \MathSameBinder{} : \MathInsApp
  {(\MathModApp{\MathReplace{\coqinline{vec}}}{\MathSameTerm{}})}
  {\coqinline{n}}
) & \MathMod{}{}
  && \text{modify \coqinline{list A} into \coqinline{vec A n}} \\
& & && \text{while keeping binder anonymous} \\
& \qquad \qquad \MathSameList{}, & && \text{no other parameter} \\
& \qquad \qquad \coqinline{(S n)} & \MathIns{}{} && \text{instantiate the \coqinline{nat} index} \\
& \qquad \qquad \MathSameList{}, & && \text{no other index} \\
& \qquad ) && \\
  \end{align*}
  \caption{Diff for our running example (\coqinline{cons} constructor only)}\label{diff-list-vec-cons}
\end{figure*}

One might also find the modification of the parameter in
Figure~\ref{diff-list-vec-cons} hard to read.  The diff $\MathInsApp
{(\MathModApp{\MathReplace{\coqinline{vec}}}{\MathSameTerm{}})} {\coqinline{n}}$
might seem inverted, but it only appears so because function application is
left-associative.  Therefore, when comparing \coqinline{list A} with
\coqinline{vec A n}, we are comparing \coqinline{(app list A)} and
\coqinline{(app (app vec A) n)}.  In this form, we can better see that the
outermost application of \coqinline{(app list A)} corresponds to the innermost
application of \coqinline{(app (app vec A) n)}, that is, \coqinline{(app vec
A)}.  This should make it clear that the diff between these two terms needs to
insert an application node first (the one with argument \coqinline{n}), and then
will modify the leftmost argument from \coqinline{list} to \coqinline{vec}.  The
reader might want to refer to rule \rulename{\InsApp} from
Section~\ref{chick-diffs-term-diff} for the semantics of $\MathInsApp{}{}$.

Again, the diff shown in Figure~\ref{diff-list-vec} would be the input to our
repair algorithm, alongside the original program.  The repair algorithm should
propagate changes in such a way as to generate all the other fixes seen in teal
in Figure~\ref{listtovec}.

\subsection{Why compute diffs rather than repaired
values?}~\label{chick-diffs-importance}

We have now hinted multiple times to the importance of computing and storing
diffs for our repair algorithm, rather than just computing and storing the
repaired data.  We now have the necessary syntax and semantics to explain how
the two differ.  Consider an original function \coqinline{f}, with type
\coqinline{nat → bool → string}.  Now consider the following two diffs that can
change this type to \coqinline{bool → nat → string}:

\begin{align*}
\delta_{1} & = \MathPermutePis{[1, 0]}{\MathSame{}}\\
\delta_{2} & =
  \MathDropPi{(\MathModPi
             {\MathSame{}}
             {\MathSame{}}
             {\MathInsPi{\coqinline{_}}{\coqinline{nat}}{\MathSame{}}}
             )}
\end{align*}

\noindent%
%
Indeed if we run our patching function, we obtain:

\begin{align*}
\MathPatches
{\coqinline{nat → bool → string}}
{\delta_{1}}
{\coqinline{bool → nat → string}}\\
\MathPatches
{\coqinline{nat → bool → string}}
{\delta_{2}}
{\coqinline{bool → nat → string}}
\end{align*}

Now, consider some client code that calls this function \coqinline{f}.  For
instance, prior to repairing, let's say that we have a call:

\begin{minted}{coq}
let result := f (S n) b in …
\end{minted}

If the only information we have available, about the type of \coqinline{f}, is
that its new type is \coqinline{bool → nat → string}, all we can do is make a
wild guess as to how we should modify this function call:

\begin{itemize}
\item should the \coqinline{bool} argument be \coqinline{b}?
\item should the \coqinline{nat} argument be \coqinline{(S n)}?
\end{itemize}

Knowing which diff turned the old type of \coqinline{f} into the new type gives
us much more structural information, allowing us to make an educated guess.  If
the diff was $\delta_{1}$, that is, $\MathPermutePis{[1, 0]}{\MathSame{}}$, then
it appears the arguments were swapped, and so we should patch the client code as
follows:

\begin{minted}{coq}
let result := f b (S n) in …
\end{minted}

\noindent%
%
On the other hand, if the diff was $\delta_{2}$, that is,
$\MathDropPi{\MathModPi{\MathSame{}}{\MathSame{}}
{\MathInsPi{\coqinline{_}}{\coqinline{nat}}{\MathSame{}}}}$, then it appears
that the parameter of type \coqinline{bool} has moved to first position, but the
parameter of type \coqinline{nat} is a brand new parameter, unrelated to the one
that was present in the old type.  Therefore, the patch to the client code
should rather produce the following:

\begin{minted}{coq}
let result := f b (_ : nat) in …
\end{minted}

\noindent where we inserted a typed hole of type \coqinline{nat} rather than
conserve the value \coqinline{(S n)} from the old program, as our diff indicates
it is not relevant.

This should illustrate how diffs contain more information than before-after
pairs.  As long as we keep propagating changes through the program as diffs,
rather than patched constructs, we can retain the semantic intent of the changes
throughout our repair algorithm.  However, this relies on obtaining an original
diff, based on the refactoring attempt from the user, that captures their
intent.  We will describe how we can guess satisfactory diffs in
Section~\ref{chick-guess}, though we will sometimes need the programmer to
disambiguate their intent explicitly.
