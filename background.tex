% Pygments/minted put annoying boxes on Unicode because the Coq lexer is stupid
\makeatletter
\expandafter\def\csname PYGmonokai@tok@err\endcsname{\def\PYGmonokai@bc##1{\textcolor[rgb]{0.98,0.15,0.45}{##1}}}
\makeatother

This section describes the various concepts that are necessary to understand the
rest of the dissertation.

\section{Static typing}
\label{static-typing}

% Define static typing

% Discipline of prescribing valid programs from invalid ones

Programmers typically do not write programs by simply manipulating bits in the
computer's memory: most languages offer abstraction mechanisms that lets their
user reason at a higher level.  One such high-level abstraction is the notion of
a \define{data type}.  A programmer will want to manipulate structured data such
as numbers, Boolean values, or compound values made out of different values
organized together in some fashion.  They will then use, and possibly define, a
set of operations to work on values from different data types.  Those operations
might sometimes be entirely oblivious to what data type they are manipulating,
but frequently, they will expect some properties out of their input values, and
provide some properties for their output values.

A \define{type system} is a mechanism that enforces some discipline about the
proper use of values in a program according to a set of typing rules.
\define{Typing rules} usually ascribe a type to every value of a program, and
prescribe the ways in which values of a given type may be used in contexts of a
given type.  A \define{type} can be anything from a simple classification of
values according to their nature (for instance, distinguishing numbers from
functions), to more semantic rules that may sometimes be defined by the user of
the language.  In this dissertation, we will use the notation \mintinline{coq}{t
: τ} to indicate that a term \coqinline{t} has type \coqinline{τ}.

A \define{static} typing discipline allows the programming environment (be it a
compiler, an interpreter, or any other language tool) to reject programs
\textbf{before they run}.  There can be several reasons for why one might want
to reject a program.  The simplest one is a breach of expectation between a
value and the context into which it is passed: this is usually called a
\define{type error}.  For instance, a function declared to expect a number input
might not be allowed to receive a string instead.

An other, less obvious, reason for rejecting a syntactically-valid,
semantically-valid program, is that it makes the static enforcement of rules too
complicated.  For instance, dependent type systems (covered
in Section~\ref{dependent-types}) cannot safely ensure their typing discipline in the
presence of arbitrary recursion.  In order to reject programs before they are
run, they must restrict the programs they allow to some subsets of ``safe''
recursive programs.

While the idea of preventing programmers from running syntactically-valid
programs might seem like a nuisance, it provides at least two benefits.  From
the programmer's point of view, a disciplined use of the type system can help
them catch, before their program is run, conditions that would make the program
crash were it to be executed.  These conditions can be prevented before they
happen, and often are indicated at locations close to the source of error.  From
the programs and programmers receiving the statically-typed code, the typing
discipline provides information that can be leveraged as documentation, or in
order to perform code analyses and transformations that would be intractable or
unsafe without types.

\section{Dependent types}
\label{dependent-types}

This dissertation will often focus on dependent type systems.  A
\define{dependent} type is a type whose definition depends on a program value.
Non-dependent type systems can only express lightweight relationships between
the input and output of functions, as well as lightweight constraints on what
these inputs and outputs may be.  In a dependent type system, one can express
such types as \textit{the type of functions that take a number and return a
larger number}, or \textit{the type of positive numbers that are less than 256}.
In order to express these, one must be able to mention in the type, either
concrete values like \textit{256}, or program values like the input value of the
function.

In order to define functions whose output type depend on the value of their
input, dependent type systems must have a \define{type former} (i.e. a syntactic
construct) for \define{dependent functions}~\footnotemark.  In the literature,
the type of functions which accept an input value $a$ of type $A$, and return an
output value of type $B(a)$ (where $B$ is a family of types indexed by values of
type $A$), is written as either:

\footnotetext{ \textit{Dependent functions} are sometimes referred to as
\textit{dependent products} in the literature.  Unfortunately, the similar but
different concept of a \textit{dependent pair} is sometimes referred to as
either a \textit{dependent sum} or, confusingly, a \textit{dependent product}.
Since both views are reasonable, and in order to avoid confusion, we will
strictly adhere to the unambiguous names of \textit{dependent function type} and
\textit{dependent pair type} in this dissertation. }

\makeatletter
\expandafter\def\csname PYGmonokai@tok@err\endcsname{\def\PYGmonokai@bc##1{\textcolor[rgb]{0.98,0.15,0.45}{##1}}}
\makeatother

\begin{itemize}

  \item \coqinline{(a : A) → B(a)} % $(a : A) \rightarrow B(a)$

  \item \coqinline{∀(a : A), B(a)}
  % $\forall (a : A), B(a)$
        ~\footnote{The $\forall$ symbol is pronounced ``for all''.}

  \item \coqinline{Π(a : A) → B(a)} % $\Pi (a : A) \rightarrow B(a)$

\end{itemize}

We will tend to use the latter, and refer to those types as $\Pi$-types.  When a
dependent function takes several arguments, we will group them all into a single
$\Pi$, and coalesce successive values of the same type in groups under the same
parenthesis.  In practice, this means that the types:

\makeatletter
\expandafter\def\csname PYGmonokai@tok@err\endcsname{\def\PYGmonokai@bc##1{\textcolor[rgb]{0.98,0.15,0.45}{##1}}}
\makeatother

\mintinline{coq}{Π(a : X) (b c : Y) (d : Z) → R(a, b, c, d)}

and

\mintinline{coq}{Π(a : X) → Π(b : Y) → Π(c : Y) → Π(d : Z) → R(a, b, c, d)}

are syntactically equivalent, and we will favor the former (shorter) syntax.

% Introduce Π-telescopes

Nested $\Pi$-types can actually depend on the values of previously-quantified
variables.  We call a \define{$\Pi$-telescope} any sequence of $\Pi$-types
nested within one another.  We can summarize them in a list of bindings and
their types, where each type is allowed to, but does not have to, depend on
previous bindings.  For instance, the type:

\mintinline{coq}{
Π(a : X) (b : Y(a)) (c : Z(a)) → R(a, b, c)
}

contains a telescope of three bindings (namely, \coqinline{a}, \coqinline{b},
and \coqinline{c}), and the type of the last two bindings depend on the value of
the first binding.

Dependent types may also manifest themselves in other forms depending on the
constructs of the language that integrates them.  For instance, languages with
records may want to allow dependent records: the type of some fields may depend
on the value of some other fields.  A typical example is the packing of a plain
data structure with extra properties that we want it to have, for instance,
ensuring the binary-search-tree (or \define{BST}) property of an arbitrary
binary tree:

\begin{minted}{coq}
data BST a = MkBST
  { tree : Tree a
  , bst  : IsBST tree
  }
\end{minted}

\section{Proof assistants}

\begin{itemize}

  \item Define proof assistants

        \begin{itemize}

          \item Present Coq-style proof assistants, introducing concepts of
proof contexts, goals, and tactics

          \item Emphasize the distinction between the program text, and the
program being built by the assistant

          \item X

        \end{itemize}

\end{itemize}

\section{Program refactoring}

\begin{itemize}

  \item Define the notion of a refactoring

        \begin{itemize}

          \item Define the notion of semantic preservation

          \item X

          \item X

        \end{itemize}

  \item X

        \begin{itemize}

          \item X

          \item X

          \item X

        \end{itemize}

\end{itemize}
