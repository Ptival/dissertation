\subsection{Repairing inductive data type definitions}

Figure~\ref{fig:repair-inductive} gives the high-level overview of how inductive
data type definitions are repaired.  The repair operates in three sequential
steps:

\begin{enumerate}

  \item First, the inductive parameters are repaired using
$\repairParametersOp{}$.  Starting with an empty local context, the first
parameter is repaired, and added to the local context, with its repaired diff,
before repairing the next parameter.  Thanks to the local context, further
parameters may be repaired appropriately if they depend on previous parameters
that have undergone repair already.  We return the final local context,
alongside the repaired parameters, because we need to update the indices in the
same context.

  \item Second, the inductive indices are repaired using $\repairIndicesOp{}$,
starting in the local context obtained at the end of the previous step.  Indices
are repaired in a similar fashion to parameters, populating the context with
each repaired index before repairing the next one.  We don't need the final
context further, so it is not returned.

  \item Once we have repaired parameters and indices, we are ready to compute
how the type of the inductive data type is changing, using
$\deltaInductiveTypeOp{}$.  We need this information in order to repair the
constructors, because constructors can contain recursive references to the
inductive data type being defined in their parameters.

\end{enumerate}

We omit most of the details of repairing an inductive data type definition
behind the symbol $\repairIndOp$.  It essentially needs to go recursively in all
parameters and indices of both the inductive type itself and all of its
constructors and repair them.  Each subsequent parameter must be repaired in a
local context where its predecessor parameter has been accounted, so as to react
to possible modifications: for instance, if the first parameter must be renamed,
and the second parameter's type refers to the first parameter's name, the type
will need to be repaired.  Indices also must repaired in the context of the
repaired parameters.  Constructors are repaired one by one, in isolation, but
the repairing of their parameters.  Constructor indices are simply instances of
the inductive indices with terms: they are simply repaired in isolation.

\begin{Rules}
  {fig:repair-inductive}
  { Rules for repairing inductive data type definitions ($\repairIndOp$) }

  \begin{mathpar}
    {
      \inferrule*
      [lab=\RepairVernac{Inductive}]
      {
        {
          \turnstile%
          {\dcontext{[]}{\MathSame{}}{E}{\delta_E}}
          {\repairParameters{\pind{}}{\dpind{}}
            {(\blackdiff{\Gamma}{\delta_{\Gamma}}, \dpind{}')}
          }
        }
        \\\\\\
        {
          \turnstile%
          {\dcontext{\Gamma}{\delta_{\Gamma}}{E}{\delta_E}}
          {\repairIndices{\iind{}}{\diind{}}{\diind{}'}}
        }
        \\\\\\
        \InductiveType{\nind{}}{\pind{}}{\iind{}}{\uind{}}{\tau}
        \\\\\\
        \deltaInductiveType%
        {\diff{\nind{}}{\dnind{}}}
        {\diff{\pind{}}{\dpind{}}}
        {\diff{\iind{}}{\diind{}}}
        {\diff{\uind{}}{\duind{}}}
        {\delta_{\tau}}
        \\\\\\
        {
          \turnstile%
          {
            \dcontext%
            {\MathCons{\MathLocalAssum{\nind{}}{\tau}}{[]}}
            {\MathMod{\MathLocalAssum{\dnind{}}{\delta_{\tau}}}{\MathSame{}}}
            {E}
            {\delta_E}
          }
          {\repairConstructors{\cind{}}{\dcind{}}{\dcind{}'}}
        }
      }
      {
        {\turnstile%
          {\denv{E}{\delta_E}}
          {\repairInd%
            {\diff{\nind{}}{\dnind{}}}
            {\diff{\pind{}}{\dpind{}}}
            {\diff{\iind{}}{\diind{}}}
            {\diff{\uind{}}{\duind{}}}
            {\diff{\cind{}}{\dcind{}}}
            {(\dpind{}', \diind{}', \dcind{}')}
          }
        }
      }
    }
  \end{mathpar}

\end{Rules}

\subsubsection{Repairing inductive parameters}

Figure~\ref{fig:repair-inductive-parameters} gives the algorithm for repairing
inductive parameters ($\repairParametersOp{}$).

\begin{Rules}
  {fig:repair-inductive-parameters}
  { Rules for repairing inductive parameters ($\repairParametersOp$) }

  \begin{mathpar}
    {
      \inferrule*
      [lab=\RepairParameters{Same}]
      {
        {\turnstile%
          {\dcontext
            {E}{\delta_E}
            {\Gamma}{\delta_{\Gamma}}
          }
          {\repairParameters%
            {\MathCons{\MathLocalAssum{v}{\tau}}{\pind{}}}
            {\MathMod{\MathLocalAssum{\MathSameVariable{}}{\MathSameTerm{}}}{\dpind{}}}
            {(\blackdiff{\Gamma'}{\delta_{\Gamma}'}, \dpind{}')}}
        }
      }
      {
        {\turnstile%
          {\dcontext
            {E}{\delta_E}
            {\Gamma}{\delta_{\Gamma}}
          }
          {\repairParameters%
            {\MathCons{\MathLocalAssum{v}{\tau}}{\pind{}}}
            {\MathSameList{}}
            {(\blackdiff{\Gamma'}{\delta_{\Gamma}'}, \dpind{}')}
          }
        }
      }
    }

    {
      \inferrule*
      [lab=\RepairParameters{Mod}]
      {
        {
          \freshOne{v}{\delta_{v}}{\delta_{v}'}
        }
        \\
        {
          \MathPatches{\tau}{\delta_{\tau}}{\tau'}
        }
        \\
        {
          \turnstile
          { \dcontext{E}{\delta_E}{\Gamma}{\delta_\Gamma} }
          { \repair{\tau}{\delta_{\tau}'}{\coqinline{Type}}{\MathSame{}} }
        }
        \\\\\\
        {\turnstile%
          {\dcontext
            {E}{\delta_E}
            {\MathCons{\MathLocalAssum{v}{\tau}}{\Gamma}}
            {\MathMod{\MathLocalAssum{\delta_{v}'}{\delta_{\tau}'}}{\delta_{\Gamma}}}
          }
          {\repairParameters%
            {\pind{}}
            {\dpind{}}
            {(\blackdiff{\Gamma'}{\delta_{\Gamma}'},
              \MathMod{\MathLocalAssum{\delta_{v}'}{\delta_{\tau}'}}{\dpind{}'})}}
        }
      }
      {
        {\turnstile%
          {\dcontext
            {E}{\delta_E}
            {\Gamma}{\delta_{\Gamma}}
          }
          {\repairParameters%
            {\MathCons{\MathLocalAssum{v}{\tau}}{\pind{}}}
            {\MathMod{\MathLocalAssum{\delta_{v}}{\delta_{\tau}} }{\dpind{}}}
            {(\blackdiff{\Gamma'}{\delta_{\Gamma}'}, \dpind{}')}}
        }
      }
    }

    {
      \inferrule*
      [lab=\RepairParameters{Drop}]
      {
        {\turnstile%
          {\dcontext
            {E}{\delta_E}
            {\MathCons{\MathLocalAssum{v}{\tau}}{\Gamma}}
            {\MathDrop{\delta_{\Gamma}}}
          }
          {\repairParameters%
            {\pind{}}
            {\dpind{}}
            {(\blackdiff{\Gamma'}{\delta_{\Gamma}'},\dpind{}')}
          }
        }
      }
      {
        {\turnstile%
          {\dcontext
            {E}{\delta_E}
            {\Gamma}{\delta_{\Gamma}}
          }
          {\repairParameters%
            {\MathCons{\MathLocalAssum{v}{\tau}}{\pind{}}}
            {\MathDrop{\dpind{}}}
            {(\blackdiff{\Gamma'}{\delta_{\Gamma}'}, \dpind{}')}}
        }
      }
    }

    {
      \inferrule*
      [lab=\RepairParameters{Ins}]
      {
        {\turnstile%
          {\dcontext
            {E}{\delta_E}
            {\MathCons{\MathLocalAssum{v}{\tau}}{\Gamma}}
            {\MathDrop{\delta_{\Gamma}}}
          }
          {\repairParameters%
            {\pind{}}
            {\dpind{}}
            {(\blackdiff{\Gamma'}{\delta_{\Gamma}'},\dpind{}')}
          }
        }
      }
      {
        {\turnstile%
          {\dcontext
            {E}{\delta_E}
            {\Gamma}{\delta_{\Gamma}}
          }
          {\repairParameters%
            {\pind{}}
            {\MathIns{\MathLocalAssum{v}{\tau}}{\dpind{}}}
            {(\blackdiff{\Gamma'}{\delta_{\Gamma}'}, \dpind{}')}
          }
        }
      }
    }
  \end{mathpar}

\end{Rules}
