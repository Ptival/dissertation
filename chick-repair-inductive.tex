We omit most of the details of repairing an inductive data type definition
behind the symbol $\repairIndOp$.  It essentially needs to go recursively in all
parameters and indices of both the inductive type itself and all of its
constructors and repair them.  Each subsequent parameter must be repaired in a
local context where its predecessor parameter has been accounted, so as to react
to possible modifications: for instance, if the first parameter must be renamed,
and the second parameter's type refers to the first parameter's name, the type
will need to be repaired.  Indices also must repaired in the context of the
repaired parameters.  Constructors are repaired one by one, in isolation, but
the repairing of their parameters.  Constructor indices are simply
instantiations of the inductive indices with terms: they are simply repaired in
isolation.

\begin{Rules}
  {fig:repair-inductive}
  { Rules for repairing inductive data type definitions ($\repairIndOp$) }

  \begin{mathpar}
    {
      \inferrule*
      [lab=\RepairVernac{Inductive}]
      {
        {
          \turnstile%
          {\dcontext{[]}{\MathSame{}}{E}{\delta_E}}
          {\repairParameters{\pind{}}{\dpind{}}
            {(\blackdiff{\Gamma}{\delta_{\Gamma}}, \dpind{}')}
          }
        }
        \\\\\\
        {
          \turnstile%
          {\dcontext{\Gamma}{\delta_{\Gamma}}{E}{\delta_E}}
          {\repairIndices{\iind{}}{\diind{}}{\diind{}'}}
        }
        \\\\\\
        \InductiveType{\nind{}}{\pind{}}{\iind{}}{\uind{}}{\tau}
        \\\\\\
        \deltaInductiveType%
        {\diff{\nind{}}{\dnind{}}}
        {\diff{\pind{}}{\dpind{}}}
        {\diff{\iind{}}{\diind{}}}
        {\diff{\uind{}}{\duind{}}}
        {\delta_{\tau}}
        \\\\\\
        {
          \turnstile%
          {\dcontext{\MathCons{\tau}{[]}}{\MathMod{TODO}{\MathSame{}}}{E}{\delta_E}}
          {\repairConstructors{\cind{}}{\dcind{}}{\dcind{}'}}
        }
      }
      {
        {\turnstile%
          {\denv{E}{\delta_E}}
          {\repairInd%
            {\diff{\nind{}}{\dnind{}}}
            {\diff{\pind{}}{\dpind{}}}
            {\diff{\iind{}}{\diind{}}}
            {\diff{\uind{}}{\duind{}}}
            {\diff{\cind{}}{\dcind{}}}
            {(\dpind{}', \diind{}', \dcind{}')}
          }
        }
      }
    }
  \end{mathpar}

\end{Rules}
