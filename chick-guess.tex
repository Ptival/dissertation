\section{Guessing diffs}~\label{chick-guess}

We showed how our algorithm could take an initial program, and a diff describing
how to obtain the partial refactoring, in order to generate a repaired diff with
the refactoring carried out.  However, we had postponed the discussion of how
this original diff was obtained.

Automatically generating this diff turns out to be a hard problem.  It is
equivalent to the tree difference problem: given two abstract syntax trees, we
want to find a mapping of pairs of nodes from the old tree and the new tree,
such that we have high confidence that they are related nodes.

There is a bit of literature on the subject.  We chose to use the
\define{GumTree} algorithm, for two reasons.  First, its has a worst-case time
complexity of $O(n^{2})$, where the worst time is unlikely to happen in real
abstract syntax trees.  Second, it works by ranking candidate pairs according to
some heuristic similarity metric, which is useful in order to compute not just
the most likely answer, but, possibly, a list of likely answers ranked by the
heuristic likelihood.

Unfortunately, our early attempts of using the algorithm with our data types
proved unsuccessful.  Let us illustrate where the algorithm fails in our
setting, by first giving a broad overview of how it works:

\begin{itemize}

  \item First, all leaves are distinguished based on their identity.  In our
figures, we will denote nodes that are considered similar using colors.  After
this first pass, all leaves should be colored, such that they are similar to
equal leaves.

  \item Then, all pairs of old internal nodes and new internal nodes are
compared (in an efficient order) for similarity of their children.  The
similarity $S$ between two nodes is defined in Figure~\ref{similarity}.

  \item As items are found to be isomorphic, they are added to a relation $n_{1}
\approx n_{2}$.  Then, remaining pairs of nodes are considered by decreasing
similarity, and added to the same relation.

\end{itemize}

\begin{figure}
\centering
\[
    \text{S}(n_{1}, n_{2}) =
    \frac
    { 2 \times \bigm| \{ c_{1} \in C(n_{1}) | \exists c_{2} \in C(n_{2}), c_{1} \approx c_{2} \} \bigm| }
    { \bigm| \{ c \in C(n_{1}) \} | + | \{ c \in C(n_{2}) \} \bigm| }
\]

where $C(n)$ is the set of children of $n$
\\
and $c_{1} \approx c_{2}$ means that the algorithm considers that $a$ should map to $b$
\caption{Similarity between two nodes}~\label{similarity}
\end{figure}

Let us now see where this algorithm underperforms in our setting.  Consider the
abstract syntax trees in Figure~\ref{chick-sub-trees-equal-similarity}.  In it,
we highlight three identical sub-trees, $T_{1}$, $T_{2}$, and $T_{3}$, in red
dashed rounded rectangles.  Due to being exactly the same sub-trees, their
similarity score will be equal.  Yet, to a human being, $T_{2} \approx T_{3}$
seems more likely than $T_{1} \approx T_{3}$: they are roughly in a similar
location in the abstract syntax tree.

\begin{figure}[htp!]
\centering
\begin{forest}
  [→
    [→
      [A,fill=color01]
      [→,tikz={\node[RoundedRedRectangle,fit=()(!1)(!l)](T1){};\node[below=1pt of T1]{$T_{1}$};}
        [C,fill=color02]
        [D,fill=color03]
      ]
    ]
    [→
      [A,fill=color01]
      [→,tikz={\node[RoundedRedRectangle,fit=()(!1)(!l)](T2){};\node[below=1pt of T2]{$T_{2}$};}
        [C,fill=color02]
        [D,fill=color03]
      ]
    ]
  ]
\end{forest}
\hspace{10pt}
\begin{forest}
  [→
    [→
      [C,fill=color03]
      [→
        [B]
        [→
          [A,fill=color01]
          [D,fill=color03]
        ]
      ]
    ]
    [→
      [A,fill=color01]
      [→
        [B]
        [→,tikz={\node[RoundedRedRectangle,fit=()(!1)(!l)](T3){};\node[below=1pt of T3]{$T_{3}$};}
          [C,fill=color02]
          [D,fill=color03]
        ]
      ]
    ]
  ]
\end{forest}
\caption{Example of sub-trees with equal similarity}\label{chick-sub-trees-equal-similarity}
\end{figure}

\begin{figure}[htp!]
\centering
\begin{forest}
  [→
    [→
      [A,fill=color01]
      [C,fill=color02]
      [D,fill=color03]
    ]
    [A,fill=color01]
    [C,fill=color02]
    [D,fill=color03]
  ]
\end{forest}
\hspace{10pt}
\begin{forest}
  [→
    [→
      [C,fill=color03]
      [B]
      [A,fill=color01]
      [D,fill=color03]
    ]
    [A,fill=color01]
    [B]
    [C,fill=color02]
    [D,fill=color03]
  ]
\end{forest}
\caption{Example of squashed telescopes}\label{chick-squashed-telescopes}
\end{figure}

% \begin{tikzpicture}

%   \tikzstyle{ast} = [%
%   align=center,
%   draw=black,
%   font=\bfseries,
%   inner sep=10pt,
%   line width=3pt,
%   rectangle,
%   ultra thick,
%   ];

%   \tikzset{level distance=45pt};
%   \tikzset{edge from parent/.style=
%     {
%       draw,
%       edge from parent path={(\tikzparentnode.south)
%         -- +(0,-10pt)
%         -| (\tikzchildnode.north)},
%       ultra thick,
%     }
%   }

%   \Tree
%   [.\node[ast]{→};
%     [.\node[ast]{→};
%       [.\node[ast,fill=color01]{A}; ]
%       [.\node[ast]{→};
%         [.\node[ast,fill=color02]{C};]
%         [.\node[ast,fill=color03]{D};]
%       ]
%     ]
%     [.\node[ast]{→};
%       [.\node[ast,fill=color01]{A}; ]
%       [.\node[ast]{→};
%         [.\node[ast,fill=color02]{C};]
%         [.\node[ast,fill=color03]{D};]
%       ]
%     ]
%   ]

% \end{tikzpicture}
