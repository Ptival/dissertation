\section{Repairing programs by propagating changes}~\label{chick-repair}

In this section, we assume that we are given an original program $p$, that is, a
sequence of vernacular commands as described in
Section~\ref{chick-syntax-programs}, and a diff of that program $\delta_p$ as
described in Section~\ref{chick-diffs}, capturing a partial refactoring made by
the user.  We assume that the original proof script type-checked, and attempt to
build a repaired diff $\delta_p'$ such that:

\begin{itemize}

\item $\delta_p'$ contains the changes from $\delta_p$,

\item $\delta_p'$ completes the refactoring started by $\delta_p$, by
propagating forward the changes from $\delta_p$ to use-sites that must be
repaired to account for those changes,

\end{itemize}
%
where propagating changes forward means:

\begin{itemize}

\item propagating renaming of constants and variables,

\item propagating changes in the number, order, and type of arguments to
functions, obtained from their definition, to their use-site,

\item propagating changes in the definition of inductive data types to use-sites
of the type, its constructors, and its eliminators (those will be explained in
more details in Section~\ref{deriving}).

\end{itemize}

We give a top-down description of the repair algorithm, starting at the level of
whole programs, descending all the way to terms and data type definitions.

\subsection{Repairing programs}~\label{repair-program}

The repair algorithm for programs $\repairProgOp$ is described formally in
Figure~\ref{fig:repair-program} using the following judgment:

{
  \[
    {\turnstile
      {\denv{E}{\delta_E}}
      {\repairProg{p}{\delta_p}{\delta_p'}}
    }
  \]
}

\paragraph{Notation} This judgment can be read as follows:
\begin{itemize}
\item \textbf{Top}: If the original program $p$ type-checked in the original environment $E$,
\item \textbf{Bottom left}: and the environment underwent modification $\delta_{E}$,
\item \textbf{Bottom center}: and the program underwent modification $\delta_{p}$,
\item \textbf{Right}: then $\repairProgOp$ proposes to repair the program with modification $\out{\delta'_{p}}$.
\end{itemize}

To be more precise, $E$ is the environment in which the original program $p$ was
defined: a list of term definitions, with type $\GlobalDefinitionText$, and of
inductive data type definitions, with type $\GlobalInductiveText$ (as presented
in Section~\ref{chick-syntax-programs}).  Each vernacular command is
type-checked in such a global typing environment, and upon being executed,
populates it with some additional definitions for the subsequent commands.

$\delta_E$ is a diff indicating how the global environment has been changed by
the time $p$ is reached by the repair algorithm: it accounts for whether some
definitions have been added, removed, or modified, in the prefix of the already
repaired program preceding $p$.  For instance, in our motivating example, by the
time we reach the inductive definition being modified (\coqinline{list},
becoming \coqinline{vec}), $E$ would contain the inductive data type definition
for the type \coqinline{nat}, and $\delta_E$ would keep it intact.  After this
definition has been processed, $E$ would contain, in order, \coqinline{nat}, and
\coqinline{list}, and $\delta_E$ would still keep the former intact, but would
register the diff turning \coqinline{list} into \coqinline{vec}.

\begin{Rules}{fig:repair-program}%
  { Rules for repairing programs ($\repairProgOp$) }

  \begin{mathpar}
    \RuleRProgSameNil{}

    {
      \inferrule*
          [lab=\RProgSameCons]
          {
            {\turnstile%
              { \denv{E}{\delta_E} }
              { \repairProg%
                {\MathCons{v}{p}}
                {\MathMod{\MathSameVernacular{}}{\MathSameProgram{}}}
                {\delta}
              }
            }
          }
          {\turnstile%
            { \denv{E}{\delta_E} }
            { \repairProg{\MathCons{v}{p}}{\MathSameProgram{}}{\delta} }
          }
    }

    {
      \inferrule*
          [lab=\RProgReplace]
          { }
          {\turnstile%
            { \denv{E}{\delta_E} }
            { \repairProg{p}{\MathReplace{q}}{\MathReplace{q}} }
          }
    }

    {
      \inferrule*
          [lab=\RProgIns]
          {\turnstile%
            { \denv%
              {E}
              {\MathIns{v}{\delta_E}}
            }
            { \repairProg{p}{\delta_{p}}{\delta} }
          }
          {\turnstile%
            { \denv{E}{\delta_E} }
            { \repairProg{p}{\MathIns{v}{\delta_{p}}}{\delta} }
          }
    }

    \RuleRProgModify{}

    {
      \inferrule*
          [lab=\RProgDrop]
          {
            {\turnstile%
              { \denv{v \Cons{} E}{\MathDrop{\delta_E}} }
              { \repairProg{p}{\delta_p}{\delta_p'} }
            }
          }
          {\turnstile%
            { \denv{E}{\delta_E} }
            { \repairProg{\MathCons{v}{p}}{\MathDrop{\delta_p}}{\MathDrop{\delta_p'}} }
          }
    }

    {
      \inferrule*
          [lab=\RProgPermute]
          {
            { \MathPatches{q}{\MathPermute{p}{}}{q'} }
            \and
            {\turnstile%
              { \denv{E}{\delta_E} }
              { \repairProg{q'}{\delta}{\delta'} }
            }
          }
          {\turnstile%
            { \denv{E}{\delta_E} }
            { \repairProg{q}{\MathPermute{p}{\delta}}{\MathPermute{p}{\delta'}} }
          }
    }

    \end{mathpar}

\end{Rules}


The repair algorithm for programs takes as input the original program $p$, and
the user-provided modification $\delta_p$.  From those, it outputs a repaired
modification $\out{\delta_p'}$, that propagates the changes introduced by
$\delta_E$ and $\delta_p$ to the rest of the program $p$.  The algorithm
essentially folds over the sequence of vernacular commands that make up the
program, propagating changes from previous commands to subsequent ones.

Rule \rulename{\RProgMod} does the bulk of the work, dispatching the repair of
each vernacular command to the repair algorithm for vernacular commands
$\repairVernacOp$ (described in Section~\ref{repair-vernacular}).  First, the
head vernacular command $v$ is repaired (using $\repairVernacOp$), returning a
repaired diff $\out{\delta_v'}$.  Then, we want to repair the rest of the
program, but the changes made to $v$ may affect the global environment for the
subsequent commands.  For instance, in our motivating example, the vernacular
command defining the inductive type \coqinline{list} gets modified to define the
inductive type \coqinline{vec} instead: when we repair the rest of the program,
we must remember that the original program was defined in a global environment
where \coqinline{list} was defined, and that the updated program must replace it
with \coqinline{vec}.  We therefore repair the rest of the program $p$ in the
appropriate updated environment and its diff.

\rulename{\RProgSameCons} simply defers to \rulename{\RProgMod}: even though it
processes unchanged commands, they might need repairs to account for changes in
their dependencies, as accounted for in the global environment diff.

\TODO{Describe the remaining rules}

\paragraph{Note} While our presentation makes it look like vernacular commands
each add exactly one element to the global environment, it is not always as
simple: inductive data type definitions add \begin{enumerate*} \item the
inductive type (e.g. \coqinline{list}), \item all its constructors
(e.g. \coqinline{nil}, \coqinline{cons}), and \item all its eliminators
(e.g. \coqinline{list_rec}, \coqinline{list_ind}, as described later in
Section~\ref{chick-deriving}) \end{enumerate*}.  For simplifying diff
operations, the whole $\InductiveText$ effectively acts as a placeholder, in our
formalism, for all of those at once.  The lookup rules, given in
Appendix~\ref{appendix-chick}, encapsulate the complexity of accounting for all
the definitions arising from one inductive data type definition.


\subsection{Repairing vernacular commands} \label{repair-vernacular}

We formally describe the repair algorithm for vernacular commands
$\repairVernacOp$ in Figure~\ref{fig:repair-vernac} using the following
judgment:

{
  \[
    {\turnstile
      {\denv{E}{\delta_E}}
      {\repairVernac{v}{\delta_v}{\delta_v'}}
    }
  \]
}

\noindent The rule \rulename{R-Vernac-Def} shows how we attempt to repair a
vernacular term definition.  It is quite involved because we account for the
possibility of a non-recursive definition becoming recursive and vice-versa.  We
want to account for the user-provided modifications $\delta_k$, $\delta_n$,
$\delta_\tau$, and $\delta_t$, but we might need to repair some of them.  We
first repair the type $\tau$, and obtain its repaired diff $\delta_\tau'$.  Now,
the user has modified the name of the definition according to $\delta_n$, but
there are two situations where we must intervene:

\begin{enumerate}

\item If $\delta_n$ is $\MathSame$, the user intends to keep the name $n$ for
this definition.  However, they could have introduced in the global environment,
which would be accounted for in $\delta_E$. In this case, we need to come up
with a fresh name that is free in the new environment.

\item If $\delta_n$ is $\MathReplace{m}$, the user intends to rename this
definition from $n$ into $m$.  Again, this is a problem if the new global
environment contains an $m$ already.  In this case, we also need to come up with
a fresh name that is free in the new environment.

\end{enumerate}

\noindent $\freshOneOp$ takes care of these two situations: it takes as input a
name $n$, and a desired modification for it $\delta_n$, and returns a suitable
modification $\delta_n'$, that tries to preserve $\delta_n$ when it is possible,
to preserve $n$ otherwise when it is possible, or comes up with a fresh name
when neither of these conditions can be met.

Finally, in order to repair the body of the definition, we need to build an
initial local context (as described in Section~\ref{chick-lookup-rules}).  This
context is empty for non-recursive definitions, but should contain a
self-reference for recursive definitions.  The side conditions defining $\Gamma$
and $\delta_\Gamma$ make sure that the local context is appropriately set in all
possible combinations of the recursive flag before and after the
user-modification.

We omit most of the details of repairing an inductive data type definition
behind the symbol $\repairIndOp$.  It essentially needs to go recursively in all
parameters and indices of both the inductive type itself and all of its
constructors and repair them.  Each subsequent parameter must be repaired in a
local context where its predecessor parameter has been accounted, so as to react
to possible modifications: for instance, if the first parameter must be renamed,
and the second parameter's type refers to the first parameter's name, the type
will need to be repaired.  Indices also must repaired in the context of the
repaired parameters.  Constructors are repaired one by one, in isolation, but
the repairing of their parameters.  Constructor indices are simply
instantiations of the inductive indices with terms: they are simply repaired in
isolation.

\begin{Rules}
  {fig:repair-vernac}
  { Rules for repairing vernacular commands ($\repairVernacOp$) }

  \begin{mathpar}
    {
      \inferrule*
      [lab=R-Vernac-Def]
      {
        {\MathPatches{\tau}{\delta_\tau}{\tau'}}
        \and
        {\turnstile
          { \dcontext{E}{\delta_E}{[]}{\MathSame} }
          { \repairTerm{\tau'}{\delta_\tau'}{\MathType}{\MathSame} }
        }
        \and
        {\MathPatches{\tau'}{\delta_\tau'}{\tau''}}
        \\\\\\
        {\turnstile
          {\denv{E}{\delta_E}}
          {\freshOne{n}{\delta_n}{\delta_{n}'}}
        }
        \and
        {\MathPatches{n}{\delta_{n}'}{n'}}
        \and
        {
          \Gamma \op{=}
          \begin{cases}
            []                         & \text{if } k = \text{Definition}\\
            [\MathLocalAssum{n}{\tau}] & \text{if } k = \text{Fixpoint}
          \end{cases}
        }
        \\\\\\
        {
          \delta_\Gamma \op{=}
          \begin{cases}
            \MathIns{\MathLocalAssum{n'}{\tau''}}{\MathSame} & \text{when } k = \text{Definition}\\
                                                             & \phantom{\text{when }} \delta_k = \MathReplace{\text{Fixpoint}}\\
            \MathDrop{\MathSame}                             & \text{when } k = \text{Fixpoint}\\
                                                             & \phantom{\text{when }} \delta_k = \MathReplace{\text{Definition}}\\
            \MathSame                                        & \text{otherwise}
          \end{cases}
        }
        \\\\\\
        {\MathPatches{t}{\delta_t[\subst{\delta_n}{\delta_{n}'}]}{t'}}
        \and
        {\turnstile
          {\dcontext{E}{\delta_E}{\Gamma}{\delta_\Gamma}}
          {\repairTerm{t'}{\delta_{t}'}{\tau}{\delta_\tau'}}
        }
      }
      {\turnstile
        { \denv{E}{\delta_E} }
        { \repairVernac
          {\Definition{k}{n}{\tau}{t}}
          {\ModifyDefinition{\delta_k}{\delta_n}{\delta_\tau}{\delta_t}}
          {\ModifyDefinition{\delta_k}{\delta_n'}{\delta_\tau'}{\delta_t'}}
        }
      }
    }

    {
      \inferrule*
      [lab=R-Vernac-Ind]
      {
        {\turnstile
          {\denv{E}{\delta_E}}
          {\freshOne{n}{\delta_n}{\delta_{n}'}}
        }
        \and
        {\turnstile
          {\denv{E}{\delta_E}}
          {\repairInd
            {\overrightarrow{p}}
            {\delta_{\overrightarrow{p}}}
            {\overrightarrow{i}}
            {\delta_{\overrightarrow{i}}}
            {\overrightarrow{c}}
            {\delta_{\overrightarrow{c}}}
            {(\delta_{\overrightarrow{p}}', \delta_{\overrightarrow{i}}',
              \delta_{\overrightarrow{c}}')}
          }
        }
      }
      {\turnstile
        { \denv{E}{\delta_E} }
        { \repairVernac
          {\Inductive
            {n}
            {\overrightarrow{p}}
            {\overrightarrow{i}}
            {u}
            {\overrightarrow{c}}
          }
          {\ModifyInductive
            {\delta_n}
            {\delta_{\overrightarrow{p}}}
            {\delta_{\overrightarrow{i}}}
            {\delta_u}
            {\delta_{\overrightarrow{c}}}
          }
          {\ModifyInductive
            {\delta_n'}
            {\delta_{\overrightarrow{p}}'}
            {\delta_{\overrightarrow{i}}'}
            {\delta_u}
            {\delta_{\overrightarrow{c}}'}
          }
        }
      }
    }
  \end{mathpar}

\end{Rules}


\subsection{Repairing terms} \label{chick-design-repair-term}

The repair algorithm for terms is described formally in
Figures~\ref{fig:repair-term} and~\ref{rules-unknown-repair} using the following
judgments:

{
  \[
    {\turnstile
      { \dcontext{E}{\delta_E}{\Gamma}{\delta_\Gamma} }
      {\repair{t}{\delta_t}{\tau}{\delta_\tau}}
    }
  \]
  % \[
  %   {\turnstile
  %     { \dcontext{E}{\delta_E}{\Gamma}{\delta_\Gamma} }
  %     {\genericrepair{t}{\delta_t}{\tau}}
  %   }
  % \]
  \[
    {\turnstile
      { \dcontext{E}{\delta_E}{\Gamma}{\delta_\Gamma} }
      {\repairTermWithoutType{t}{\delta_t}}
    }
  \]
}%
%
The first judgment (Figure~\ref{fig:repair-term}) describes how a term is
repaired when we know both its old type $\tau$, and how that type got modified
$\delta_\tau$.  It is syntax-directed by the structure of $\delta_\tau$.  The
second judgment (Figure~\ref{rules-unknown-repair}) describes how a term is
repaired when we no longer make decisions based on its old type and type diff:
this happens for constructs whose type does not inform us about the type of its
constituents (for instance, function application), and for simple terms (holes,
universes, variables).

The judgments for repairing terms include a local context $\Gamma$, behaving
similarly to the global environment $E$ but holding local variables introduced
by the binding structures of terms.  The first judgment is therefore to be
understood as: given that a term $t$ had type $\tau$ in the global environment
$E$ and the local context $\Gamma$, if those latter three respectively undergo
the modifications $\delta_\tau$, $\delta_E$, and $\delta_\Gamma$, then the term
$t$ should be repaired according to $\delta_t$.  The other judgment is to be
read similarly, except the type information is missing.  Importantly, the first
judgment assumes the diff $\delta_\tau$ is repaired already: it is the
responsibility of the caller to ensure so.

\subsubsection*{Diff-directed term repair algorithm}

The algorithm for $\repairTermOneOp$ is directed by the syntax of the diff for
the type of the term being repaired.

The rules~\rulename{\RSamePi} and~\rulename{\RSame} apply when the type has not
been modified.  The first one turns the $\MathSame$ diff for a $\Pi$-type into a
$\MathModPiOp$, to defer the handling of the $\lambda$'s binder, and the
populating of the local context, to the rule \rulename{\RModPi} described
further below.  The second one deals with all other types.  Naively, one might
expect the output to simply be $\MathSame$, however, this would not account for
changes to definitions in the global environment and local context.  The rule
therefore defers this work to the type-directed term repair algorithm
$\repairTermTwoOp$.

The rule that handles the modification of a dependent $\Pi$-type for an explicit
$\lambda$ function, \rulename{\RModPi}, requires a lot of care.  A simple
version of it would not alter the binder $x$ in the term abstraction.
Unfortunately, there is a potential for a binder conflict.  Consider the term
$(\MathLam{a}{b\ a})$ defined in a global environment where $b$ is bound, and
suppose the user renamed that global $b$ into $a$.  The algorithm will
eventually need to replace the occurrence of $b$ in the term abstraction with an
$a$, but in order to do so without capture, we had better $\alpha$-rename the
abstraction's binder with a variable that is fresh in the repaired program.  We
use $\freshOneOp$ again to pick a suitable binder.

A similar issue arises when we want to repair the body of the abstraction: we
want to instantiate the term abstraction's binder $x$ and the type abstraction's
binder $\chi$ with the same variable.  Unfortunately, we can't always choose $x$
because the body of the type abstraction $\tau_2$ might depend on an existing
$x$ in either the global environment or local context, neither can we always
choose $\chi$ because the body of the term abstraction $t$ might depend on such
a $\chi$ too.  The $\freshTwoOp$ algorithm picks such a suitably-fresh variable
$z$.  This $z$ is only necessary for the algorithm, and will never be shown to
the programmer.

Now armed with $\delta_x$ and $z$, we can introduce $(z : \tau_1)$ in the local
context, and have it not need be renamed in the diffed context.  In this
enriched context, we can repair the body of the abstraction, $t$, after having
performed the substitution of $z$ for $x$.  We must perform similar
substitutions in $\tau_2$ and $\delta_{\tau_2}$.  We obtain the diff
$\delta_\tau$, which repairs the body of the abstraction when its binder is
named $z$.  However, in the original program, we will use $\delta_x$ to repair
the binder.  We must therefore substitute back instances of $\MathReplace{x}$
where we find free instances of $\MathReplace{x}$ in $\delta_\tau$.  The same
considerations must be taken into account when a $\Pi$ is inserted
(\rulename{\RInsPi}).

To deal properly with $\Pi$-permutations (\rulename{\RPermutePis}), we must be
able to syntactically extract as many $\lambda$s as the length of the
permutation.  When we can do so, the repair proceeds recursively after having
permuted both the term and the type abstractions.  When the original type $\tau$
gets replaced by an arbitrary type $\tau'$ (\rulename{\RReplace}), we have no
better option than inserting a typed hole.  The final rule
(\rulename{\RepairTermPrefix-Otherwise-\GenericRepairPrefix}) triggers if none
of the previous rules have, and dispatches the repair to the type-directed term
repair algorithm $\repairTermTwoOp$.

\begin{Rules}{fig:repair-term}{ Rules for repairing terms, diff-directed ($\repairTermOneOp$) }

\begin{mathpar}
  {
    \inferrule*
    [lab=\RSamePi]
    {
      {\turnstile
        { \dcontext{E}{\delta_E}{\Gamma}{\delta_\Gamma} }
        { \repair
          {\MathLam{x}{t}}
          {\delta}
          {\MathPi{\tau_1}{\chi}{\tau_2}}
          {\MathModPi{\MathSame}{\MathSame}{\MathSame}}
        }
      }
    }
    {\turnstile
      { \dcontext{E}{\delta_E}{\Gamma}{\delta_\Gamma} }
      { \repair
        {\MathLam{x}{t}}
        {\delta}
        {\MathPi{\tau_1}{\chi}{\tau_2}}
        {\MathSame}
      }
    }
  }

  {
    \inferrule*
    [lab=\RSame]
    {\turnstile
      { \dcontext{E}{\delta_E}{\Gamma}{\delta_\Gamma} }
      { \repairTermWithoutType{t}{\delta_{t}}}
    }
    {\turnstile
      { \dcontext{E}{\delta_E}{\Gamma}{\delta_\Gamma} }
      { \repair{t}{\delta_{t}}{\tau}{\MathSame} }
    }
  }

  \newcommand{\RepairTermDiffModPi}{
  \inferrule*
  [lab=\RModPi]
  {
    {\turnstile%
      {\dcontext{E}{\delta_E}{\Gamma}{\delta_\Gamma}}
      {\freshOne{x}{\MathSame}{\delta_x}}
    }
    \and
    {\turnstile%
      {\dcontext{E}{\delta_E}{\Gamma}{\delta_\Gamma}}
      {\freshTwo{\blackdiff{x}{\delta_x}}{\blackdiff{\chi}{\delta_\chi}}{z}}
    }
    \\\\\\
    {\MathPatches{x}{\delta_x}{x'}}
    \and
    {\MathPatches{\chi}{\delta_\chi}{\chi'}}
    \\\\\\
    {\turnstile%
      { \dcontext{E}{\delta_E}
        { \MathCons{\MathLocalAssum{z}{\tau_1}}{\Gamma} }
        { \MathMod{\MathLocalAssum{\MathSame}{\delta_{\tau_1}}}{\delta_\Gamma} }
      }
      { \repair%
        {t[\subst{x}{z}]}
        {\delta_{t}}
        {\tau_2[\subst{\chi}{z}]}
        {\delta_{\tau_2}[\subst{\MathReplace{\chi'}}{\MathReplace{z}}]}
      }
    }
  }
  {\turnstile%
    { \dcontext{E}{\delta_E}{\Gamma}{\delta_\Gamma} }
    {\repair%
      { \MathLam{x}{t} }
      { \MathModLam{\delta_x}{\delta_t[\subst{\MathReplace{z}}{\MathReplace{x'}}]} }
      { \MathPi{\tau_1}{\chi}{\tau_2} }
      { \MathModPi{\delta_{\tau_1}}{\delta_\chi}{\delta_{\tau_2}} }
    }
  }
}

  \RepairTermDiffModPi{}

  {
    \inferrule*
    [lab=\RInsPi]
    {
      {\turnstile
        {\dcontext{E}{\delta_E}{\Gamma}{\delta_\Gamma}}
        {\freshOne{\chi}{\MathSame}{\delta_\chi}}
      }
      \and
      {\MathPatches{\chi}{\delta_\chi}{x}}
      \and
      {\MathPatches{\tau}{\delta_{\tau_1}}{\tau_1}}
      \\\\\\
      \turnstile
      {\dcontext{E}{\delta_E}{\Gamma}
        {\MathIns{\MathLocalAssum{x}{\tau_1}}{\delta_\Gamma}}
      }
      { \repair{t}{\delta_{t}}{\tau}
        {\delta_{\tau_2}[\subst{\MathReplace{\chi}}{\MathReplace{x}}]}
      }
    }
    {
      \turnstile
      { \dcontext{E}{\delta_E}{\Gamma}{\delta_\Gamma} }
      { \repair
        {t}
        { \MathInsLam{x}{\delta_{t}} }
        {\tau}
        { \MathInsPi{\delta_{\tau_1}}{\chi}{\delta_{\tau_2}} }
      }
    }
  }

  {
    \inferrule*
    [lab=\RDropPi]
    {
      {\turnstile
        {\dcontext{E}{\delta_E}{\Gamma}{\delta_\Gamma}}
        {\freshTwo{\blackdiff{x}{\MathSame}}{\blackdiff{\chi}{\MathSame}}{z}}
      }
      \\
      {\turnstile
        {\dcontext
          {E}{\delta_E}
          {\MathCons{\MathLocalAssum{z}{\tau_1}}{\Gamma}}
          {\MathDrop{\delta_\Gamma}}
        }
        {\repair
          {t[\subst{x}{z}]}
          {\delta_{t}}
          {\tau[\subst{\chi}{z}]}
          {\delta_{\tau_2}}
        }
      }
    }
    {
      \turnstile
      { \dcontext{E}{\delta_E}{\Gamma}{\delta_\Gamma} }
      { \repair
        {\MathLam{x}{t}}
        {\MathDropLam{\delta_{t}}}
        {\MathPi{\tau_1}{\chi}{\tau_2}}
        {\MathDropPi{\delta_{\tau_2}}}
      }
    }
  }

  {
    \inferrule*
    [lab=\RPermutePis]
    {\turnstile
      { \dcontext{E}{\delta_E}{\Gamma}{\delta_\Gamma} }
      {\repair
        { \MathLam{x_{p(1)} \ldots x_{p(|p|)}}{t} }
        { \delta }
        { \mkMathPiRaw{(\chi_{p(1)} : \tau_{p(1)}) \ldots (\chi_{p(|p|)} : \tau_{p(|p|)})}{\tau}{}{} }
        { \dtau{} }
      }
    }
    {\turnstile
      { \dcontext{E}{\delta_E}{\Gamma}{\delta_\Gamma} }
      {\repair
        { \MathLam{x_1 \ldots x_{|p|}}{t} }
        { \MathPermuteLams{p}{\delta} }
        { \mkMathPiRaw{(\chi_{1} : \tau_i) \ldots (\chi_{|p|} : \tau_{|p|})}{\tau}{} }
        { \MathPermutePis{p}{\dtau{}} }
      }
    }
  }

  {
    \inferrule*
    [lab=\RReplace]
    { }
    {\turnstile
      { \dcontext{E}{\delta_E}{\Gamma}{\delta_\Gamma} }
      { \repair
        {t}
        {\MathReplace{\MathAnnot{\MathHole}{\tau'}}}
        {\tau}
        {\MathReplace{\tau'}}
      }
    }
  }

  {
    \inferrule*
    [lab=\RepairTermPrefix-Otherwise-\GenericRepairPrefix]
    {\turnstile
      { \dcontext{E}{\delta_E}{\Gamma}{\delta_\Gamma} }
      { \genericrepair
        { t }
        { \delta_t }
        { \tau }
      }
    }
    {\turnstile
      { \dcontext{E}{\delta_E}{\Gamma}{\delta_\Gamma} }
      {\repair
        { t }
        { \delta_t }
        { \tau }
        { \delta_\tau }
      }
    }
  }

\end{mathpar}

\end{Rules}


\subsubsection*{Term-directed term repair algorithm}

$\repairTermThreeOp$ is an algorithm directed by the syntax of the term being
repaired, used when no information is known about its type and type diff.

For repairing variables (Rule \rulename{\UTRVar}) and function applications
(Rule \rulename{\UTRApp}), we use another helper, $\repairArgsOp$, to compute
the diff.  First, we syntactically extract as many application nodes as
possible, building a list of the arguments, until the function symbol (in the
case of a variable, this corresponds to an empty list of arguments).  Once we
lookup the type diff for the function symbol (resp. $\delta_{\tau_v}$ and
$\delta_{\tau_f}$), we can start producing the output diff.  An example will be
illustrative of the $\repairArgsOp$ algorithm: consider the case where a
function $f$ had type $(A \rightarrow C)$ and undergoes the transformation
$\MathModPi{\MathSame}{\MathSame}{(\MathInsPi{B}{\_}{\MathSame})}$, yielding the
type $(A \rightarrow B \rightarrow C)$.  Suppose the original code contains a
call to $f$, for instance $(f\ a)$.  A repaired version of this call would look
like $(f\ a\ (\MathAnnot{\MathHole}{B}))$.  This means the repair diff should be
$(\MathInsApp{(\MathModApp{\MathSame}{\MathSame})}{(\MathAnnot{\MathHole}{B})})$.
Notice how the outermost term modification, $\MathInsAppOp$, corresponds to the
innermost type modification $\MathInsPiOp$.  This is not surprising: a function
$g$ whose type is $(A \rightarrow (B \rightarrow (C \rightarrow D)))$, and its
application to arguments $(((g\ a)\ b)\ c)$, exhibit the same inversion.  For
this reason, $\repairArgsOp$ works by processing the $\Pi$-telescope type diff
(i.e. a sequence of nested $\Pi$s) from outside-in, and builds the applications
term diff from inside-out.  We omit the rules for $\repairArgsOp$ as they would
be cumbersome and not enlightening.

We also omit the rules for repairing a \coqinline{match} construct behind the
symbol $\repairBranchesOp$.  Once the discriminee is repaired, the branches must
be repaired: first, the discriminated inductive data type is identified, and its
diff is retrieved from the global environment.  From this, the list of branches
is repaired, with proper care for the binding structure of the patterns.  Our
current implementation only supports non-nested pattern matches, and proceeds by
computing a permutation from the order in which patterns appear in the program
to the order in which they appear in the inductive data type declaration.  The
process of repairing the branches is then straightforward: if a constructor was
added, a branch is added, if a constructor was removed, its branch (if any) is
removed, and modifications to constructors impact the pattern and arm of its
matching branch.  Notice that we do not support the full power of
dependently-typed elimination yet: our \coqinline{match} construct does not have
an index binding declaration or a return type annotation (respectively known as
the \coqinline{in}-clause and the \coqinline{return}-clause in \Gallina{}).

Also of note, in Rule \rulename{\UTRAnnot}, we actually retrieve type
information from the annotation, and can repair the annotated term with more
information.  In practice however, we don't expect much code to contain type
annotations.  Finally, Rule \rulename{\UTROtherwise} captures terms such as
holes, and universes, which we repair by doing nothing.

\begin{Rules}
  {rules-unknown-repair}
  { Rules for $\repairTermThreeOp$, the term-directed repair algorithm. }

\begin{mathpar}
  {
    \inferrule*
    [lab=\UTRVar]
    {
      {\turnstile
        { \dcontext{E}{\delta_E}{\Gamma}{\delta_\Gamma} }
        { \declDiff{v}{\delta_v}{\delta_{\tau_v}} }
      }
      \and
      {\turnstile
        { \dcontext{E}{\delta_E}{\Gamma}{\delta_\Gamma} }
        { \repairArgs{ [] }{\tau_v}{\delta_{\tau_v}}{\delta_v}{\delta}
        }
      }
    }
    {\turnstile
      { \dcontext{E}{\delta_E}{\Gamma}{\delta_\Gamma} }
      { \repairTermWithoutType{v}{\delta} }
    }
  }

  {
    \inferrule*
    [lab=\UTRApp]
    {
      {\turnstile
        { \dcontext{E}{\delta_E}{\Gamma}{\delta_\Gamma} }
        { \declDiff{f}{\delta_f}{\delta_{\tau_f}} }
      }
      \and
      {\turnstile
        { \dcontext{E}{\delta_E}{\Gamma}{\delta_\Gamma} }
        { \repairArgs{[a_1, \ldots, a_n]}{\tau_f}{\delta_{\tau_f}}
          {\delta_f}{\delta}
        }
      }
    }
    {\turnstile
      { \dcontext{E}{\delta_E}{\Gamma}{\delta_\Gamma} }
      { \repairTermWithoutType{ f\ a_1\ \ldots\ a_n }{\delta} }
    }
  }

  {
    \inferrule*
    [lab=\UTRPi]
    {
      {\turnstile
        {\dcontext{E}{\delta_E}{\Gamma}{\delta_\Gamma}}
        {\freshOne{x}{\MathSame}{\delta_x}}
      }
      \and
      {\turnstile
        { \dcontext { e } { \delta_E }
          { \Gamma }
          { \delta_\Gamma }
        }
        { \repair { \tau_1 } { \delta_{\tau_1} } { \MathType } { \MathSame } }
      }
      \\
      {\turnstile
        { \dcontext { e } { \delta_E }
          { \MathCons{\MathLocalAssum{x}{\tau_1}}{\Gamma} }
          { \MathMod{\MathLocalAssum{\delta_x}{\delta_{\tau_1}}}{\delta_\Gamma} }
        }
        { \repair { \tau_2 } { \delta_{\tau_2} } { \MathType } { \MathSame } }
      }
    }
    {\turnstile
      { \dcontext{E}{\delta_E}{\Gamma}{\delta_\Gamma} }
      { \repairTermWithoutType{ \MathPi{\tau_1}{x}{\tau_2} }
        { \MathModPi{\delta_{\tau_1}}{\delta_x}{\delta_{\tau_2}} }
      }
    }
  }

  {
    \inferrule*
    [lab=\UTRMatch]
    {
      {\turnstile
        { \dcontext{E}{\delta_E}{\Gamma}{\delta_\Gamma} }
        { \repairTermWithoutType{t}{\delta_t} }
      }
      \and
      {\turnstile
        { \dcontext{E}{\delta_E}{\Gamma}{\delta_\Gamma} }
        { \repairBranches{t}{[b_1 \ldots b_n]}{\delta_b} }
      }
    }
    {\turnstile
      {\dcontext{E}{\delta_E}{\Gamma}{\delta_\Gamma}}
      {\repairTermWithoutType
        {\MathMatch{t}{b_1 \ldots b_n}}
        {\delta}
      }
    }
  }

  {
    \inferrule*
    [lab=\UTRAnnot]
    {
      {\turnstile
        { \dcontext{E}{\delta_E}{\Gamma}{\delta_\Gamma} }
        { \repairTermWithoutType{\tau}{\delta_\tau} }
      }
      \and
      {\turnstile
        { \dcontext{E}{\delta_E}{\Gamma}{\delta_\Gamma} }
        { \repair{t}{\delta_t}{\tau}{\delta_\tau} }
      }
    }
    {\turnstile
      { \dcontext{E}{\delta_E}{\Gamma}{\delta_\Gamma} }
      { \repairTermWithoutType{\MathAnnot{t}{\tau}}{\MathAnnot{\delta_t}{\delta_\tau}} }
    }
  }

  % {
  %   \inferrule*
  %   [lab=\UTRType]
  %   { u \in \{ \MathProp, \MathSet, \MathType \} }
  %   {\turnstile
  %     { \dcontext{E}{\delta_E}{\Gamma}{\delta_\Gamma} }
  %     { \repairTermWithoutType{u}{\MathSame} }
  %   }
  % }

  {
    \inferrule*
    [lab=\UTROtherwise]
    {  }
    {\turnstile
      {\dcontext{E}{\delta_E}{\Gamma}{\delta_\Gamma}}
      {\repairTermWithoutType{t}{\MathSame}}
    }
  }

  \end{mathpar}

\end{Rules}



