\subsection{Syntax of \Chick{} terms}\label{chick-syntax-terms}

\Chick{} is designed as a small functional programming language upon which the
repair algorithm will operate.  The aim was to represent a language such as
\Gallina{}, and as such, \Chick{} is a dependently-typed lambda calculus, with
inductive constructions.  The abstract syntax of \Chick{} terms is given in the
following grammar:

\begin{grammar}%
<term> ::= \ %trick LaTeX
\alt <term> \coqinline{:} <term>                         \hfill (type annotation)
\alt <term> <term>                                       \hfill (function application)
\alt \coqinline{_}                                       \hfill (hole)
\alt \coqinline{λ} <binder> \coqinline{,} <term>         \hfill (value abstraction)
\alt \coqinline{match} <term> \coqinline{with}
$\overline{ \synt{pattern} }$ \coqinline{end}            \hfill (pattern matching)
\alt \coqinline{Π(} <binder> \coqinline{:} <term>
\coqinline{) →} <term>                                   \hfill (type abstraction)
\alt <universe>                                          \hfill (universes)
\alt <var>                                               \hfill (variable)

<universe> ::= \coqinline{Prop} | \coqinline{Set} | \coqinline{Type}

<binder> ::= \ % trick LaTeX
\alt <var> \hfill (named)
\alt _     \hfill (anonymous)

<pattern> ::= ($\overline{ \synt{binder} }$, <term>)
\end{grammar}

The \emph{function application}, \emph{value abstraction}, \emph{type
abstraction}, and \emph{variable} constructs, are standard constructs of a
dependently-typed lambda calculus.

The \emph{type annotation} construct lets us ascribe a type to a given term.
The \emph{hole} construct can take the place of a term anywhere: it stands for a
missing value that should be filled by the programmer.  Together, these two
constructs let us have \define{typed holes}, that is, placeholder values whose
type is known.  This proves invaluable throughout the repair algorithm, as we
will see, since there are situations where the algorithm must come up with
arbitrary values, knowing only the type they should bear.  In the absence of
synthesis techniques to generate such values, the algorithm can safely insert a
typed hole, leaving to the programmer the task of figuring out the proper value.

Finally, the \emph{pattern matching} construct is found in languages with
algebraic data types, and lets us dispatch code based on the shape of some such
data type.  The attentive reader may notice that our pattern language is
somewhat limited.  We do not cover nested patterns, or other advanced pattern
features.

We use the same universe names as found in \Gallina{}, though we do not yet
support its cumulative universe hierarchy syntactically.  This does \emph{not}
mean that \Chick{} cannot repair \Gallina{} programs that use the universe
hierarchy, but rather, that it cannot syntactically represent universe levels.
Instead, \Chick{} is oblivious to universe levels, so it will repair programs
regardless of their universe level.  The downside is that, because it cannot
represent universe levels explicitly, it might break programs that require them.
Bare support for universe levels can be added almost for free, since the repair
algorithm could simply carry them along and attempt no repair.  A universe-aware
repair algorithm would require further investigation.

\paragraph{About underscores} While we use the underscore symbol ($\_$) to
represent both a hole term and an anonymous binder, the two concepts can never
appear in the same program location, and therefore, there can be no confusion:
in a term context, the symbol represents a hole, while in a binding context, it
represents an anonymous binder.
