\chapter{Related work}

\subsection*{Refactoring functional programs}

A main line of work in refactoring functional programs is the work done on the
HaRe refactoring tool for \Haskell{}~\citep{li2005haskell} and subsequently on
refactoring \Erlang{} programs~\citep{li2006}.  The scope of the former is much
larger than this paper: they implement a large library of refactoring
algorithms, including some that this paper covers, for the \Haskell{}
programming language.  In particular, their tool handles \Haskell{} 98, and
provides an API to perform refactorings.  They also refactor while preserving
the user's syntactic choices, which proves challenging in layout-based languages
like \Haskell{}.  However, their publications do not distill the details of how
to implement those refactorings, and thus, does not allow for easy
experimentation, porting to other languages, or verification of their
techniques.

A very promising ongoing line of work is that of
\define{ornaments}~\citep{williams2017}.  Their formalism allows to describe
data type transformations from within the language itself, and perform similar
transformations to ours.  The ornamentation language is more expressive than our
language of diffs in many ways, allowing arbitrary mappings between constructors
before and after modification: for instance, one of their motivating example
cuts a constructor into two different constructors depending on some Boolean
condition.  As a result, their technique requires the programmer to describe the
transformation they wish to carry as an ornament, and the programmer's help is
required alongside the transformation to indicate to the repair algorithm what
to do in corner cases.  Our restriction to a simple language of modifications
allows us to guess the transformation based on the syntactic modification of the
program itself, and grants us more leverage on automation, at the price of a
much lower expressiveness.

\subsection*{Refactoring proofs}

One of the few tools that attempts refactoring in the context of the proof
assistants is the Levity tool by~\citet{ruegenberg2011}.  The refactoring it
performs is referred to as \define{levitation}, and corresponds to the move of a
lemma up its dependency chain, to a location that is deemed more relevant.  This
transformation is hard to perform in many proof assistants, because figuring out
where the lemma can exist can be challenging and time-consuming, and because the
action of levitating the lemma might derail other parts of the proof
development, even when they seem unrelated, because the declaration of a lemma
might have unintended side-effects through automation mechanisms.

Another main line of work in this domain is the line of work
of~\citet{whiteside2011}, followed up by~\citet{dietrich2013}.  They define a
framework for specifying refactoring rules over an idealized proof language
called Hiproof, and its idealized tactic language called Hitac.  The latter is
designed to be close to the declarative Isar tactic language of Isabelle.  This
framework allows them to define many interesting refactorings, and to prove
their correctness, informally, with low effort.  They acknowledge that it would
be more difficult to develop the same refactorings for a less declarative tactic
language like the \Ltac{} language of the \Coq{} proof assistant.

\citet{boite2004} also addresses the problem of modifying an existing inductive
type definition while reusing existing proofs.  Their technique does not alter
existing code, rather, they internalize the notion of \define{extension} of an
inductive data type (that is, the adding of a constructor, or parameters), and
provide commands that perform such extensions, deriving extended data types and
extended functions from the original ones.  It would be interesting to have
proof assistants internalizing and exposing this kind of algebraic manipulations
of data types.

Finally, recent work by~\citet{ringer2018adapting} builds semantic patches to
help programmer refactor their proofs.  Their approach does not modify the old
program into a new program.  Rather, the old program is kept with its proofs,
alongside the new program, and patches are automatically derived to transport
proof obligations between the old program and the new program.  Their approach
seems complementary to ours, depending on the use case for the programmer.

\subsection*{Automatic code generation}

Another exciting line of work consists in removing the friction of performing
those manual refactorings altogether.

General purpose automation like Isabelle's sledgehammer~\citet{blanchette2011}
or the recent for on a similar hammer for Coq~\citet{czajka2017} can help reduce
the work of programmers in those systems by avoiding writing terms entirely.
Hopefully, doing so removes a lot of brittle tactic calls, but they do not help
in refactoring data type definitions when the objects of discourse evolve.

The body of work on program synthesis could also prove very beneficial.  For
instance,~\cite{polikarpova2016} can perform program synthesis from refinement
types, which could potentially spare the programmer from fixing code that is
never manually-written, but rather synthesized.  However, the specifications
themselves could still require refactoring tools, since they are themselves
terms from a refinement type system.

\subsection*{IDE-based detection of refactoring attempts}

The work of~\citet{foster2012} focuses on detecting when a programmer is trying
to perform a refactoring in Java programs.  Many times, the programmer will not
know that refactoring tools exist in their IDE, or will not realize that they
are performing a task that corresponds to an existing refactoring algorithm.  In
particular, they use a text-based diff in order to be able to recognize
refactoring attempts even when the program is in a transient, non-parseable
state.

\subsection*{Incremental computation}

The literature on incremental computation focuses on the problem of reliably and
efficiently re-computing data that is the output of some computation after the
input undergoes a modification, which is similar to our deriving of repairing
functions from Section~\ref{chick-deriving}.  For instance,~\citet{acar2006},
\citet{carlsson2002}, as well as~\citet{firsov2016}, all use an abstract data
type for values that can be efficiently recomputed, and propose a monadic
interface to build incremental computations while providing the programmer with
an interface that hides those details behind the scenes.  Similarly,
\citet{chen2011} automatically transform a non-incremental program to inject
incremental behavior into it.  While this is similar to what we do in spirit, we
care about more than just obtaining the result of the incremental computation,
since we sometimes reuse the diff of one data type to compute a diff for
another, related data type.  Incremental computation techniques are typically
not interested in this, and do not usually expose the internals of their
functioning to the user.

\citet{cai2014} define a general way of using \define{derivatives} of functions
to compute their reactions to some abstract modifications, and compute those
derivatives for first-order functions.  They demonstrate it on examples using
unordered collections.  Our diffs are an instance of their general notion, but
we depart from it by having more specialized diffs to fit our needs, and by
supporting a small family of higher-order functions (namely, folds over lists),
using our deriving technique to help compute their derivative.


% \section*{HaRe: a \Haskell{} refactoring tool}

% \mycite{li2005haskell} propose a refactoring tool for \Haskell{} programs that
% provide several of the benefits of the tool we describe in
% Chapter~\ref{chick}.


% \section*{Ornaments}

% In order to refactor functional programs, as well as enabling novel methods of
% programming with data types, the work on \define{ornaments} approaches the
% treatment of programs as mathematical objects that can be related algebraically.

% \mycite{dagand2017essence} describes a theoretical framework for describing data
% types in such a way as to be able to express relationships between two data type
% definitions.  By representing inductive data type definitions with their
% underlying many-sorted signature, their work allows describing the relationship
% between two data types, describing one as an extension of the signature of the
% other one.  \mycite{williams2014ornaments} show how to use ornaments to perform
% a refactoring in which a data type representation gets reorganized without
% adding or removing information.

% Their approach is much more general and structured than ours: their work can
% effectively lift existing code through ornaments, or be used to express the
% relationship between a surface language and its syntactically-desugared version
% as an ornament.  However, this power comes with a price: first, the programmer
% must write their refactoring attempt in terms of an ornament, and then, they
% must participate in the lifting process by indicating to the algorithm what to
% do in some corner cases.  In our approach, the limited expressiveness of our
% diffs allows an algorithm to perform guesses as to what changed between two
% definitions, relieving the user from having to specify it.  We do, however, have
% similar limitations to our algorithm's ability to deal with ambiguous
% refactoring attempts, and we also rely on typed holes to fill blanks where the
% programmer must provide values to finish the generated code.

% \section*{Patching proof terms}

% \mycite{ringer2018adapting} propose a technique for automatically inferring
% proof patches based on an example of one patch.  This proves invaluable in
% refactoring settings, where a user might need to repeatedly apply the same
% repairing technique to several places in a code base.  By semantically analyzing
% one instance of such a repair, and performing abstraction techniques to make the
% repair more general, they can yield a patch that can be instantiated in many
% places to carry out the refactoring.

% Their current prototype does not support changes that don't preserve shape yet,
% so they could not be patching examples that ornaments and our technique handles.
% Their technique seems orthogonal to ours, since their intent is to obtain the
% new, patched proof as a proof term with the correct type, not as a repaired
% tactic script.  For repairing programs, this is similar, but for repairing
% proofs, their approach would yield a repaired proof object, but not a repaired
% proof text.  With our approach, we'd hope to be able to give tactic repairs, at
% the syntax level.

% However, the great advantage of their approach comes from the ability to find
% semantic, meaningful patches from examples.  While this requires some work from
% the user, they hope that the time saved by applying the patch repeatedly overall
% outweighs the time investment.
