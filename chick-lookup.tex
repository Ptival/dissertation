\section{Lookup rules}

When encountering a variable in a program under repair, we will often need to
answer the question ``What has happened to this variable?''.  In order to do so,
we generalize the usual notion of variable lookup.  In presentations of lambda
calculi where both a global environment and a local context carry typing
information, the usual notion of variable lookup first looks in the local
context (since it carries the ``closest'' enclosing scopes), then, if no binding
is found, it looks in the global environment.

We use the following judgments for looking up a variable in the local context,
on the left, and in the global environment, on the right:~\footnote{While the
two judgments share the same syntax, it should always be clear whether the value
to the left of the turnstile is a local context or a global environment.}

\[
  \turnstile{\Gamma}{ \hasType{ v }{ \out{\tau} } }
  \hspace{1cm}
  \turnstile{E}{ \hasType{ v }{ \out{\tau} } }
\]

The rules for looking up types are straightforward, as shown in
Figures~\ref{lookup-type-dispatch},~\ref{lookup-type-context},
and~\ref{lookup-type-environment}.  In the latter, TODO TODO TODO

\begin{Rules}{lookup-type-dispatch}{ Lookup rules (local context and global environment) }

\begin{mathpar}
  {
    \inferrule*
    [lab=\LookupCtxt]
    { \turnstile{ \Gamma }{ \hasType{ v }{ \out{\tau} } } }
    { \turnstile{ \context{E}{\ \Gamma} }{ \hasType{ v }{ \out{\tau} } } }
  }

  {
    \inferrule*
    [lab=\LookupEnv]
    {
      \nturnstile{ \Gamma }{ \hasType{ v }{ \out{\tau} } }
      \and
      \turnstile{ E }{ \hasType{ v }{ \out{\tau} } }
    }
    { \turnstile{ \context{E}{\ \Gamma} }{ \hasType{ v }{ \out{\tau} } } }
  }

\end{mathpar}

\end{Rules}

\begin{Rules}{lookup-type-context}{ Lookup rules (local context) }

\begin{mathpar}
  {
    \inferrule*
    [lab=TODO]
    {  }
    { \turnstile{ \MathCons{\MathLocalAssum{v}{\tau}}{\Gamma} }{ \hasType{ v }{ \out{\tau} } } }
  }

  {
    \inferrule*
    [lab=TODO]
    { w \neq v \and \turnstile{ \Gamma }{ \hasType{ v }{ \out{\tau} } } }
    { \turnstile{ \MathCons{\MathLocalAssum{w}{\tau}}{\Gamma} }{ \hasType{ v }{ \out{\tau} } } }
  }

\end{mathpar}

\end{Rules}

\begin{Rules}{lookup-type-environment}{ Lookup rules (global environment) }

\begin{mathpar}

  {
    \inferrule*
    [lab=TODO]
    { w \neq v \and \turnstile{ E }{ \hasType{ v }{ \out{\tau} } } }
    { \turnstile{ {\MathCons{\Definition{k}{v}{\tau}{t}}{E}} }{ \hasType{ v }{ \out{\tau} } } }
  }

\end{mathpar}

\end{Rules}









Our second notion of a lookup essentially follows the same strategy, but we are
\emph{\textbf{not}} interested in finding the type of the variable, but rather
in knowing how \emph{both} the binding \emph{and} the binding type have been
modified.  To do so, we use the following judgment:

{
  \[
    \turnstile
    {\dcontext {E} {\delta_E} {\Gamma} {\delta_\Gamma} }
    { \declDiff { v } { \delta_v } { \delta_{\tau} } }
  \]
}

\paragraph{Notation} This judgment can be read as follows:
\begin{itemize}
\item \textbf{Top}: If you are interested in the variable $v$, as bound in the original global environment $E$ and original local context $\Gamma$,
\item \textbf{Bottom left}: and the global environment underwent modification $\delta_{E}$,
\item \textbf{Bottom left}: and the local context underwent modification $\delta_{\Gamma}$,
\item \textbf{Bottom right}: then the variable underwent modification
$\out{\delta_{v}}$,
\item \textbf{Bottom right}: and its type underwent modification
$\out{\delta_{\tau}}$.
\end{itemize}

The rules in Figure~\ref{rules-lookup} summarize the global strategy of first
looking up in the local context, and then falling back to the global
environment:

\begin{Rules}{rules-lookup}{ Lookup rules (local context and global environment) }

\begin{mathpar}
  {
    \inferrule*
    [lab=\LookupCtxt]
    {\turnstile%
      { \diff{\Gamma}{\delta_\Gamma} }
      { \declDiff{ v }{ \delta_v }{ \delta_{\tau} } }
    }
    {\turnstile%
      {\dcontext{E}{\delta_E}{\Gamma}{\delta_\Gamma} }
      { \declDiff{ v }{ \delta_v }{ \delta_{\tau} } }
    }
  }

  {
    \inferrule*
    [lab=\LookupEnv]
    {
      {\turnstile%
        { \diff{E} {\delta_E} }
        { \declDiff{ v } { \delta_v } { \delta_{\tau} } }
      }
    }
    {\turnstile%
      { \dcontext{E} {\delta_E} {\Gamma} {\delta_\Gamma} }
      { \declDiff{ v } { \delta_v } { \delta_{\tau} } }
    }
  }

\end{mathpar}

\end{Rules}

TODO Blah blah blah

\begin{Rules}{rules-lookup-context}{ Lookup in the local context }

  \begin{mathpar}
    {
      \inferrule*
      [lab=\LCSame]
      {
        { \turnstile {\Gamma} {\hasType { v } { \out{\tau_v} } } }
      }
      {\turnstile
        { \diff {\Gamma} {\MathSame} }
        { \declDiff { v } { \MathSame } { \MathSame } }
      }
    }

    {
      \inferrule*
      [lab=\LCIns]
      {\turnstile
        { \diff {\Gamma} {\delta_{\Gamma}} }
        { \declDiff { v } { \delta_v } { \delta_{\tau_v} } }
      }
      {\turnstile
        { \diff {\Gamma} {\MathIns{ \texttt{\_} }{\delta_{\Gamma}}} }
        { \declDiff { v } { \delta_v } { \delta_{\tau_v} } }
      }
    }

    % {
    % \inferrule*
    % [lab=\LCKeep]
    % {\turnstile
    % { \diff {\Gamma} {\MathMod{\MathSame}{\delta_{\Gamma}}} }
    % { \declDiff { v } { \delta_v } { \delta_{\tau_v} } }
    % }
    %   {\turnstile
    %   { \diff {\Gamma} {\MathKeep{\delta_{\Gamma}}} }
    %   { \declDiff { v } { \delta_v } { \delta_{\tau_v} } }
    % }
    % }

  \end{mathpar}

  \begin{mathpar}
    {
      \inferrule*
      [lab=\LCMod{$=$}]
      {
      }
      {\turnstile
        { \diff {\MathCons{\MathLocalAssum{v}{\tau}}{\Gamma}} {\MathMod{(\delta_v : \delta_{\tau})}{\delta_{\Gamma}}} }
        { \declDiff { v } { \delta_v } { \delta_{\tau} } }
      }
    }

    {
      \inferrule*
      [lab=\LCMod{$\neq$}]
      {
        v' \neq v
        \and
        {\turnstile
          { \diff {\Gamma} {\delta_{\Gamma}} }
          { \declDiff { v } { \delta_v } { \delta_{\tau} } }
        }
      }
      {\turnstile
        { \diff {\MathCons{\MathLocalAssum{v'}{\tau}}{\Gamma}} {\MathMod{(\delta_v : \delta_{\tau})}{\delta_{\Gamma}}} }
        { \declDiff { v } { \delta_v } { \delta_{\tau} } }
      }
    }

  \end{mathpar}

  % \begin{mathpar}
  %   {
  %     \inferrule*
  %     [lab=\LCMod{Def-$=$}]
  %     {
  %     }
  %     {\turnstile
  %       { \diff {\MathCons{\MathLocalDef{v}{\tau}{t}}{\Gamma}} {\MathMod{\MathLocalDef{\delta_v}{\delta_{\tau}}{\delta_t}}{\delta_{\Gamma}}} }
  %       { \declDiff { v } { \delta_v } { \delta_{\tau} } }
  %     }
  %   }

  %   {
  %     \inferrule*
  %     [lab=\LCMod{Def-$\neq$}]
  %     {
  %       v' \neq v
  %       \and
  %       {\turnstile
  %         {\diff{\Gamma}{\delta_{\Gamma}}}
  %         {\declDiff{v}{\delta_v}{\delta_{\tau}}}
  %       }
  %     }
  %     {\turnstile
  %       {\diff
  %         {\MathCons{\MathLocalDef{v'}{\tau}{t}}{\Gamma}}
  %         {\MathMod{\MathLocalDef{\delta_v}{\delta_{\tau}}{\delta_t}}{\delta_{\Gamma}}}
  %       }
  %       {\declDiff{v}{\delta_v}{\delta_{\tau}}}
  %     }
  %   }

  % \end{mathpar}

\end{Rules}
