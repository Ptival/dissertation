\section{Avoiding problematic binders using scope sets}

Our treatment of binders in Section~\ref{chick-design-repair-term} is extremely
cumbersome.  Not only do we have to be careful about capture-avoiding
substitutions, both in the old program and the new program, but we also have to
be careful to repair every term only once, because the repair operation is
\emph{not} idempotent.

A novel approach from~\mycite{flatt2016binding} developed for the new Racket
macro expander might be useful here.  In their setting, every scope created is
assigned a label, and every reference to a bound variable is initially marked
with the set of scopes within which it appears in the original program text.
Subsequently, the macro expander discards the renaming approach entirely.
Instead, every binding form, as well as every macro expansion, creates a scope,
adds that scope to all identifiers in binding position, and keeps track of a
mapping from identifiers with scope sets to the representation of a binding.

Subsequently, the macro expander is free to perform expansions without need for
capture-avoiding freshness conditions, or, as it is called in the macro
expansion literature, \define{hygienic macro
expansion}~\mycite{kohlbecker1986hygienic}.  Instead, a reference's binding is
found by looking in the mapping for the binder with same name whose scope of
sets is the largest subset of the reference.

We believe this approach could potentially be adapted to solve the problems we
encounter.  Our first issue (demonstrated in Rule \rulename{\RModPi}) was that
of an outer binding becoming shadowed by a binding position, when said outer
binding is renamed by the programmer to have the same name as the binding
position.  For instance, in the following code:

\begin{figure*}[htp]
\noindent
\cprotect\fbox{
\begin{minipage}[t]{0.50\textwidth-2\fboxrule-2\fboxsep}%
\begin{minted}[linenos=false,escapeinside=@@]{coq}
Definition f := … @\dummystrut{ }@

… (λ g → f g) …
\end{minted}
\end{minipage}}%
\cprotect\fbox{
\begin{minipage}[t]{0.50\textwidth-2\fboxrule-2\fboxsep}%
\begin{minted}[linenos=false,escapeinside=@@]{coq}
Definition @\modified{g}@ := …

… (λ g → f g) …
\end{minted}
\end{minipage}}%
\end{figure*}

\noindent%
%
we could not simply rename the variable \coqinline{f} on the right side to
\coqinline{g}, because it would accidentally become captured by the term
abstraction.  We needed to be very careful about this problem in our rules.
With a ``set of scopes'' approach, we would instead have:

\begin{figure*}[htp]
\noindent
\cprotect\fbox{
\begin{minipage}[t]{0.50\textwidth-2\fboxrule-2\fboxsep}%
\begin{minted}[linenos=false,escapeinside=@@]{coq}
Definition f@$^{\{a\}}$@ := … @\dummystrut{ $^{\phantom{\{a, m\}}}$}@

… (λ g@$^{\{a, b\}}$@ → f@$^{\{a, b\}}$@ g@$^{\{a, b\}}$@) …
\end{minted}
\end{minipage}}%
\cprotect\fbox{
\begin{minipage}[t]{0.50\textwidth-2\fboxrule-2\fboxsep}%
\begin{minted}[linenos=false,escapeinside=@@]{coq}
Definition @\modified{g$^{\{a, m\}}$}@ := …

… (λ g@$^{\{a, b\}}$@ →  f@$^{\{a, b\}}$@ g@$^{\{a, b\}}$@) …
\end{minted}
\end{minipage}}%
\end{figure*}

\noindent%
%
where scope label $a$ corresponds to the scope created by the definition of
\coqinline{f}, scope label $b$ corresponds to the scope created by the term
abstraction introducing \coqinline{g}, and scope label $m$ corresponds to a
virtual scope that we introduce to mark where the user has modified their
program.  As a result, in this approach, we can repair the program on the right
with simple substitution, yielding:

\begin{figure*}[htp]
\noindent
\cprotect\fbox{
\begin{minipage}[t]{0.50\textwidth-2\fboxrule-2\fboxsep}%
\begin{minted}[linenos=false,escapeinside=@@]{coq}
Definition f@$^{\{a\}}$@ := … @\dummystrut{ $^{\phantom{\{a, m\}}}$}@

… (λ g@$^{\{a, b\}}$@ → f@$^{\{a, b\}}$@ g@$^{\{a, b\}}$@) … @\dummystrut{ $^{\phantom{\{a, m\}}}$}@
\end{minted}
\end{minipage}}%
\cprotect\fbox{
\begin{minipage}[t]{0.50\textwidth-2\fboxrule-2\fboxsep}%
\begin{minted}[linenos=false,escapeinside=@@]{coq}
Definition @\modified{g$^{\{a, m\}}$}@ := …

… (λ g@$^{\{a, b\}}$@ →  @\repaired{g$^{\{a, m\}}$}@ g@$^{\{a, b\}}$@) …
\end{minted}
\end{minipage}}%
\end{figure*}

\noindent%
%
where the inner mention of \coqinline{g} with scope set $\{a, m\}$ is \emph{not}
captured by the term abstraction, because its set of scopes mentions scope label
$m$, preventing it from being a subset of the enclosing $\{a, b\}$ set.  We will
still need to resolve the name collision when producing the concrete repaired
program, but as far as the repair algorithm is concerned, there is no need to
pick a fresh name for the binder, as it can \emph{not} capture the repaired
variable.

The second problem we had with binders, still in Rule ~\rulename{RModPi}, was to
pick a name that was both fresh for the term abstraction, and fresh for the type
abstraction, so that we could add this name to the typing context, as we
proceeded to repair the term abstraction's body, assuming its type was the type
abstraction's body.  Here again, we could circumvent the renaming altogether by
introducing a new scope label.

If we decide to attach scope labels to any term of the abstract syntax tree,
rather than only variables, we could even consider using scope labels as
markers, separating terms that have been repaired from terms that have not been
repaired yet.
