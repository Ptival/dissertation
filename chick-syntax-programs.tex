\subsection{Syntax of \Chick{} programs}\label{chick-syntax-programs}

Our programs are defined as sequences of commands in the following vernacular
(to borrow Coq's parlance):

\begin{grammar}
<program> ::= $\overline{ \synt{vernacular} }$

<vernacular> ::= \ %trick LaTeX
\alt \coqinline{Definition}(\{ kind : <definition-kind> , name : <var>, type : <term>, body : <term> \})
\alt \coqinline{Inductive}(<inductive>)

<definition-kind> ::= \coqinline{Definition} | \coqinline{Fixpoint}
\end{grammar}

Our \define{definition kinds} follow \Coq{}'s \Vernacular{} conventions: a
\coqinline{Definition} is \emph{never} recursive, while a \coqinline{Fixpoint}
is \emph{allowed} to be recursive.  Declarations are ordered, and may only
depend on previous declarations.  This will ease the task of our repair
algorithm, since dependencies are syntactically ordered.  In a language where
declarations are allowed to depend on other declarations backward \emph{and}
forward, we would want to repair programs according to the dependency tree
rather than the syntactic order of declarations.

Inductive definitions are not mutual, and follow the following syntax:

\noindent
\begin{grammar}
<inductive> ::= Inductive(\{\\
name : \synt{var},\\
parameters : $\overline{ ( \synt{var} , \synt{term} ) }$,\\
indices : $\overline{ ( \synt{binder} , \synt{term} ) }$,\\
universe : \synt{universe},\\
constructors : $\overline{ \synt{constructor} }$\\
\})

<constructor> ::= Constructor(\{\\
name : \synt{var},\\
parameters : $\overline{ ( \synt{binder}, \synt{term} ) }$,\\
indices : $\overline{ \synt{term} }$\\
\})
\end{grammar}
