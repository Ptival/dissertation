\subsection{Syntax of \Chick{} programs}\label{chick-syntax-programs}

Our programs are defined as sequences of commands in the following vernacular
(to borrow Coq's parlance):

\begin{grammar}
<program> ::= $\overline{ \synt{vernacular} }$

<vernacular> ::= \ %trick LaTeX
\alt \coqinline{Definition}(\{ kind : <definition-kind> , name : <var>, type : <term>, body : <term> \})
\alt \coqinline{Inductive}(<inductive>)

<definition-kind> ::= \coqinline{Definition} | \coqinline{Fixpoint}
\end{grammar}

Our \define{definition kinds} follow \Coq{}'s \Vernacular{} conventions: a
\coqinline{Definition} is \emph{never} recursive, while a \coqinline{Fixpoint}
is \emph{allowed} to be recursive.  Declarations are ordered, and may only
depend on previous declarations.  This will ease the task of our repair
algorithm, since dependencies are syntactically ordered.  In a language where
declarations are allowed to depend on other declarations backward \emph{and}
forward, we would want to repair programs according to the dependency tree
rather than the syntactic order of declarations.

Inductive definitions are not mutual, and follow the following syntax:

\noindent
\begin{grammar}
<inductive> ::= \coqinline{Inductive}(\{\\
name : \synt{var},\\
parameters : $\overline{ ( \synt{var} , \synt{term} ) }$,\\
indices : $\overline{ ( \synt{binder} , \synt{term} ) }$,\\
universe : \synt{universe},\\
constructors : $\overline{ \synt{constructor} }$\\
\})

<constructor> ::= \coqinline{Constructor}(\{\\
name : \synt{var},\\
parameters : $\overline{ ( \synt{binder}, \synt{term} ) }$,\\
indices : $\overline{ \synt{term} }$\\
\})
\end{grammar}

For a presentation of inductive data type declarations, we refer the reader
to~\mycite{coquand1990inductively} for an academic description, or
to~\mycite{pierce2010software} for some educational material.  We will briefly
recall the difference between \define{inductive parameters}, \define{inductive
indices}, \define{constructor parameters}, and \define{constructor indices}.

\define{Inductive parameters} indicate that a type is generic in some input
type.  Most often, this type parameter will be a type that the constructors may
contain instances of, or operate on values of.  For instance, many containers
will have a carrier type for the type of values they may contain.  The choice of
type to be contained does \emph{not} restrict the shapes that the container may
take: we say that it behaves \emph{parametrically}.  For instance, the type of
list of boolean values, \coqinline{list bool}, and the type of list of natural
number values, \coqinline{list nat}, contain the same ``shapes'':
\coqinline{[]}, \coqinline{_ ∷ []}, \coqinline{_ ∷ _ ∷ []}, etc.  The choice of
the type parameter only tells us what values we are allowed to put in those
holes, but does not dictate which of those shapes we may or may not instantiate.

On the other hand, \define{inductive indices} can affect what inhabitants we may
find in an inductive type.  Instead of being called an inductive type, one that
is indexed is often instead called an \define{inductive family of types}, or
\define{inductive type family}, to capture the idea that the choice of index has
an effect on what inhabitants may exist.  Knowing the index of an inductive
value gives us information about what shape it may have.  For instance, we could
build a type of lists indexed by whether their length is odd or even as:%
%
\footnote{\Coq{} syntax does not let us define constructors that look like
operators, but we do it here to keep the code clear and concise.}%
%
\textsuperscript{,}%
%
\footnote{In constructor \coqinline{(∷)}, we use curly braces \coqinline{p} to
mark the constructor parameter as implicit, indicating it can be implicitly
inserted unambiguously by Coq when the next constructor parameter,
\coqinline{t}, is supplied.}

\begin{minted}{coq}
Inductive parity : Type :=
| Even : parity
| Odd  : parity
.

Definition next_parity (p : parity) : parity :=
  match p with
  | Even => Odd
  | Odd  => Even
  end.

Inductive parity_list (T : Type) : parity → Type :=
| []  : parity_list T Even
| (∷) : ∀ (h : T) {p : parity} (t : parity_list T p) →
        parity_list T (next_parity p)
.
\end{minted}

Now, if we consider the type family \coqinline{parity_list T p} as a whole, for
a given choice of carrier type \coqinline{T}, the possible shapes of values are
still the same as for \coqinline{list}: \coqinline{[]}, \coqinline{_ ∷ []},
\coqinline{_ ∷ _ ∷ []}, etc.  However, for a given choice of \coqinline{p}, the
valid shapes become restricted by construction.  The type \coqinline{parity_list
T Even} only contains shapes \coqinline{[]}, \coqinline{_ ∷ _ ∷ []},
\coqinline{_ ∷ _ ∷ _ ∷ _ ∷ []}, etc.  Dually, the type \coqinline{parity_list T
Odd} only contains shapes \coqinline{_ ∷ []}, \coqinline{_ ∷ _ ∷ _ ∷ []}, etc.

In general, different families may have entirely disjoint shapes, as presented
here, or partially-overlapping shapes (for instance, the type family of finite
sets), or even all the same shapes (in which case, the index might be used as a
type-level marker for some information, a technique often called \coqinline{phantom
type} in statically-typed languages).

\define{Constructor parameters} are just arguments that are passed to a given
constructor.  They are completely independent for each constructor, and are
simply used to store data relevant to the given constructor.  For instance, in
our \coqinline{parity_list} example, \coqinline{h}, \coqinline{p}, and
\coqinline{t} are three constructor parameters for constructor \coqinline{(∷)}.

The return type of a constructor must indicate in what family is exists.  To do
so, it must provide values for the inductive indices.  We call those values the
\define{constructor indices}.  For instance, in our \coqinline{parity_list}
example, the inductive index of type \coqinline{parity} is instantiated with
value \coqinline{(next_parity p)} in the \coqinline{(∷)} constructor.  Note that
it is possible to have indices be computed values, as is the case here.

Finally, let us discuss scope for those constructs.

\begin{itemize}

\item Following \Coq{}'s convention, inductive parameters \emph{must} be named.
They are naturally in scope for the whole definition of the inductive type.

\item Inductive indices \emph{may} be anonymous, or named, and they form a
telescope (as described in Section~\ref{dependenty-types}).  They are \emph{not}
scoped over the rest of the definition.  Therefore, the only purpose for named
indices is to create dependent indices.

\item Constructor parameters \emph{may} be anonymous, or named.  They also form
a telescope all the way to the constructor's return type, such that both
subsequent parameters, as well as the constructor's indices, may refer to them.

\item Constructor indices \emph{must} be passed as explicit arguments to the
return type, and as such, there is no binding involved.

\end{itemize}
