\section{Deriving repair functions}~\label{deriving}

Accounting for changes in inductive definitions requires quite a bit of work:

\begin{itemize}

\item the inductive type itself must be repaired, including its parameters, and
its indices,

\item each constructor must be repaired, including their parameters, and their
instantiation of indices,

\item the automatically-generated elimination principles for the type must also
be repaired accordingly.

\end{itemize}%
%
Let us focus on the latter.  In a proof assistant like Coq, when an inductive
data type \coqinline{t} is defined, the system also defines a family of
\textit{elimination principles} \coqinline{t_ind}, \coqinline{t_rec}, and
\coqinline{t_rect}, that essentially encode an induction principle for the given
type.  For instance, for the types \coqinline{list} and \coqinline{vec}, the
induction principles are given in Figure~\ref{list-ind-vec-ind}.

\noindent
\begin{figure*}[!htp]

  \noindent%
  \begin{minipage}[t]{0.45\textwidth}
\begin{minted}[linenos=false,escapeinside=@@]{coq}
list_ind :
∀ (A : Type)        @\tikzmark{list1}@
  (P : list A →     @\tikzmark{list2}@
       Prop),
  P (nil A) →       @\tikzmark{list31}@
  (∀ (a : A)        @\tikzmark{list32}@
     (l : list A),
     P l →
     P (cons A a l)
  ) →
                    @\tikzmark{list4}@
  ∀ l : list A,     @\tikzmark{list5}@
    P l             @\tikzmark{list6}@
\end{minted}
\begin{tikzpicture}[remember picture,overlay]
  % \draw[help lines,xstep=.1,ystep=.1] (0,0) grid (8,10);
  % \foreach \x in {0,1,...,9} { \node [anchor=north] at (\x,0) {\x}; }
  % \foreach \y in {0,1,...,9} { \node [anchor=east] at (0,\y) {\y}; }
  \tikzstyle{circlednumber} = [shape=circle,draw,font=\bfseries,inner sep=0pt,thick];
  \node[circlednumber, above=-0.05 of pic cs:list1 ] {1};
  \node[circlednumber, above=-0.05 of pic cs:list2 ] {2};
  \node[circlednumber, above=-0.05 of pic cs:list31] {3};
  \node[circlednumber, above=-0.05 of pic cs:list32] {3};
  \node[circlednumber, above=-0.05 of pic cs:list4 ] {4};
  \node[circlednumber, above=-0.05 of pic cs:list5 ] {5};
  \node[circlednumber, above=-0.05 of pic cs:list6 ] {6};
\end{tikzpicture}
  \end{minipage}%
  \begin{minipage}[t]{0.55\textwidth}
\begin{minted}[linenos=false,escapeinside=@@]{coq}
vec_ind :
∀ (A : Type)                 @\tikzmark{vec1}@
  (P : ∀ n : nat, vec A n →  @\tikzmark{vec2}@
         Prop),
  P 0 (vnil A) →             @\tikzmark{vec31}@
  (∀ (a : A) (n : nat)       @\tikzmark{vec32}@
     (v : vec A n),
     P n v →
     P (S n) (vcons A a n v)
  ) →
  ∀ (n : nat)                @\tikzmark{vec4}@
    (v : vec A n),           @\tikzmark{vec5}@
    P n v                    @\tikzmark{vec6}@
\end{minted}
\begin{tikzpicture}[remember picture,overlay]
  % \draw[help lines,xstep=.1,ystep=.1] (0,0) grid (8,10);
  % \foreach \x in {0,1,...,9} { \node [anchor=north] at (\x,0) {\x}; }
  % \foreach \y in {0,1,...,9} { \node [anchor=east] at (0,\y) {\y}; }
  \tikzstyle{circlednumber} = [shape=circle,draw,font=\bfseries,inner sep=0pt,thick];
  \node[circlednumber, above=-0.05 of pic cs:vec1 ] {1};
  \node[circlednumber, above=-0.05 of pic cs:vec2 ] {2};
  \node[circlednumber, above=-0.05 of pic cs:vec31] {3};
  \node[circlednumber, above=-0.05 of pic cs:vec32] {3};
  \node[circlednumber, above=-0.05 of pic cs:vec4 ] {4};
  \node[circlednumber, above=-0.05 of pic cs:vec5 ] {5};
  \node[circlednumber, above=-0.05 of pic cs:vec6 ] {6};
\end{tikzpicture}
  \end{minipage}

  \caption{ Induction principles for \coqinline{list} and \coqinline{vec}. }~\label{list-ind-vec-ind}

\end{figure*}

The property being returned \coqinline{P} is usually referred to as the
\textit{motive} of the elimination principle.  The motive is a property about a
value of the inductive type being eliminated: we will refer to this value as the
\textit{target} of the motive.  The value being eliminated (the last
universally quantified value, respectively \coqinline{l} and \coqinline{v}) are
usually referred to as the \textit{discriminee}.  The process that computes the
elimination principle's type from the inductive data type definition
$\Inductive{n}{\overrightarrow{p}}{\overrightarrow{i}}{u}{\overrightarrow{c}}$
is involved, but straightforward:

\begin{enumerate}[label=\protect\circled{\arabic*}]

\item~\label{ind-params} the inductive parameters $\overrightarrow{p}$ are
universally quantified over,

\item the motive \coqinline{P} is universally quantified over, its type being
computed by:

\begin{itemize}

\item universally quantifying over the inductive indices $\overrightarrow{i}$,

\item universally quantifying over a target, i.e.\ an element of the inductive
type undergoing elimination, fully-instantiated with parameters
from~\ref{ind-params} and the indices from the previous sub-step,

\item and returning in the appropriate universe for the desired elimination
principle,

\end{itemize}

\item for each constructor
$\Constructor{n_c}{\overrightarrow{p_c}}{\overrightarrow{i_c}}$, a case is
universally quantified over, whose type is obtained thus:

\begin{itemize}

\item each constructor parameter $\overrightarrow{p_c}$ is
universally quantified over, and for each parameter $p_{c_i}$ that is a
recursive occurrence of the data type being eliminated, a fully-instantiated
motive, with appropriate arguments so as to target this parameter, is also
universally quantified over immediately after this parameter,

\item the return type is also a fully-instantiated motive, where inductive
indices are instantiated with the constructor indices $\overrightarrow{i_c}$,
and the motive's target is an instantiation of the constructor $n_c$, with
inductive parameters from step~\ref{ind-params}, and constructor parameters from
the previous sub-step,

\end{itemize}

\item~\label{ind-indices} inductive indices $\overrightarrow{i}$ are
universally quantified over,

\item the discriminee is universally quantified over, its type being the
inductive type applied to inductive parameters from~\ref{ind-params} and
inductive indices from~\ref{ind-indices},

\item finally, the elimination rule returns a value of the fully-instantiated
motive, with indices from~\ref{ind-indices}, and the discriminee as target.

\end{enumerate}
%
We will use Haskell syntax to describe meta-language implementation details (we
reserve Coq syntax for programs in the object language).  Our function computing
the type of an eliminator has the following high-level skeleton:

\noindent
\begin{minted}{haskell}
eliminatorType ... =
  quantifyVars  inductiveParameters                 {- 1 -}
. mkPi          motiveType            motive        {- 2 -}
. quantifyCases inductiveConstructors               {- 3 -}
. quantifyVars  inductiveIndices                    {- 4 -}
. mkPi          discrimineeType       discriminee   {- 5 -}
$ mkApp         indices               discriminee   {- 6 -}
  where
   indices = applyVars inductiveIndices motive
\end{minted}

\noindent where \coqinline{quantifyVariables}, \coqinline{applyVariables} are
folding functions that create \coqinline{Pi}s, and \coqinline{App}s,
respectively, and \coqinline{motiveType} and \coqinline{discrimineeType} are
computed by additional \coqinline{fold} operations.

Our insight is as follows: while we cannot derive a diff-propagating function
for any \coqinline{fold} in general, we can describe how a given folding
operation \coqinline{f} reacts to differences in the input list \coqinline{l} in
the call \coqinline{(fold f l b)}.  From this description, we can derive the
diff-propagating function for the whole \coqinline{fold} operation.  To make
things more concrete, let's focus on \coqinline{quan\-tify\-Variables} and build
its corresponding, diff-propagating version \coqinline{δquan\-tify\-Variables}.
The original function is a right-fold that universally quantifies over every
element encountered. We can describe how it reacts to changes in the input list
using the following record type:

\begin{minted}{haskell}
data ΔListFold τ δτ δ = ΔListFold
  { onInsert  ::     τ →     [τ] → δ → δ
  , onModify  ::    δτ → τ → [τ] → δ → δ
  , onPermute :: [Int] →     [τ] → δ → δ
  , onRemove  ::         τ → [τ] → δ → δ
  , onReplace ::   [τ] →     [τ] → δ → δ
  , onSame    ::             [τ] → δ → δ
  }
\end{minted}

\noindent For each diff operation on lists, we record a transformer on the
output diff type \coqinline{δ}.  In our example, the fold is building a
\coqinline{Term}, so the output diff type is \coqinline{ΔTerm}, and we describe
how the output term is affected by changes to the input list: when an element is
inserted in the input list, a $\Pi$ is added to the output term, when an element
is removed from the input list, a $\Pi$ is removed from the output term, etc.
When the entire list gets replaced, the handler receives the old list
\coqinline{l} and the new list \coqinline{l'}: the output term will see
\coqinline{(length l')} $\Pi$s removed, and for each element of \coqinline{l}, a
corresponding $\Pi$ will be added.  We then have two functions:

\noindent
\begin{minted}{haskell}
δListFoldLeft, δListFoldRight ::
  ΔListFold τ δτ δ ->
  [τ] -> ΔList τ δτ -> Maybe (δ -> δ)
\end{minted}

\noindent which take such a description, an original list, and a list diff, and
compute the resulting diff transformer for the output type, as long as the diff
makes sense for the list.  Receiving the original list lets us check the sanity
of the list diff at each step (e.g.\ a diff should not ask to drop the head from
an empty list), and lets us pass the element being modified or removed to the
handlers who need the information.  With all this machinery ready, we can derive
\coqinline{δEliminator\-Type} as:

\noindent
\begin{minted}{haskell}
δEliminatorType ... =
    δquantifyVars  ips              δips
<$> CopyPi         δmotiveType      Same
<$> δquantifyCases cs               δcs
 $  δquantifyVars  iis              δiis
 $  CopyPi         δdiscrimineeType Same
 $  CopyApp        δindices         Same
  where
    δindices = δapplyVars iis δiis Same
\end{minted}

\noindent which matches closely the original definition of
\coqinline{Eliminator\-Type}.  We use the same mechanism to derive a
diff-propagating version of the function that computes the type of an inductive
type family (based on the diff of its parameters and indices lists), and a
diff-propagating version of the function that computes the type of a constructor
(based on the diff of its parameters and indices lists).  At a high-level, we
have the following property:

\[
  {
    \inferrule*
    [width=5em]
    {
      {
        {\MathMkElimType
          {\Inductive
            {n}
            {\overrightarrow{p}}
            {\overrightarrow{i}}
            {u}
            {\overrightarrow{c}}
          }
        }
        \op{\ =\ }
        \out{e}
      }
      \\
      {\MathPatches
        {\Inductive
          {n}
          {\overrightarrow{p}}
          {\overrightarrow{i}}
          {u}
          {\overrightarrow{c}}
        }
        {\delta_i}
        {\Inductive
          {n'}
          {\overrightarrow{p'}}
          {\overrightarrow{i'}}
          {u'}
          {\overrightarrow{c'}}
        }
      }
      \\
      {
        {\MathMkΔElimType
          {\Inductive
            {n}
            {\overrightarrow{p}}
            {\overrightarrow{i}}
            {u}
            {\overrightarrow{c}}
          }
          {\delta_i}
        }
        \op{\ =\ }
        \out{\delta_e}
      }
      \\
      {
        {\MathMkElimType
          {\Inductive
            {n'}
            {\overrightarrow{p'}}
            {\overrightarrow{i'}}
            {u'}
            {\overrightarrow{c'}}
          }
        }
        \op{\ =\ }
        \out{e'}
      }
    }
    {
      \MathPatches{e}{\delta_e}{e'}
    }
  }
\]

\noindent i.e., \coqinline{δEliminator\-Type} produces the correct diff to
transport the old eliminator type to the new eliminator type.  In fact, we even
have the following corollary:

\[
  {
    \inferrule*
    [width=5em]
    {
      {{\MathMkElimType{i}} \op{\ =\ } \out{e}}
      \\
      {
        {\MathMkΔElimType
          {i}
          {\ModifyInductive
            {\MathReplace{n}}
            {\MathReplace{\overrightarrow{p}}}
            {\MathReplace{\overrightarrow{i}}}
            {\MathReplace{u}}
            {\MathReplace{\overrightarrow{c}}}
          }
        }
        \op{\ =\ }
        \out{\delta_e}
      }
      \\
      {
        {\MathMkElimType
          {\Inductive
            {n}
            {\overrightarrow{p}}
            {\overrightarrow{i}}
            {u}
            {\overrightarrow{c}}
          }
        }
        \op{\ =\ }
        \out{e'}
      }
    }
    {\MathPatches{e}{\delta_e}{e'}}
  }
\]

\noindent making \coqinline{eliminatorType} redundant as it coincides with a
modification where each field of the inductive data type definition is replaced
with a new value.
