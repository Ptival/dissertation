\subsection{Global design of \PeaCoq{}}

We will discuss the global architecture of \PeaCoq{} that allows us to build all
the features we mentioned.  Each feature will then be described individually.

\PeaCoq{} is designed following a client-server architecture.  The server
handles the \Coq{} process, and exposes end-points to communicate with it via
HTTP.\@ The client is a web application, containing the front-end view that users
interact with, and facilities for driving the interactions with the server.  Let
us describe each side further.

\tikzset{>=latex}

\begin{figure}[!htp]

  \centering
  \begin{tikzpicture}[node distance=1.5cm]

  % \tikzstyle{state} = [draw, very thick, fill=white, rectangle, minimum height=3em, minimum width=7em, node distance=8em, font={\sffamily\bfseries}]
  % \tikzstyle{stateEdgePortion} = [black,thick];
  % \tikzstyle{stateEdge} = [stateEdgePortion,->];
  % \tikzstyle{edgeLabel} = [pos=0.5, text centered, font={\sffamily\small}];

  \tikzstyle{outer} = [rectangle,draw = black,thick,inner sep = 10pt];
  \tikzstyle{inner} = [rectangle,rounded corners,draw = black,thick];
  \tikzstyle{arrow} = [thick];

  \node [inner]                         (snap)       {Snap Server};
  \node [inner,below=of snap]           (sapi)       {SerTop (SerAPI)};
  \node [below=of sapi]                 (coq-label)  {Coq};
  \node [inner, below=0cm of coq-label] (plugin)     {PeaCoq plug-in};

  \node [inner,fit={(coq-label) (plugin)}] (coq) {};

  \node [above=0.5cm of snap] (backend-label) {Back-end};
  \node [outer,fit={(backend-label) (snap) (sapi) (coq)}] (backend) {};

  \draw[<->,ultra thick] (snap.south) -- (sapi.north);
  \draw[<->,ultra thick] (sapi.south) -- (coq.north);

  \node [inner,right=5.5cm of snap] (driver)     {Driver};
  \node [inner,below=of driver]     (editor)     {Editor};
  \node [inner,left=of editor]      (prooftree)  {Proof-tree};
  \node [inner,right=of editor]     (automation) {Automation};

  \draw[<->,ultra thick] (driver.south west) -- (prooftree.north);
  \draw[<->,ultra thick] (driver.south) -- (editor.north);
  \draw[<->,ultra thick] (driver.south east) -- (automation.north);

  \node [above=0.5cm of driver] (frontend-label) {Front-end};
  \node [outer,fit={(frontend-label) (driver) (prooftree) (editor) (automation)}] (frontend) {};

  \draw[<->,ultra thick] (snap.east) -- (driver.west);

\end{tikzpicture}

  \caption{\PeaCoq{}'s architecture}

  \label{peacoq-architecture}

\end{figure}

\subsubsection*{Back-end / Server}

In order to let front-ends interact with it, the \Coq{} process exposes an API
that serializes metadata and accepts certain commands.  Unfortunately, prior to
\PeaCoq{}'s development, very few tools had used this API, and as a result, it
is lacking both in polish and in features.

The protocol it uses is based upon XML syntax, and exposes commands to
manipulate an abstract document (a collection of \Coq{} sentences) and command
and observe its execution.

While developing \PeaCoq{}, we interacted with Emilio Jesús Gallego Arias, who
was developing a layer over this XML protocol, named \SerAPI{} (for
serialization API).  This layer offers a less rough API based on s-expressions,
abstracting over some minute details of \Coq{}'s implementation.

However, at the time \PeaCoq{} was built, \Coq{} was not exposing enough of its
internals for our needs.  In order to remedy this, we also built a \Coq{}
plug-in to expose this data.  Plug-ins circumvent the IDE API by directly being
compiled and loaded alongside the \Coq{} code: thus, they have access to all the
public interfaces of \Coq{}'s internal modules.  Once our plug-in is loaded, it
registers as a \Vernacular{} command that can be invoked either by users, or
through the IDE API.\@

In order to communicate with front-ends, the back-end is driven by a HTTP
server, based on \Haskell{}'s \Snap{} framework.  The details of its
implementation are fairly mundane: it simply listens to requests from the
front-end, passes them down to \Coq{} through \SerAPI{}, and forwards the
responses back without much processing.  However, this architecture provided
several advantages:

\begin{itemize}

  \item Since the back-end communicates with the front-end over HTTP, the two
need not be on the same physical machine.  For instance, we have had the
back-end running on a powerful Internet-facing machine, and connected it from a
less powerful phone, on public transit.  It worked flawlessly, even in the
presence of computation-intensive loads like our automation, because the
computation was happening server-side.  Meanwhile, the lightweight client was
only processing display and communication with the server.

  \item The back-end was built in such a way that it could accept multiple
connections at once.  Each connection would spawn its own, separate instance of
\Coq{}.  This means that multiple clients could independently connect and work
on the same server.  We used this in our study at the University of Washington,
where one server was shared among all students, who did not need to install
\Coq{} on their machine.

\end{itemize}

\subsubsection*{Front-end / Client}

The front-end of \PeaCoq{} is a simple web application.  It is currently written
in \TypeScript{}, a statically-typed dialect of \JavaScript{}.  We call driver
the part of the code responsible for communicating with the back-end.  We use a
reactive programming library called \RxJS{}, which provides a stream abstraction
for sequences of values.  Messages from the back-end are published as streams,
and each of the components involved in the front-end (namely, the editor, the
proof-tree, and the automation layer) can subscribe to those messages they need
to observe.  Conversely, those components publish streams of commands they'd
like the back-end to process, which the driver subscribes to, and orchestrates
the dispatch of.

This orchestration can require some finesse.  The API provided by \Coq{} is
fairly stateless, that is, \emph{not} in the sense that there are no states, but
rather, in the sense that there is no notion of a current, mutable state.
Instead, each state is given a unique identifier, and subsequent commands may
explicitly state over which previous state they wish to be applied.  This
abstraction holds well at the document level, but unfortunately, some commands
change global, mutable flags, in ways that can be observed.

In other to prevent issues with such observable, mutable state, some sequences
of commands must be processed atomically, that is, no other action may be
interleaved with the sequence.  In order to achieve this, the layers emit their
commands not as a simple sequence of commands, but as a sequence of sequences of
commands, to be dispatched atomically.

This raises another concern: some atomic sequences of commands have data
dependencies.  In particular, later commands often need to know the
\emph{output} of previous commands.  A pervasive example of such a pattern is
found in our automation layer, where commands must be silently tried in the
background, their output must be gathered, and then their effect must be
cancelled.  In order to cancel an action, we must tell \Coq{} the identifier of
the state(s) to be cancelled, but that identifier is only know asynchronously,
in a response from the action that created said state.

Once again, this problem is easily solved by issuing atomic sequences to the
driver not as an array of sequences, but as a stream of commands.  When the
driver chooses to emit a given atomic sequence, it subscribes to it.  It will
subsequently, and asynchronously, receive one or many commands, until the stream
indicates it has completed.  Elements from this stream can be asynchronously
built from other streams, and so we can build those dependent atomic sequences
as:

\begin{minted}{javascript}
const commandToCancel = new Add('Command To Cancel')

// By convention, we put a $ sign behind stream variables.
const answer$ =
  added$.filter(sameTagAs(commandToCancel))
  .takeUntil(completed$.filter(sameTagAs(commandToCancel)))

const output$ = answer$.map(...) // do what you need

const cancelCommand$ =
  answer$.map(a => new Cancel([a.stateId]))

const commands$$ =
  Rx.Observable.concat([
    Rx.Observable.of(commandToCancel),
    cancelCommand$
  ])
\end{minted}

Understanding this code is not necessary, but here is a high-level explanation
for the interested reader:

\begin{itemize}

  \item We create (but, do \emph{not} issue yet!) the command whose output we
want to observe.  When a command is created, it acquires a unique tag, that is
sent to \Coq{}, and appears in responses.  This helps us filter those answers
that correspond to this query.

  \item We preemptively subscribe to the stream of answers, looking for ones
with the tag of our command.  In case there is no answer, we also cut this
subscription short when the stream of completed answers emits.  Every command
will produce such an item, so we are guaranteed to terminate.

  \item We can listen to this answer, and compute our output however we see
fit.

  \item We can also listen to this answer in order to emit a cancel command for
it.

  \item The final atomic sequence of commands we send to the driver is the
concatenation of two observables: the command, and its corresponding cancel
command.

\end{itemize}

This is the most important abstraction on the front-end side.  Most of the code
is event-driven by subscribing to those streams and producing streams of
requests.
