\section{Evaluation}

We built a tool, called \PeaCoq{}, to try and evaluate the usefulness of those
techniques in a beginner setting.  We conducted two studies: a longitudinal
study in the classroom during one quarter, and a short A/B study with beginners.

\subsection{Longitudinal study}

The first study was conducted at the University of Washington, with the help of
instructor Zachary Tatlock, during the Winter quarter of 2015.

The study was approved by the institutional review board of the University of
California, San Diego (project~\#141713), and the institutional review board of
the University of Washington (project~\#48738).

Every student in the class was given the option to use \PeaCoq{} instead of
other IDEs for working on their homework.  At any moment, they could opt in and
out of using \PeaCoq{} with no overhead.  This study helped us iron out details
on the automation and the display, based on students' feedback.

% TODO: find PeaCoq notes and write it up
