\section{Evaluation}

We built a tool, called \PeaCoq{}, to try and evaluate the usefulness of those
techniques in a beginner setting.  We conducted two studies: a longitudinal
study in the classroom during one quarter, and a short A-B study with beginners.

\subsection{Longitudinal study}~\label{peacoq-longitudinal-study}

The first study was conducted at the University of Washington, with the help of
instructor Zachary Tatlock, during the Winter quarter of 2015.

The study was approved by the institutional review board of the University of
California, San Diego (project~\#141713), and the institutional review board of
the University of Washington (project~\#48738).

Every student in the class was given the option to use \PeaCoq{} instead of
other IDEs for working on their homework.  At any moment, they could opt in and
out of using \PeaCoq{} with no overhead.  This study helped us iron out details
on the automation and the display, based on students' feedback.

% TODO: find UW PeaCoq notes and write it up

\subsection{A-B study}~\label{peacoq-a-b-study}

Participants in the A-B study were volunteers who received no financial
compensation, but were offered free slices of pizza, a highly sought treat in
academic settings.

\paragraph{Study setup}

The study was done in two instances, totaling 20 participants.  For each
instance of the study, 10 participants were chosen randomly but based on their
availability, then participants were randomly paired in 5 groups.  Both groups
were informed that they would be testing a novel programming environment.

The 5 groups in the \define{control group} were provided with an instance of
\PeaCoq{} designed to imitate the usual IDEs for \Coq{}: all the special
features of \PeaCoq{} were disabled, except for its syntax highlighting, and
keyboard shortcuts.

The 5 groups in the \define{study group} were provided with an instance of
\PeaCoq{} with the proof tree view always enabled, visual diffs overlaid on top
of nodes in the proof tree view, and automation running in the background to
populate the proof tree view with candidate tactics, all enabled.

Neither group was informed about the fact that this was an A-B study, or about
whether they were testing the real prototype or the control version.  Both
instances were structured identically: the first hour was a general presentation
of the tool they would use, and of basics of the \Coq{} proof assistant.  Then,
each pair of participant was tasked with solving 16 problems of increasing
difficulty, testing their understanding of what they have seen so far, as well
as their ability to learn about new proof solving tactics and use them
effectively.

After the study was over, and before they left, the participants were handed an
anonymous questionnaire, asking them about some qualitative feedback and
information about their education level.

\paragraph{Study material}

During the first part of the study, participants were introduced, by the study
coordinator, to the following concepts:

\begin{itemize}

\item an \emph{inductive datatype definition} \coqinline{day}, with constructors
\coqinline{monday}, \coqinline{tuesday}, etc.,

\item a \emph{function definition} \coqinline{tomorrow}, defined by
pattern-matching,

\item a theorem asserting that \coqinline{tomorrow saturday = sunday}, proven
introducing the tactics \coqinline{simpl} and \coqinline{reflexivity},

\item a recursive data type, \coqinline{natlist}, representing a monomorphic
list of natural numbers,

\item a \emph{recursive} function, \coqinline{concat}, performing list
concatenation,

\item another theorem about the concatenation of two concrete lists, displaying
an instance of associativity, and proven using the same tactics seen so far,

\item a theorem asserting that \coqinline{nil} is a left-unit for
\coqinline{concat}, proven with the previous tactics, with the addition of the
\coqinline{intros} tactic to introduce the universally-quantified list
\coqinline{l},

\item a theorem asserting that \coqinline{nil} is a right-unit for
\coqinline{concat}, proven by introducing the \coqinline{induction} tactic, and
also introducing the first \coqinline{rewrite},

\item a theorem asserting the associativity of the \coqinline{concat} operation,
proven by using \coqinline{induction}, and the other tactics mentioned so far.

\end{itemize}

During the second part of the study, they had to try and solve all 16 exercises,
as listed exhaustively in Appendix~\ref{appendix-peacoq-material}.  The
exercises can be roughly described as follows:

\begin{enumerate}

  \item \coqinline{rev_snoc}: After introducing a recursive function
\coqinline{snoc} to append one element to the end of a list, and a recursive
function \coqinline{rev} to reverse a list, participants were asked to
demonstrate that a sequence of \coqinline{rev} after \coqinline{snoc} is
equivalent to a sequence of \coqinline{cons} after \coqinline{rev}.

  \item \coqinline{rev_involutive}: Participants were asked to demonstrate that
\coqinline{rev} is involutive (i.e.\ applying it twice consecutively yields the
original input).

  \item \coqinline{concat_cons_snoc}: Participants were asked to prove an
equality about the interplay of \coqinline{concat} and \coqinline{snoc}.

  \item \coqinline{go_somewhere}: A new concept was introduced: disjunction.
Two tactics to manipulate this concept were introduced: \coqinline{left} and
\coqinline{right}.  An example was given of proving the disjunction of a
falsehood on the left, and a tautology on the right.  Participants were then
asked to find the only disjunct that was a tautology within a nested disjunction
of falsehoods.  Finding the only tautology required uses of both
\coqinline{left} and \coqinline{right}.

  \item \coqinline{B_is_enough}: Participants were introduced to the tactic
\coqinline{apply}, and asked to prove a disjunction where one disjunct was given
in the premisses, and one was not.

  \item \coqinline{more_facts}: Participants were introduced to the concept of
conjunction.  A new tactic, \coqinline{split}, was then introduced to prove
conjunctions.  They were then asked to prove the conjunction of two tautologies.

  \item \coqinline{A_and_B}: To re-assert the importance of \coqinline{apply},
participants were asked to prove a conjunction where each conjunct was found in
the premisses.

  \item \coqinline{snoc_concat_end}: Two harder exercises about list were then
presented.  The first one, asking about a more complex interplay between
\coqinline{concat} and \coqinline{snoc},

  \item \coqinline{rev_distributes_over_concat}: the second one, asking to prove
that \coqinline{rev} distributes over \coqinline{concat}.

  \item \coqinline{map_commutes}: Participants were introduced to the concept of
the function \coqinline{map} over lists.  They then showed that, if two
functions commute, then mapping these two functions also commutes.

  \item \coqinline{map_fusion}: Participants were asked to prove the map fusion
property.

  \item \coqinline{fold_snoc}: Participants were introduced to the concept of a
\coqinline{fold} over a list.  They then demonstrated an interplay between
\coqinline{fold} and \coqinline{snoc}.

  \item \coqinline{map'_unroll}: Participants were asked to demonstrate that
performing the operation \coqinline{map} over a list obtained via
\coqinline{cons} can be unrolled one step.

  \item \coqinline{map_map'}: Participants were then shown how \coqinline{map}
can be implemented as a \coqinline{fold}.  The resulting implementation, named
\coqinline{map'}, was then demonstrated to be functionally equivalent to
\coqinline{map}.  To help them, we axiomatized a small theorem that they could
use without needing to prove.

  \item \coqinline{In_cons}: We introduced a recursive predicate,
\coqinline{In}, asserting the presence of an element in a list.  They were first
asked to prove that if an element is in a list, it is still present in a list
with an additional element.

  \item \coqinline{In_concat_left}: Two final concepts were introduced: the
\coqinline{cases} tactic is a custom tactic that let participants break a
conjunction in their context into its components, and the
\coqinline{contradiction} tactic allowing them to point to the presence of
falsehoods in the context.  Participants were finally asked to prove that if an
element belongs in a list, it also belongs in the result of concatenating said
list with an arbitrary other list.

\end{enumerate}


\begin{table}[!htbp]
  \centering
  \caption{\PeaCoq{} A-B study exercises design}
  \begin{tabular}{l *{11}{c}}
    \toprule
    Exercise & \multicolumn{11}{c}{What tactics were expected?} \\
    & \rotatebox{90}{\safecoqinline{simpl}}
    & \rotatebox{90}{\safecoqinline{reflexivity}}
    & \rotatebox{90}{\safecoqinline{intros}}
    & \rotatebox{90}{\safecoqinline{induction}}
    & \rotatebox{90}{\safecoqinline{rewrite}}
    & \rotatebox{90}{\safecoqinline{left}}
    & \rotatebox{90}{\safecoqinline{right}}
    & \rotatebox{90}{\safecoqinline{apply}}
    & \rotatebox{90}{\safecoqinline{split}}
    & \rotatebox{90}{\safecoqinline{cases}}
    & \rotatebox{90}{\safecoqinline{contradiction}}
    \\
%                                           Sim Rfx Int Ind Rwt Lft Rgt App Spt Cas Con
    \midrule
    01. \safecoqinline{rev_snoc          } &\OK&\OK&\OK&\OK&\OK&   &   &   &   &   & \\
    02. \safecoqinline{rev_involutive    } &\OK&\OK&\OK&\OK&\OK&   &   &   &   &   & \\
    03. \safecoqinline{concat_cons_snoc  } &\OK&\OK&\OK&\OK&\OK&   &   &   &   &   & \\
    04. \safecoqinline{go_somewhere      } &   &\OK&   &   &   &   &\OK&   &   &   & \\
    05. \safecoqinline{B_is_enough       } &   &\OK&   &   &   &\OK&\OK&   &   &   & \\
    06. \safecoqinline{more_facts        } &   &\OK&   &   &   &   &\OK&   &\OK&   & \\
    07. \safecoqinline{A_and_B           } &   &   &\OK&   &   &   &   &\OK&\OK&   & \\
    08. \safecoqinline{snoc_concat_end   } &   &\OK&\OK&   &\OK&   &   &   &   &   & \\
    09. \safecoqinline{rev_distributes...} &\OK&\OK&\OK&\OK&\OK&   &   &   &   &   & \\
    10. \safecoqinline{map_commutes      } &\OK&\OK&\OK&\OK&\OK&   &   &   &   &   & \\
    11. \safecoqinline{map_fusion        } &\OK&\OK&\OK&\OK&\OK&   &   &   &   &   & \\
    12. \safecoqinline{fold_snoc         } &\OK&\OK&\OK&\OK&\OK&   &   &   &   &   & \\
    13. \safecoqinline{map'_unroll       } &   &\OK&\OK&   &   &   &   &   &   &   & \\
    14. \safecoqinline{map_map'          } &\OK&\OK&\OK&\OK&\OK&   &   &   &   &   & \\
    15. \safecoqinline{In_cons           } &\OK&   &\OK&   &   &   &\OK&\OK&   &   & \\
    16. \safecoqinline{In_concat_left    } &\OK&   &\OK&\OK&   &\OK&\OK&\OK&   &\OK&\OK\\
    \bottomrule
  \end{tabular}{\parfillskip=0pt\par}
\end{table}


% TODO: show what the actual exercise was

TODO TODO TODO

\paragraph{Study results}

TODO TODO TODO

All the answers provided by participants are anonymously catalogued
in Appendix~\ref{appendix-peacoq-a-b-study}.

\begin{table}[!htbp]
  \centering
  \caption{\PeaCoq{} A-B study exercises timings}
  \begin{tabular}{l r r}
    \toprule
    Exercise & A average & B average \\
    \midrule
    01. \safecoqinline{rev_snoc                   } &  213s & 609s \\
    02. \safecoqinline{rev_involutive             } &  571s & 180s \\
    03. \safecoqinline{concat_cons_snoc           } &  158s & 292s \\
    04. \safecoqinline{go_somewhere               } &   24s &  14s \\
    05. \safecoqinline{B_is_enough                } &  134s &  66s \\
    06. \safecoqinline{more_facts                 } &   34s &  27s \\
    07. \safecoqinline{A_and_B                    } &   49s &  21s \\
    08. \safecoqinline{snoc_concat_end            } &  104s & 156s \\
    09. \safecoqinline{rev_distributes_over_concat} & 1446s & 748s \\
    10. \safecoqinline{map_commutes               } &  348s & 189s \\
    11. \safecoqinline{map_fusion                 } &  109s &  76s \\
    12. \safecoqinline{fold_snoc                  } &   86s & 207s \\
    13. \safecoqinline{map'_unroll                } &  471s &  13s \\
    14. \safecoqinline{map_map'                   } &  317s &  80s \\
    15. \safecoqinline{In_cons                    } &   52s & 160s \\
    16. \safecoqinline{In_concat_left             } &  492s & 791s \\
    \bottomrule
  \end{tabular}{\parfillskip=0pt\par}
\end{table}

\paragraph{Threats to validity}

We provide a list of the many threats to validity of our results, in no
particular order:

\begin{itemize}

  \item The sample size was very small (20 participants total, 10 per instance).

  \item The sample was biased towards graduate students, with only 3
undergraduate students (1 in the A group, 2 in the B group).

  \item The study was fast-paced, with only one hour to learn the rudiments of
theorem proving, and one hour to start solving exercises.  A typical
\emph{graduate} programming language course would most likely cover the topics
we saw in three to five lectures.

  \item A bug in \PeaCoq{} slowed down some pairs in group B: they had to wait
for us to come fix it, and reload the page to go back to where they initially
were.

  \item Participant $B4_{2}$ vocally explained their dislike for mathematical
logic.  This can be seen reflected in Appendix~\ref{appendix-peacoq}, where
their ratings were significantly lower than everyone else's.  We might expect better ratings from users who are willingly trying to learn, but we should also expect similar ratings from users who are forced to learn, for instance in a course setting.

\end{itemize}
