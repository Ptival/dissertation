\section{Evaluation}

We built a tool, called \PeaCoq{}, to try and evaluate the usefulness of those
techniques in a beginner setting.  We conducted two studies: a longitudinal
study in the classroom during one quarter, and a short A-B study with beginners.

\subsection{Longitudinal study}~\label{peacoq-longitudinal-study}

The first study was conducted at the University of Washington, with the help of
instructor Zachary Tatlock, during the Winter quarter of 2015.

The study was approved by the institutional review board of the University of
California, San Diego (project~\#141713), and the institutional review board of
the University of Washington (project~\#48738).

Every student in the class was given the option to use \PeaCoq{} instead of
other IDEs for working on their homework.  At any moment, they could opt in and
out of using \PeaCoq{} with no overhead.  This study helped us iron out details
on the automation and the display, based on students' feedback.

% TODO: find PeaCoq notes and write it up

\subsection{A-B study}~\label{peacoq-a-b-study}

Participants in the A-B study were volunteers who received no financial
compensation, but were offered free slices of pizza, a highly sought treat in
academic settings.

\paragraph{Study setup}

The study was done in two instances, totaling 20 participants.  For each
instance of the study, 10 participants were chosen randomly but based on their
availability, then participants were randomly paired in 5 groups.

The 5 groups in the \define{control group} were provided with an instance of
\PeaCoq{} designed to imitate the usual IDEs for \Coq{}: all the special
features of \PeaCoq{} were disabled, except for its syntax highlighting, and
keyboard shortcuts.

The 5 groups in the \define{study group} were provided with an instance of
\PeaCoq{} with the proof tree view, the visual diffs, and automation in the
background, all enabled.

Both instances were structured identically: the first hour was a general
presentation of the tool they would use, and of basics of the \Coq{} proof
assistant.  To be precise, the following was demonstrated:

\begin{itemize}

\item an \emph{inductive datatype definition} \coqinline{day}, with constructors
\coqinline{monday}, \coqinline{tuesday}, etc.,

\item a \emph{function definition} \coqinline{tomorrow}, defined by
pattern-matching,

\item a theorem asserting that \coqinline{tomorrow saturday = sunday}, proven
introducing the tactics \coqinline{simpl} and \coqinline{reflexivity},

\item a recursive data type, \coqinline{natlist}, representing a monomorphic
list of natural numbers,

\item a \emph{recursive} function, \coqinline{concat}, performing list
concatenation,

\item another theorem about the concatenation of two concrete lists, displaying
an instance of associativity, and proven using the same tactics seen so far,

\item a theorem asserting that \coqinline{nil} is a left-unit for
\coqinline{concat}, proven with the previous tactics, with the addition of the
\coqinline{intros} tactic to introduce the universally-quantified list
\coqinline{l},

\item a theorem asserting that \coqinline{nil} is a right-unit for
\coqinline{concat}, proven by introducing the \coqinline{induction} tactic, and
also introducing the first \coqinline{rewrite},

\item a theorem asserting the associativity of the \coqinline{concat} operation,
proven by using \coqinline{induction}, and the other tactics mentioned so far.

\end{itemize}

Subsequently, the participants were introduced to the principle of induction
over a data structure, and the concept of proving a property inductively by
solving the base case, and then solving the inductive case with the presence of
an induction hypothesis.

All the answers provided by participants are anonymously catalogued
in Appendix~\ref{appendix-peacoq-a-b-study}.

\begin{table}[!htbp]
  \centering
  \caption{\PeaCoq{} A-B study exercise timings}
  \begin{tabular}{l||r|r}
    \toprule
    Exercise & A average & B average \\
    \midrule
    01. \safecoqinline{rev_snoc                   } &  213s & 609s \\
    02. \safecoqinline{rev_involutive             } &  571s & 180s \\
    03. \safecoqinline{concat_cons_snoc           } &  158s & 292s \\
    04. \safecoqinline{go_somewhere               } &   24s &  14s \\
    05. \safecoqinline{B_is_enough                } &  134s &  66s \\
    06. \safecoqinline{more_facts                 } &   34s &  27s \\
    07. \safecoqinline{A_and_B                    } &   49s &  21s \\
    08. \safecoqinline{snoc_concat_end            } &  104s & 156s \\
    09. \safecoqinline{rev_distributes_over_concat} & 1446s & 748s \\
    11. \safecoqinline{map_commutes               } &  348s & 189s \\
    12. \safecoqinline{map_fusion                 } &  109s &  76s \\
    13. \safecoqinline{fold_snoc                  } &   86s & 207s \\
    14. \safecoqinline{map'_unroll                } &  471s &  13s \\
    15. \safecoqinline{map_map'                   } &  317s &  80s \\
    16. \safecoqinline{In_cons                    } &   52s & 160s \\
    17. \safecoqinline{In_concat_left             } &  492s & 791s \\
    \bottomrule
  \end{tabular}{\parfillskip=0pt\par}
\end{table}
