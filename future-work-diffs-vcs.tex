\section{Integrating code diffs in version control systems}

An equally ambitious goal could be to treat our diffs, that is, first-class
values that represent the intent of changes made to a program, the same way we
treat the code of the program.  For instance, as a program evolves, we keep a
published index on the changes made to it using a version control system.  We
can envision a diff-aware version system wherein diffs are also published.

Consider a use case where Ada is publishing a software library that Bertrand
depends upon for his project.  When Ada decides to make a breaking change to her
library, she could perform the change by using a diff-based repair algorithm
like the one we present.  When the repair is finished, she has a complete,
structural diff, that captures exactly all of the changes that have happened to
the library.  Ada can publish this diff, alongside the new code, to her public
version control system.  Bertrand can update his dependency to Ada's new
version, but he needs to update his client code to adapt to the changes made by
Ada.  Thanks to the published diffs, Bertrand could also run a repair algorithm
over his code, automating much of the necessary code changes needed to stay
compatible with Ada's library.

Of course, this workflow would need to be extremely polished to convince both
users to opt in.  In particular, we would definitely want a strong diff guessing
algorithm, like the one presented in Section~\ref{chick-guess}, and would need a
nice way of presenting different guesses to the programmer so that they can
safely pick the choice that best matches their intent.
