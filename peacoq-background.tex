\section{Background}

Proof assistants have a steep learning curve.  Not only do they require an
understanding of high-level mathematical concepts, including but extending
beyond the ones presented in Chapter~\ref{background}, but they also require the
user to learn the idiosyncrasies of their proof assistant of choice.  Even
experts of a given proof assistant would require a decent amount of time to
become proficient in a different proof assistant.

Let's focus on the \Coq{} proof assistant.  Learning to use it requires getting
familiar with the Calculus of (Co-)Inductive Constructions, a dependently-typed
lambda calculus.  Proficiency in this language requires becoming familiar with
dependent pattern matching, universe hierarchies, induction and co-induction,
well-founded relations, etc.  This covers the knowledge required to write code
and specifications in this proof assistant.

In order to write proofs, the user must learn an additional, untyped language of
tactics called \Ltac{}.  This language contains more than a hundred tactic
formers, with several variations for each tactic.  While many proofs can be
carried out with just a small subset of those, complicated proofs often require
a large arsenal of tactics and a thorough understanding of the tactic language.

Additionally, writing proofs over standard data types requires being able to
locate relevant theorems within the standard library.  This standard library,
by default, includes more than a thousand lemmas.  While \Coq{} offers facilities
to search a relevant lemma by giving the shape of its expected type, this is still
a somewhat tedious endeavor.

We have identified three challenges in the learning process for novice users
that we would like to tackle with \PeaCoq{}:

\begin{enumerate}

  \item conceptualizing and keeping track of the \emph{proof tree structure}
while building proofs,

  \item identifying the effects of a tactic on the proof context,

  \item identifying relevant tactics that can be applied in a given proof
obligation.

\end{enumerate}
