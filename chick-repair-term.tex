\subsection{Repairing terms}\label{chick-design-repair-term}

The repair algorithm for terms is described formally in
Figures~\ref{fig:repair-term-1}, \ref{fig:repair-term-2},
\ref{fig:repair-term-two-1}, and~\ref{fig:repair-term-two-2} using the following
judgments:

{
  \[
    {\turnstile
      { \dcontext{E}{\delta_E}{\Gamma}{\delta_\Gamma} }
      {\repair{t}{\delta_t}{\tau}{\delta_\tau}}
    }
  \]
  % \[
  %   {\turnstile
  %     { \dcontext{E}{\delta_E}{\Gamma}{\delta_\Gamma} }
  %     {\genericrepair{t}{\delta_t}{\tau}}
  %   }
  % \]
  \[
    {\turnstile
      { \dcontext{E}{\delta_E}{\Gamma}{\delta_\Gamma} }
      {\repairTermWithoutType{t}{\delta_t}}
    }
  \]
}%
%
The first judgment (described in Figures~\ref{fig:repair-term-1}
and~\ref{fig:repair-term-2}) describes how a term is repaired when we know both
its old type $\tau$, and how that type got modified $\delta_\tau$.  It is
syntax-directed by the structure of $\delta_\tau$.  The second judgment
(described in Figures~\ref{fig:repair-term-two-1}
and~\ref{fig:repair-term-two-2}) describes how a term is repaired when we no
longer make decisions based on its old type and type diff: this happens for
constructs whose type does not inform us about the type of its constituents (for
instance, function application), and for simple terms (holes, universes,
variables).

\paragraph{Notation} The first judgment can be read as follows:

\begin{itemize}

  \item \textbf{Top}: Given that a term $t$ had type $\tau$ in local context
$\Gamma$ and global environment $E$,

  \item \textbf{Bottom left}: and the global environment underwent modification
$\delta_{E}$,

  \item \textbf{Bottom left}: and the local context underwent modification
$\delta_{\Gamma}$,

  \item \textbf{Bottom right}: and the type underwent modification
$\delta_{\tau}$,

  \item \textbf{Right}: then, we suggest to repair the term $t$ with diff
$\out{\delta_{t}}$.

\end{itemize}

\noindent The second judgment can be read analogously, but does not have access
to typing information for the term being repaired.  Importantly, the first
judgment assumes the diff $\delta_\tau$ is repaired already: it is the
responsibility of the caller to ensure so.

\subsubsection*{Diff-directed term repair algorithm}

\newcommand{\RepairTermOneSamePi}{
  \inferrule*
  [lab=\RSamePi]
  {
    {\turnstile
      { \dcontext{E}{\delta_E}{\Gamma}{\delta_\Gamma} }
      { \repair
        {t}
        {\delta}
        {\MathPi{\tau_1}{\chi}{\tau_2}}
        {\MathModPi{\MathSameTerm{}}{\MathSameBinder{}}{\MathSameTerm{}}}
      }
    }
  }
  {\turnstile
    { \dcontext{E}{\delta_E}{\Gamma}{\delta_\Gamma} }
    { \repair
      {t}
      {\delta}
      {\MathPi{\tau_1}{\chi}{\tau_2}}
      {\MathSameTerm{}}
    }
  }
}

\newcommand{\RepairTermOneSameOther}{
  \inferrule*
  [lab=\RSame]
  {
    \text{\rulename{\RSamePi} does not apply}
    \and
    {\turnstile
      { \dcontext{E}{\delta_E}{\Gamma}{\delta_\Gamma} }
      { \repairTermWithoutType{t}{\delta_{t}}}
    }
  }
  {\turnstile
    { \dcontext{E}{\delta_E}{\Gamma}{\delta_\Gamma} }
    { \repair{t}{\delta_{t}}{\tau}{\MathSame} }
  }
}

\newcommand{\RepairTermOneModPi}{
  \inferrule*
  [lab=\RModPi]
  {
    {\turnstile%
      {\dcontext{E}{\delta_E}{\Gamma}{\delta_\Gamma}}
      {\freshOne{x}{\MathSame}{\delta_x}}
    }
    \and
    {\MathPatches{x}{\delta_x}{x'}}
    % \and
    % {\diffExpand{\tau_{2}}{\delta_{\tau_{2}}}{\delta_{\tau_{2}}'}
    % }
    \\\\\\
    {\turnstile%
      {\dcontext{E}{\delta_E}{\Gamma}{\delta_\Gamma}}
      {\freshTwo{\blackdiff{x}{\delta_x}}{\blackdiff{\chi}{\delta_\chi}}{z}}
    }
    \and
    {\MathPatches{\chi}{\delta_\chi}{\chi'}}
    \\\\\\
    {\turnstile%
      { \dcontext{E}{\delta_E}
        { \MathCons{\MathLocalAssum{z}{\tau_1}}{\Gamma} }
        { \MathMod%
          {\MathLocalAssum{\MathReplace{z}}{\delta_{\tau_1}}}
          {\delta_\Gamma}
        }
      }
      { \repair%
        {t[\subst{x}{z}]}
        {\delta_{t}}
        {\tau_2[\subst{\chi}{z}]}
        {\delta_{\tau_2}[\subst{\MathReplace{\chi'}}{\MathReplace{z}}]}
      }
    }
    % \\\\\\
    % {\diffExpand{t}{\delta_{t}}{\delta_{t}'}
    % }
  }
  {\turnstile%
    { \dcontext{E}{\delta_E}{\Gamma}{\delta_\Gamma} }
    {\repair%
      { \MathLam{x}{t} }
      { \MathModLam{\delta_x}{\delta_{t}[\subst{\MathReplace{z}}{\MathReplace{x'}}]} }
      { \MathPi{\tau_1}{\chi}{\tau_2} }
      { \MathModPi{\delta_{\tau_1}}{\delta_\chi}{\delta_{\tau_2}} }
    }
  }
}

\newcommand{\RepairTermOneInsPi}{
  \inferrule*
  [lab=\RInsPi]
  {
    {\turnstile
      {\dcontext{E}{\delta_E}{\Gamma}{\delta_\Gamma}}
      {\freshOne{\chi}{\MathSame}{\delta_\chi}}
    }
    \and
    {\MathPatches{\chi}{\delta_\chi}{x}}
    \and
    {\MathPatches{\tau}{\delta_{\tau_1}}{\tau_1}}
    \\\\\\
    \turnstile
    {\dcontext{E}{\delta_E}{\Gamma}
      {\MathIns{\MathLocalAssum{x}{\tau_1}}{\delta_\Gamma}}
    }
    { \repair{t}{\delta_{t}}{\tau}
      {\delta_{\tau_2}[\subst{\MathReplace{\chi}}{\MathReplace{x}}]}
    }
  }
  {
    \turnstile
    { \dcontext{E}{\delta_E}{\Gamma}{\delta_\Gamma} }
    { \repair
      {t}
      { \MathInsLam{x}{\delta_{t}} }
      {\tau}
      { \MathInsPi{\delta_{\tau_1}}{\chi}{\delta_{\tau_2}} }
    }
  }
}


\begin{Rules}{fig:repair-term-1}%
{ Rules for repairing terms, diff-directed ($\repairTermOneOp$, part 1/2) }

\begin{mathpar}
  \RepairTermOneSamePi{}

  \RepairTermOneSameOther{}

  \RepairTermOneModPi{}

  \RepairTermOneInsPi{}
\end{mathpar}

\end{Rules}


\newcommand{\RepairTermOneDropPi}{
  \inferrule*
  [lab=\RDropPi]
  {
    {\turnstile
      {\dcontext{E}{\delta_E}{\Gamma}{\delta_\Gamma}}
      {\freshTwo{\blackdiff{x}{\MathSame}}{\blackdiff{\chi}{\MathSame}}{z}}
    }
    \\\\\\
    {\turnstile
      {\dcontext
        {E}{\delta_E}
        {\MathCons{\MathLocalAssum{z}{\tau_1}}{\Gamma}}
        {\MathDrop{\delta_\Gamma}}
      }
      {\repair
        {t[\subst{x}{z}]}
        {\delta_{t}}
        {\tau[\subst{\chi}{z}]}
        {\delta_{\tau_2}}
      }
    }
  }
  {
    \turnstile
    { \dcontext{E}{\delta_E}{\Gamma}{\delta_\Gamma} }
    { \repair
      {\MathLam{x}{t}}
      {\MathDropLam{\delta_{t}}}
      {\MathPi{\tau_1}{\chi}{\tau_2}}
      {\MathDropPi{\delta_{\tau_2}}}
    }
  }
}

\newcommand{\RepairTermOnePermutePis}{
  \inferrule*
  [lab=\RPermutePis]
  {\turnstile
    { \dcontext{E}{\delta_E}{\Gamma}{\delta_\Gamma} }
    {\repair
      % { \MathLam{x_{p(1)} \ldots x_{p(|p|)}}{t} }
      { \MathLam{\permutationRange{x_{p(i)}}}{t} }
      { \delta }
      { \mkMathPiRaw{\permutationRange{(\chi_{p(i)} : \tau_{p(i)})}}{\tau}{}{} }
      { \dtau{} }
    }
  }
  {\turnstile
    { \dcontext{E}{\delta_E}{\Gamma}{\delta_\Gamma} }
    {\repair
      % { \MathLam{x_1 \ldots x_{|p|}}{t} }
      { \MathLam{\permutationRange{x_{i}}}{t} }
      { \MathPermuteLams{p}{\delta} }
      { \mkMathPiRaw{\permutationRange{(\chi_{i} : \tau_{i})}}{\tau}{} }
      { \MathPermutePis{p}{\dtau{}} }
    }
  }
}

\newcommand{\RepairTermOneReplace}{
  \inferrule*
  [lab=\RReplace]
  { }
  {\turnstile
    { \dcontext{E}{\delta_E}{\Gamma}{\delta_\Gamma} }
    { \repair
      {t}
      {\MathReplace{\MathAnnot{\MathHole}{\tau'}}}
      {\tau}
      {\MathReplace{\tau'}}
    }
  }
}

\newcommand{\RepairTermOneOtherwise}{
  \inferrule*
  [lab=\RepairTermPrefix-Otherwise]
  {\turnstile%
    { \dcontext{E}{\delta_E}{\Gamma}{\delta_\Gamma} }
    { \repairTermWithoutType%
      { t }
      { \delta_t }
    }
  }
  {\turnstile%
    { \dcontext{E}{\delta_E}{\Gamma}{\delta_\Gamma} }
    {\repair%
      { t }
      { \delta_t }
      { \tau }
      { \delta_\tau }
    }
  }
}

\begin{Rules}{fig:repair-term-2}{ Rules for repairing terms, diff-directed ($\repairTermOneOp$, part 2/2) }

\begin{mathpar}
  \RepairTermOneDropPi{}

  \RepairTermOnePermutePis{}

  \RepairTermOneReplace{}

  \RepairTermOneOtherwise{}
\end{mathpar}

\end{Rules}


$\repairTermOneOp$ (described in Figures~\ref{fig:repair-term-1}
and~\ref{fig:repair-term-2}) is a repair algorithm for terms, directed by the
syntax of the diff for the type of the term being repaired.  We will describe
these rules one by one, as they are often quite complex.

\paragraph{\rulename{\RSamePi}}

$$\RepairTermOneSamePi{}$$

This rule applies when the type of the term being repaired used to be a
dependent function type, such as $\MathPi{\tau_{1}}{\chi}{\tau_{2}}$, and this
type has not been modified.  In order to not repeat ourselves, this rule simply
expands $\MathSameTerm{}$ into the equivalent
$\MathModPi{\MathSameTerm{}}{\MathSameBinder{}}{\MathSameTerm{}}$, so that rule
\rulename{\RModPi} applies.

\paragraph{\rulename{\RSame}}

$$\RepairTermOneSameOther{}$$

This rule applies when the type of the term being repaired has not been
modified, for all other cases than a dependent function type.  Naively, one
might expect the output to simply be $\out{\MathSame}$, however, this would not
account for changes to definitions in the global environment and local context,
that may have repercussions within $t$.  The rule therefore defers to the
term-directed repair algorithm $\repairTermTwoOp$.

\paragraph{\rulename{\RModPi}}

$$\RepairTermOneModPi{}$$

This rule requires a lot of care.  It applies when the term is an explicit term
abstraction, of the form $\MathLam{x}{t}$, its type is an explicit type
abstraction, of the form $\MathPi{\tau_{1}}{\chi}{\tau_{2}}$, and the diff to
its type is an explicit modification of the type abstraction, of the form
$\MathModPi{\delta_{\tau_1}}{\delta_\chi}{\delta_{\tau_2}}$.

Perhaps surprisingly, the first thing we'll do is possibly find a fresh name for
$x$.  Consider that the term abstraction under repair could be $(\MathLam{x}{f
x})$, where $f$ was a function in scope.  Now consider what happens in the rare
but possible case where the user renamed $f$ into $x$.  If we aim to repair the
term abstraction, we will want to rename the occurrence of $f$ into $x$, but it
would be wrong to do so, because this $x$ would be accidentally captured by the
term abstraction!  In order to repair such an $f$ within the term abstraction,
we had better alpha-rename it first, using a suitably fresh variable, say
$(\MathLam{y}{f y})$.  This lets us safely perform the renaming, obtaining
$(\MathLam{y}{x y})$.  We use a helper function, $\freshOneOp$, in order to
detect the conditions within which the variable $x$ would require freshening.
The output of the helper has type $\DeltaBinder$, and will be either
$\out{\MathSame{}}$ when the variable does not need to be refreshed, or
$\out{\MathReplace{y}}$ when the variable needs to be refreshed and $y$ is a
suitable fresh variable.

A similar issue arises when we want to repair the body of the abstraction: we
want to introduce the variable being abstracted over in our local context, but
it already has two names, $x$ in the term abstraction, and $\chi$ in the type
abstraction.  For the same reason as in the previous paragraph, neither $x$, nor
$\chi$, might always be suitable in the repaired program.  We need a new helper
function, $\freshTwoOp$, to pick a suitably fresh variable, $\out{z}$, that can
be substituted in both the term abstraction \emph{and} the type abstraction
without accidental capture.  In many cases, $z$ will simple be $x$ or $\chi$,
but in some cases, it might need to be a new variable.

Now armed with $\out{\delta_x}$ and $\out{z}$, we can introduce $(z : \tau_1)$
in the local context.  As for the local context diff, we preserve the binding,
using $\MathLocalAssum{\MathReplace{z}}{\delta_{\tau_{1}}}$.  Using
$\MathReplace{z}$ for the binder diff ensures that all occurrences of $z$, that
is, old occurrences of $x$, get properly rewritten into $z$.  We must now
perform appropriate substitutions so as to keep the body of the term
abstraction, the body of the type abstraction, and the diff of the body the type
abstraction, all in agreement.  For the body of the term abstraction, we
substitute $z$ for $x$.  For the body of the type abstraction, we substitute $z$
for $\chi$.  But how should we alter the diff of this body, that is,
$\delta_{\tau_{2}}$?  Looking back at the provenance of $\delta_{\tau_{2}}$, we
see it is a diff that has been created under the assumption that variable $\chi$
was to undergo modification $\delta_{\chi}$.  However, we changed this
resolution so that variable $\chi$ will instead become $x$.  Therefore, within
$\delta_{\tau_{2}}$, we should replace occurrences of $\chi'$, the variable that
$\delta_{\tau_{2}}$ was expecting, to $z$, the variable that will actually be in
scope.  Since the only diff construct that can mentions variables is the
``replace'' atomic diff operation, $\MathReplace{…}$, this means we only need to
substitute $\MathReplace{z}$ everywhere we see a free $\MathReplace{\chi'}$.

Repairing the body of the term abstraction gives us back a diff,
$\out{\delta_{t}}$.  However, remember that this diff was built with $z$
substituted for the bound variable.  Repairing the term abstraction does
\emph{not} require renaming $x$ to $z$ though, this was only necessary to
account for the dependent function type in repairing the body.  In fact,
remember we had decided that variable $x$ should change according to
$\delta_{x}$.  We can therefore build the appropriate output diff by replacing
the binder following $\delta_{x}$, and performing a diff substitution within
$\delta_{t}$, substituting $\MathReplace{x'}$ everywhere we see a free
$\MathReplace{z}$, where $x'$ is the variable chosen by $\delta_{x}$.

\paragraph{\rulename{\RInsPi}}

$$\RepairTermOneInsPi{}$$

This rule applies when the type of a term $t$ has changed to include a new type
abstraction, as in $\MathInsPi{\delta_{\tau_1}}{\chi}{\delta_{\tau_2}}$.  To
allow the term to receive this new argument, we will want to turn it into a term
abstraction.  But what name should we pick for the variable to abstract over?
The obvious candidate is $\chi$, the same variable that appears in the type
abstraction.  However, once again, there is a chance that some unrelated $\chi$
appears free in $t$.  Abstracting over the same variable would result in an
accidental capture!  We must pick a suitably fresh name, possibly $\chi$, in the
current context.  This is once again achieved using $\freshOneOp{}$, and we
obtain a diff, $\out{\delta_{\chi}}$, which will either be $\MathSame{}$ if it
is possible to name the binder $\chi$, or $\MathReplace{y}$ with a suitably
fresh $y$ otherwise.  Let us name $\out{x}$ the variable name that was picked.

We can compute the type that this $x$ should have, $\tau_{1}$, and add a binding
$\MathLocalAssum{x}{\tau_{1}}$ to the local context as we repair $t$.  Once
again, we must perform a diff substitution, because $\delta_{\tau_{2}}$ was
built under the assumption that the value abstracted over would be named $\chi$,
but we instead chose to name it $x$.  The resulting $\out{\delta_{t}}$ is how we
want to repair the body of our soon-to-be term abstraction.  The final output is
simply $\out{\MathInsLam{x}{\delta_{t}}}$.  We do \emph{not} need to undo the
diff substitution, because $\delta_{t}$ indeed appears under a binder named $x$.

\paragraph{\rulename{\RDropPi}}

$$\RepairTermOneDropPi{}$$

This rules applies when some explicit term abstraction, $\MathLam{x}{t}$, had an
explicit type abstraction for its type, $\MathPi{\tau_1}{\chi}{\tau_2}$, but no
longer receives this argument.  In this case, we will want to drop the $λ$, but
repair its body.  Once again, we will use $\freshTwoOp{}$ to pick a suitably
fresh variable that will neither be captured by $x$ in $t$, nor by $\chi$ in
$\tau_{2}$.  We simply add the binding $\MathLocalAssum{z}{\tau_{1}}$ to the
local context, and $\MathDrop{}$ for its diff.  Thanks to our lookup rules (from
Section~\ref{chick-lookup-diff-context}), any reference to $z$ will be replaced
with a deprecated variable name.  We simply need to repair the body at the
appropriate type and type diff.  The result, $\out{\delta_{\tau}}$, will contain
both repairs to $t$, and remove all references to the now obsolete binder $x$,
replacing those with some variable name, depending on the implementation of
$\deprecateOp{}$.

\paragraph{\rulename{\RPermutePis}}

$$\RepairTermOnePermutePis{}$$

This rule applies when the term being repaired is a sequence of term
abstractions $\MathLam{\permutationRange{x_{i}}}{t}$, while its type used to be
some telescope of type abstractions
$\mkMathPiRaw{\permutationRange{(\chi_{i}:\tau_{i})}}{\tau}{}{}$, and its type
has undergone a permutation $\MathPermutePis{p}{\dtau{}}$.  Note that the
sequence of term abstractions, as well as the telescope of type abstractions,
could be longer than the length of the permutation, and of different lengths.
All that matters is that we have \emph{at least} $|p|$ explicit $\lambda$s in
the term, and \emph{at least} $|p|$ explicit $\Pi$s in the type abstraction, so
that we can perform the permutation.  When this is possible, the rule is fairly
simple: it applies the permutation to both the term and the type, and repairs
the resulting term against the resulting type.  The output of this result is
then permuted to obtain the result of the original problem.

\paragraph{\rulename{\RReplace}}

$$\RepairTermOneReplace{}$$

When the type of a term has been replaced by an arbitrary type $\tau'$, there is
nothing meaningful that can be done to repair the old term.  We have to give up
and return a typed hole, $(\MathAnnot{\MathHole{}}{\tau'})$.

\paragraph{\rulename{\RepairTermPrefix-Otherwise}}

$$\RepairTermOneOtherwise{}$$

Finally, when none of the other rules apply, we have exhausted the ways in which
we could use the type to guide the repair.  There are still opportunities for
repairing the term, but they are no longer directed by its type, only by the
syntax of the term itself.  Therefore, we simply forget about the type
information, and call $\repairTermThreeOp{}$, which we will discuss now.


\subsubsection*{Term-directed term repair algorithm}

The algorithm for $\repairTermThreeOp$ is directed by the syntax of the term
being repaired.

For variables (\rulename{\UTRVar}) and function applications
(\rulename{\UTRApp}), we use another helper, $\repairArgsOp$, to compute the
diff.  First, we syntactically extract as many application nodes as possible,
building a list of the arguments, until the function symbol (in the case of a
variable, this corresponds to an empty list of arguments).  Once we lookup the
type diff for the function symbol (resp. $\delta_{\tau_v}$ and
$\delta_{\tau_f}$), we can start producing the output diff.  An example will be
illustrative of the $\repairArgsOp$ algorithm: consider the case where a
function $f$ had type $(A \rightarrow C)$ and undergoes the transformation
$\MathModPi{\MathSame}{\MathSame}{(\MathInsPi{B}{\_}{\MathSame})}$, yielding the
type $(A \rightarrow B \rightarrow C)$.  Suppose the original code contains a
call to $f$, for instance $(f\ a)$.  A repaired version of this call would look
like $(f\ a\ (\MathAnnot{\MathHole}{B}))$.  This means the repair diff should be
$(\MathInsApp{(\MathModApp{\MathSame}{\MathSame})}{(\MathAnnot{\MathHole}{B})})$.
Notice how the outermost term modification, $\MathInsAppOp$, corresponds to the
innermost type modification $\MathInsPiOp$.  This is not surprising: a function
$g$ whose type is $(A \rightarrow (B \rightarrow (C \rightarrow D)))$, and its
application to arguments $(((g\ a)\ b)\ c)$, exhibit the same inversion.  For
this reason, $\repairArgsOp$ works by processing the $\Pi$-telescope type diff
(i.e. a sequence of nested $\Pi$s) from outside-in, and builds the $\App$ term
diff from inside-out.  We omit the rules for $\repairArgsOp$ as they would be
cumbersome and not enlightening.

We also omit the rules for repairing a \coqinline{match} construct behind the
symbol $\repairBranchesOp$.  Once the discriminee is repaired, the branches must
be repaired: first, the discriminated inductive data type is identified, and its
diff is retrieved from the global environment.  From this, the list of branches
is repaired, with proper care for the binding structure of the patterns.  Our
current implementation only supports non-nested pattern matches, and proceeds by
computing a permutation from the order in which patterns appear in the program
to the order in which they appear in the inductive data type declaration.  The
process of repairing the branches is then straightforward: if a constructor was
added, a branch is added, if a constructor was removed, its branch (if any) is
removed, and modifications to constructors impact the pattern and arm of its
matching branch.  Notice that we do not support the full power of
dependently-typed elimination yet: our \coqinline{match} construct does not have
an index binding declaration or a return type annotation (respectively known as
the \coqinline{in}-clause and the \coqinline{return}-clause in Gallina).

Also of note, in \rulename{\UTRAnnot}, we actually retrieve type information
from the annotation, and can repair the annotated term with more information.
In practice however, we don't expect much code to contain type annotations.
Finally, the rule \rulename{\UTROtherwise} captures terms such as holes, and
universes, which we repair by doing nothing.

\begin{Rules}{rules-unknown-repair}{ Rules for $\repairTermThreeOp$, the
term-directed repair algorithm. }

\begin{mathpar}
  {
    \inferrule*
    [lab=\UTRVar]
    {
      {\turnstile
        { \dcontext{E}{\delta_E}{\Gamma}{\delta_\Gamma} }
        { \declDiff{v}{\delta_v}{\delta_{\tau_v}} }
      }
      \and
      {\turnstile
        { \dcontext{E}{\delta_E}{\Gamma}{\delta_\Gamma} }
        { \repairArgs{ [] }{\tau_v}{\delta_{\tau_v}}{\delta_v}{\delta}
        }
      }
    }
    {\turnstile
      { \dcontext{E}{\delta_E}{\Gamma}{\delta_\Gamma} }
      { \repairTermWithoutType{v}{\delta} }
    }
  }

  {
    \inferrule*
    [lab=\UTRApp]
    {
      {\turnstile
        { \dcontext{E}{\delta_E}{\Gamma}{\delta_\Gamma} }
        { \declDiff{f}{\delta_f}{\delta_{\tau_f}} }
      }
      \and
      {\turnstile
        { \dcontext{E}{\delta_E}{\Gamma}{\delta_\Gamma} }
        { \repairArgs{[a_1, \ldots, a_n]}{\tau_f}{\delta_{\tau_f}}
          {\delta_f}{\delta}
        }
      }
    }
    {\turnstile
      { \dcontext{E}{\delta_E}{\Gamma}{\delta_\Gamma} }
      { \repairTermWithoutType{ f\ a_1\ \ldots\ a_n }{\delta} }
    }
  }

  {
    \inferrule*
    [lab=\UTRPi]
    {
      {\turnstile
        {\dcontext{E}{\delta_E}{\Gamma}{\delta_\Gamma}}
        {\freshOne{x}{\MathSame}{\delta_x}}
      }
      \and
      {\turnstile
        { \dcontext { e } { \delta_E }
          { \Gamma }
          { \delta_\Gamma }
        }
        { \repair { \tau_1 } { \delta_{\tau_1} } { \MathType } { \MathSame } }
      }
      \\
      {\turnstile
        { \dcontext { e } { \delta_E }
          { \MathCons{\MathLocalAssum{x}{\tau_1}}{\Gamma} }
          { \MathMod{\MathLocalAssum{\delta_x}{\delta_{\tau_1}}}{\delta_\Gamma} }
        }
        { \repair { \tau_2 } { \delta_{\tau_2} } { \MathType } { \MathSame } }
      }
    }
    {\turnstile
      { \dcontext{E}{\delta_E}{\Gamma}{\delta_\Gamma} }
      { \repairTermWithoutType{ \MathPi{\tau_1}{x}{\tau_2} }
        { \MathModPi{\delta_{\tau_1}}{\delta_x}{\delta_{\tau_2}} }
      }
    }
  }

  {
    \inferrule*
    [lab=\UTRMatch]
    {
      {\turnstile
        { \dcontext{E}{\delta_E}{\Gamma}{\delta_\Gamma} }
        { \repairTermWithoutType{t}{\delta_t} }
      }
      \and
      {\turnstile
        { \dcontext{E}{\delta_E}{\Gamma}{\delta_\Gamma} }
        { \repairBranches{t}{[b_1 \ldots b_n]}{\delta_b} }
      }
    }
    {\turnstile
      {\dcontext{E}{\delta_E}{\Gamma}{\delta_\Gamma}}
      {\repairTermWithoutType
        {\MathMatch{t}{b_1 \ldots b_n}}
        {\delta}
      }
    }
  }

  {
    \inferrule*
    [lab=\UTRAnnot]
    {
      {\turnstile
        { \dcontext{E}{\delta_E}{\Gamma}{\delta_\Gamma} }
        { \repairTermWithoutType{\tau}{\delta_\tau} }
      }
      \and
      {\turnstile
        { \dcontext{E}{\delta_E}{\Gamma}{\delta_\Gamma} }
        { \repair{t}{\delta_t}{\tau}{\delta_\tau} }
      }
    }
    {\turnstile
      { \dcontext{E}{\delta_E}{\Gamma}{\delta_\Gamma} }
      { \repairTermWithoutType{\MathAnnot{t}{\tau}}{\MathAnnot{\delta_t}{\delta_\tau}} }
    }
  }

  % {
  %   \inferrule*
  %   [lab=\UTRType]
  %   { u \in \{ \MathProp, \MathSet, \MathType \} }
  %   {\turnstile
  %     { \dcontext{E}{\delta_E}{\Gamma}{\delta_\Gamma} }
  %     { \repairTermWithoutType{u}{\MathSame} }
  %   }
  % }

  {
    \inferrule*
    [lab=\UTROtherwise]
    {  }
    {\turnstile
      {\dcontext{E}{\delta_E}{\Gamma}{\delta_\Gamma}}
      {\repairTermWithoutType{t}{\MathSame}}
    }
  }

  \end{mathpar}

\end{Rules}

