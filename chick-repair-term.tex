\subsection{Repairing terms}\label{chick-design-repair-term}

The repair algorithm for terms is described formally in
Figures~\ref{fig:repair-term-1}, \ref{fig:repair-term-2},
\ref{fig:repair-term-two-1}, and~\ref{fig:repair-term-two-2} using the following
judgments:

{
  \[
    {\turnstile
      { \dcontext{E}{\delta_E}{\Gamma}{\delta_\Gamma} }
      {\repair{t}{\delta_t}{\tau}{\delta_\tau}}
    }
  \]
  % \[
  %   {\turnstile
  %     { \dcontext{E}{\delta_E}{\Gamma}{\delta_\Gamma} }
  %     {\genericrepair{t}{\delta_t}{\tau}}
  %   }
  % \]
  \[
    {\turnstile
      { \dcontext{E}{\delta_E}{\Gamma}{\delta_\Gamma} }
      {\repairTermWithoutType{t}{\delta_t}}
    }
  \]
}%
%
The first judgment (described in Figures~\ref{fig:repair-term-1}
and~\ref{fig:repair-term-2}) describes how a term is repaired when we know both
its old type $\tau$, and how that type got modified $\delta_\tau$.  It is
syntax-directed by the structure of $\delta_\tau$.  The second judgment
(described in Figures~\ref{fig:repair-term-two-1}
and~\ref{fig:repair-term-two-2}) describes how a term is repaired when we no
longer make decisions based on its old type and type diff: this happens for
constructs whose type does not inform us about the type of its constituents (for
instance, function application), and for simple terms (holes, universes,
variables).

\paragraph{Notation} The first judgment can be read as follows:

\begin{itemize}

  \item \textbf{Top}: Given that a term $t$ had type $\tau$ in local context
$\Gamma$ and global environment $E$,

  \item \textbf{Bottom left}: and the global environment underwent modification
$\delta_{E}$,

  \item \textbf{Bottom left}: and the local context underwent modification
$\delta_{\Gamma}$,

  \item \textbf{Bottom right}: and the type underwent modification
$\delta_{\tau}$,

  \item \textbf{Right}: then, we suggest to repair the term $t$ with diff
$\out{\delta_{t}}$.

\end{itemize}

\noindent The second judgment can be read analogously, but does not have access
to typing information for the term being repaired.  Importantly, the first
judgment assumes the diff $\delta_\tau$ is repaired already: it is the
responsibility of the caller to ensure so.

\subsubsection*{Diff-directed term repair algorithm}

The algorithm for $\repairTermOneOp$ is directed by the syntax of the diff for
the type of the term being repaired.

The rules~\rulename{\RSamePi} and~\rulename{\RSame} apply when the type has not
been modified.  The first one turns the $\MathSame$ diff for a $\Pi$-type into a
$\MathModPiOp$, to defer the handling of the $\lambda$'s binder, and the
populating of the local context, to the rule \rulename{\RModPi} described
further below.  The second one deals with all other types.  Naively, one might
expect the output to simply be $\MathSame$, however, this would not account for
changes to definitions in the global environment and local context.  The rule
therefore defers this work to the type-directed term repair algorithm
$\repairTermTwoOp$.

The rule that handles the modification of a dependent $\Pi$-type for an explicit
$\lambda$ function, \rulename{\RModPi}, requires a lot of care.  A simple
version of it would not alter the binder $x$ in the term abstraction.
Unfortunately, there is a potential for a binder conflict.  Consider the term
$(\MathLam{a}{b\ a})$ defined in a global environment where $b$ is bound, and
suppose the user renamed that global $b$ into $a$.  The algorithm will
eventually need to replace the occurrence of $b$ in the term abstraction with an
$a$, but in order to do so without capture, we had better $\alpha$-rename the
abstraction's binder with a variable that is fresh in the repaired program.  We
use $\freshOneOp$ again to pick a suitable binder.

A similar issue arises when we want to repair the body of the abstraction: we
want to instantiate the term abstraction's binder $x$ and the type abstraction's
binder $\chi$ with the same variable.  Unfortunately, we can't always choose $x$
because the body of the type abstraction $\tau_2$ might depend on an existing
$x$ in either the global environment or local context, neither can we always
choose $\chi$ because the body of the term abstraction $t$ might depend on such
a $\chi$ too.  The $\freshTwoOp$ algorithm picks such a suitably-fresh variable
$z$.  This $z$ is only necessary for the algorithm, and will never be shown to
the programmer.

Now armed with $\delta_x$ and $z$, we can introduce $(z : \tau_1)$ in the local
context, and have it not need be renamed in the diffed context.  In this
enriched context, we can repair the body of the abstraction, $t$, after having
performed the substitution of $z$ for $x$.  We must perform similar
substitutions in $\tau_2$ and $\delta_{\tau_2}$.  We obtain the diff
$\delta_\tau$, which repairs the body of the abstraction when its binder is
named $z$.  However, in the original program, we will use $\delta_x$ to repair
the binder.  We must therefore substitute back instances of $\MathReplace{x}$
where we find free instances of $\MathReplace{x}$ in $\delta_\tau$.  The same
considerations must be taken into account when a $\Pi$ is inserted
(\rulename{\RInsPi}).

To deal properly with $\Pi$-permutations (\rulename{\RPermutePis}), we must be
able to syntactically extract as many $\lambda$s as the length of the
permutation.  When we can do so, the repair proceeds recursively after having
permuted both the term and the type abstractions.  When the original type $\tau$
gets replaced by an arbitrary type $\tau'$ (\rulename{\RReplace}), we have no
better option than inserting a typed hole.  The final rule
(\rulename{\RepairTermPrefix-Otherwise-\GenericRepairPrefix}) triggers if none
of the previous rules have, and dispatches the repair to the type-directed term
repair algorithm $\repairTermTwoOp$.

\begin{Rules}{fig:repair-term}{ Rules for repairing terms, diff-directed ($\repairTermOneOp$) }

\begin{mathpar}
  {
    \inferrule*
    [lab=\RSamePi]
    {
      {\turnstile
        { \dcontext{E}{\delta_E}{\Gamma}{\delta_\Gamma} }
        { \repair
          {\MathLam{x}{t}}
          {\delta}
          {\MathPi{\tau_1}{\chi}{\tau_2}}
          {\MathModPi{\MathSame}{\MathSame}{\MathSame}}
        }
      }
    }
    {\turnstile
      { \dcontext{E}{\delta_E}{\Gamma}{\delta_\Gamma} }
      { \repair
        {\MathLam{x}{t}}
        {\delta}
        {\MathPi{\tau_1}{\chi}{\tau_2}}
        {\MathSame}
      }
    }
  }

  {
    \inferrule*
    [lab=\RSame]
    {\turnstile
      { \dcontext{E}{\delta_E}{\Gamma}{\delta_\Gamma} }
      { \repairTermWithoutType{t}{\delta_{t}}}
    }
    {\turnstile
      { \dcontext{E}{\delta_E}{\Gamma}{\delta_\Gamma} }
      { \repair{t}{\delta_{t}}{\tau}{\MathSame} }
    }
  }

  \newcommand{\RepairTermDiffModPi}{
  \inferrule*
  [lab=\RModPi]
  {
    {\turnstile%
      {\dcontext{E}{\delta_E}{\Gamma}{\delta_\Gamma}}
      {\freshOne{x}{\MathSame}{\delta_x}}
    }
    \and
    {\turnstile%
      {\dcontext{E}{\delta_E}{\Gamma}{\delta_\Gamma}}
      {\freshTwo{\blackdiff{x}{\delta_x}}{\blackdiff{\chi}{\delta_\chi}}{z}}
    }
    \\\\\\
    {\MathPatches{x}{\delta_x}{x'}}
    \and
    {\MathPatches{\chi}{\delta_\chi}{\chi'}}
    \\\\\\
    {\turnstile%
      { \dcontext{E}{\delta_E}
        { \MathCons{\MathLocalAssum{z}{\tau_1}}{\Gamma} }
        { \MathMod{\MathLocalAssum{\MathSame}{\delta_{\tau_1}}}{\delta_\Gamma} }
      }
      { \repair%
        {t[\subst{x}{z}]}
        {\delta_{t}}
        {\tau_2[\subst{\chi}{z}]}
        {\delta_{\tau_2}[\subst{\MathReplace{\chi'}}{\MathReplace{z}}]}
      }
    }
  }
  {\turnstile%
    { \dcontext{E}{\delta_E}{\Gamma}{\delta_\Gamma} }
    {\repair%
      { \MathLam{x}{t} }
      { \MathModLam{\delta_x}{\delta_t[\subst{\MathReplace{z}}{\MathReplace{x'}}]} }
      { \MathPi{\tau_1}{\chi}{\tau_2} }
      { \MathModPi{\delta_{\tau_1}}{\delta_\chi}{\delta_{\tau_2}} }
    }
  }
}

  \RepairTermDiffModPi{}

  {
    \inferrule*
    [lab=\RInsPi]
    {
      {\turnstile
        {\dcontext{E}{\delta_E}{\Gamma}{\delta_\Gamma}}
        {\freshOne{\chi}{\MathSame}{\delta_\chi}}
      }
      \and
      {\MathPatches{\chi}{\delta_\chi}{x}}
      \and
      {\MathPatches{\tau}{\delta_{\tau_1}}{\tau_1}}
      \\\\\\
      \turnstile
      {\dcontext{E}{\delta_E}{\Gamma}
        {\MathIns{\MathLocalAssum{x}{\tau_1}}{\delta_\Gamma}}
      }
      { \repair{t}{\delta_{t}}{\tau}
        {\delta_{\tau_2}[\subst{\MathReplace{\chi}}{\MathReplace{x}}]}
      }
    }
    {
      \turnstile
      { \dcontext{E}{\delta_E}{\Gamma}{\delta_\Gamma} }
      { \repair
        {t}
        { \MathInsLam{x}{\delta_{t}} }
        {\tau}
        { \MathInsPi{\delta_{\tau_1}}{\chi}{\delta_{\tau_2}} }
      }
    }
  }

  {
    \inferrule*
    [lab=\RDropPi]
    {
      {\turnstile
        {\dcontext{E}{\delta_E}{\Gamma}{\delta_\Gamma}}
        {\freshTwo{\blackdiff{x}{\MathSame}}{\blackdiff{\chi}{\MathSame}}{z}}
      }
      \\
      {\turnstile
        {\dcontext
          {E}{\delta_E}
          {\MathCons{\MathLocalAssum{z}{\tau_1}}{\Gamma}}
          {\MathDrop{\delta_\Gamma}}
        }
        {\repair
          {t[\subst{x}{z}]}
          {\delta_{t}}
          {\tau[\subst{\chi}{z}]}
          {\delta_{\tau_2}}
        }
      }
    }
    {
      \turnstile
      { \dcontext{E}{\delta_E}{\Gamma}{\delta_\Gamma} }
      { \repair
        {\MathLam{x}{t}}
        {\MathDropLam{\delta_{t}}}
        {\MathPi{\tau_1}{\chi}{\tau_2}}
        {\MathDropPi{\delta_{\tau_2}}}
      }
    }
  }

  {
    \inferrule*
    [lab=\RPermutePis]
    {\turnstile
      { \dcontext{E}{\delta_E}{\Gamma}{\delta_\Gamma} }
      {\repair
        { \MathLam{x_{p(1)} \ldots x_{p(|p|)}}{t} }
        { \delta }
        { \mkMathPiRaw{(\chi_{p(1)} : \tau_{p(1)}) \ldots (\chi_{p(|p|)} : \tau_{p(|p|)})}{\tau}{}{} }
        { \dtau{} }
      }
    }
    {\turnstile
      { \dcontext{E}{\delta_E}{\Gamma}{\delta_\Gamma} }
      {\repair
        { \MathLam{x_1 \ldots x_{|p|}}{t} }
        { \MathPermuteLams{p}{\delta} }
        { \mkMathPiRaw{(\chi_{1} : \tau_i) \ldots (\chi_{|p|} : \tau_{|p|})}{\tau}{} }
        { \MathPermutePis{p}{\dtau{}} }
      }
    }
  }

  {
    \inferrule*
    [lab=\RReplace]
    { }
    {\turnstile
      { \dcontext{E}{\delta_E}{\Gamma}{\delta_\Gamma} }
      { \repair
        {t}
        {\MathReplace{\MathAnnot{\MathHole}{\tau'}}}
        {\tau}
        {\MathReplace{\tau'}}
      }
    }
  }

  {
    \inferrule*
    [lab=\RepairTermPrefix-Otherwise-\GenericRepairPrefix]
    {\turnstile
      { \dcontext{E}{\delta_E}{\Gamma}{\delta_\Gamma} }
      { \genericrepair
        { t }
        { \delta_t }
        { \tau }
      }
    }
    {\turnstile
      { \dcontext{E}{\delta_E}{\Gamma}{\delta_\Gamma} }
      {\repair
        { t }
        { \delta_t }
        { \tau }
        { \delta_\tau }
      }
    }
  }

\end{mathpar}

\end{Rules}


\subsubsection*{Term-directed term repair algorithm}

$\repairTermThreeOp$ is an algorithm directed by the syntax of the term being
repaired, used when no information is known about its type and type diff.

For repairing variables (Rule \rulename{\UTRVar}) and function applications
(Rule \rulename{\UTRApp}), we use another helper, $\repairArgsOp$, to compute
the diff.  First, we syntactically extract as many application nodes as
possible, building a list of the arguments, until the function symbol (in the
case of a variable, this corresponds to an empty list of arguments).  Once we
lookup the type diff for the function symbol (resp. $\delta_{\tau_v}$ and
$\delta_{\tau_f}$), we can start producing the output diff.  An example will be
illustrative of the $\repairArgsOp$ algorithm: consider the case where a
function $f$ had type $(A \rightarrow C)$ and undergoes the transformation
$\MathModPi{\MathSame}{\MathSame}{(\MathInsPi{B}{\_}{\MathSame})}$, yielding the
type $(A \rightarrow B \rightarrow C)$.  Suppose the original code contains a
call to $f$, for instance $(f\ a)$.  A repaired version of this call would look
like $(f\ a\ (\MathAnnot{\MathHole}{B}))$.  This means the repair diff should be
$(\MathInsApp{(\MathModApp{\MathSame}{\MathSame})}{(\MathAnnot{\MathHole}{B})})$.
Notice how the outermost term modification, $\MathInsAppOp$, corresponds to the
innermost type modification $\MathInsPiOp$.  This is not surprising: a function
$g$ whose type is $(A \rightarrow (B \rightarrow (C \rightarrow D)))$, and its
application to arguments $(((g\ a)\ b)\ c)$, exhibit the same inversion.  For
this reason, $\repairArgsOp$ works by processing the $\Pi$-telescope type diff
(i.e. a sequence of nested $\Pi$s) from outside-in, and builds the applications
term diff from inside-out.  We omit the rules for $\repairArgsOp$ as they would
be cumbersome and not enlightening.

We also omit the rules for repairing a \coqinline{match} construct behind the
symbol $\repairBranchesOp$.  Once the discriminee is repaired, the branches must
be repaired: first, the discriminated inductive data type is identified, and its
diff is retrieved from the global environment.  From this, the list of branches
is repaired, with proper care for the binding structure of the patterns.  Our
current implementation only supports non-nested pattern matches, and proceeds by
computing a permutation from the order in which patterns appear in the program
to the order in which they appear in the inductive data type declaration.  The
process of repairing the branches is then straightforward: if a constructor was
added, a branch is added, if a constructor was removed, its branch (if any) is
removed, and modifications to constructors impact the pattern and arm of its
matching branch.  Notice that we do not support the full power of
dependently-typed elimination yet: our \coqinline{match} construct does not have
an index binding declaration or a return type annotation (respectively known as
the \coqinline{in}-clause and the \coqinline{return}-clause in \Gallina{}).

Also of note, in Rule \rulename{\UTRAnnot}, we actually retrieve type
information from the annotation, and can repair the annotated term with more
information.  In practice however, we don't expect much code to contain type
annotations.  Finally, Rule \rulename{\UTROtherwise} captures terms such as
holes, and universes, which we repair by doing nothing.

\begin{Rules}
  {rules-unknown-repair}
  { Rules for $\repairTermThreeOp$, the term-directed repair algorithm. }

\begin{mathpar}
  {
    \inferrule*
    [lab=\UTRVar]
    {
      {\turnstile
        { \dcontext{E}{\delta_E}{\Gamma}{\delta_\Gamma} }
        { \declDiff{v}{\delta_v}{\delta_{\tau_v}} }
      }
      \and
      {\turnstile
        { \dcontext{E}{\delta_E}{\Gamma}{\delta_\Gamma} }
        { \repairArgs{ [] }{\tau_v}{\delta_{\tau_v}}{\delta_v}{\delta}
        }
      }
    }
    {\turnstile
      { \dcontext{E}{\delta_E}{\Gamma}{\delta_\Gamma} }
      { \repairTermWithoutType{v}{\delta} }
    }
  }

  {
    \inferrule*
    [lab=\UTRApp]
    {
      {\turnstile
        { \dcontext{E}{\delta_E}{\Gamma}{\delta_\Gamma} }
        { \declDiff{f}{\delta_f}{\delta_{\tau_f}} }
      }
      \and
      {\turnstile
        { \dcontext{E}{\delta_E}{\Gamma}{\delta_\Gamma} }
        { \repairArgs{[a_1, \ldots, a_n]}{\tau_f}{\delta_{\tau_f}}
          {\delta_f}{\delta}
        }
      }
    }
    {\turnstile
      { \dcontext{E}{\delta_E}{\Gamma}{\delta_\Gamma} }
      { \repairTermWithoutType{ f\ a_1\ \ldots\ a_n }{\delta} }
    }
  }

  {
    \inferrule*
    [lab=\UTRPi]
    {
      {\turnstile
        {\dcontext{E}{\delta_E}{\Gamma}{\delta_\Gamma}}
        {\freshOne{x}{\MathSame}{\delta_x}}
      }
      \and
      {\turnstile
        { \dcontext { e } { \delta_E }
          { \Gamma }
          { \delta_\Gamma }
        }
        { \repair { \tau_1 } { \delta_{\tau_1} } { \MathType } { \MathSame } }
      }
      \\
      {\turnstile
        { \dcontext { e } { \delta_E }
          { \MathCons{\MathLocalAssum{x}{\tau_1}}{\Gamma} }
          { \MathMod{\MathLocalAssum{\delta_x}{\delta_{\tau_1}}}{\delta_\Gamma} }
        }
        { \repair { \tau_2 } { \delta_{\tau_2} } { \MathType } { \MathSame } }
      }
    }
    {\turnstile
      { \dcontext{E}{\delta_E}{\Gamma}{\delta_\Gamma} }
      { \repairTermWithoutType{ \MathPi{\tau_1}{x}{\tau_2} }
        { \MathModPi{\delta_{\tau_1}}{\delta_x}{\delta_{\tau_2}} }
      }
    }
  }

  {
    \inferrule*
    [lab=\UTRMatch]
    {
      {\turnstile
        { \dcontext{E}{\delta_E}{\Gamma}{\delta_\Gamma} }
        { \repairTermWithoutType{t}{\delta_t} }
      }
      \and
      {\turnstile
        { \dcontext{E}{\delta_E}{\Gamma}{\delta_\Gamma} }
        { \repairBranches{t}{[b_1 \ldots b_n]}{\delta_b} }
      }
    }
    {\turnstile
      {\dcontext{E}{\delta_E}{\Gamma}{\delta_\Gamma}}
      {\repairTermWithoutType
        {\MathMatch{t}{b_1 \ldots b_n}}
        {\delta}
      }
    }
  }

  {
    \inferrule*
    [lab=\UTRAnnot]
    {
      {\turnstile
        { \dcontext{E}{\delta_E}{\Gamma}{\delta_\Gamma} }
        { \repairTermWithoutType{\tau}{\delta_\tau} }
      }
      \and
      {\turnstile
        { \dcontext{E}{\delta_E}{\Gamma}{\delta_\Gamma} }
        { \repair{t}{\delta_t}{\tau}{\delta_\tau} }
      }
    }
    {\turnstile
      { \dcontext{E}{\delta_E}{\Gamma}{\delta_\Gamma} }
      { \repairTermWithoutType{\MathAnnot{t}{\tau}}{\MathAnnot{\delta_t}{\delta_\tau}} }
    }
  }

  % {
  %   \inferrule*
  %   [lab=\UTRType]
  %   { u \in \{ \MathProp, \MathSet, \MathType \} }
  %   {\turnstile
  %     { \dcontext{E}{\delta_E}{\Gamma}{\delta_\Gamma} }
  %     { \repairTermWithoutType{u}{\MathSame} }
  %   }
  % }

  {
    \inferrule*
    [lab=\UTROtherwise]
    {  }
    {\turnstile
      {\dcontext{E}{\delta_E}{\Gamma}{\delta_\Gamma}}
      {\repairTermWithoutType{t}{\MathSame}}
    }
  }

  \end{mathpar}

\end{Rules}


