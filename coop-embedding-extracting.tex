\section{Embedding and extracting programs}\label{coop-embedding-extracting}

In order to embed a new language into \Chick{}, we simply need to add a
constructor to the constructor of our \coqinline{Vernacular} and
\coqinline{Term} types.  For instance, we added support for \OCaml{} with the
following extension:

\begin{grammar}
<vernacular> ::= \ %trick LaTeX
\alt … \hfill (same as in Section~\ref{chick-syntax-programs})
\alt \coqinline{OCamlStructureItem} <ocaml-structure-item>

<term> ::= \ %trick LaTeX
\alt … \hfill (same as in Section~\ref{chick-syntax-terms})
\alt \coqinline{OCamlExpression} <ocaml-expression>
\end{grammar}

Now, our \define{embedder} can inspect the incoming abstract syntax tree, and
decide to map constructs from the source language to \Chick{} constructs, when
it makes sense to do so, or store the constructs in the extra constructors when
there is no matching concept in \Chick{}.

For instance, in the \OCaml{} embedder, we map constructs such as \OCaml{}'s
(non-polymorphic) variants to our notion of inductive data types, since the
concepts match.  We can also map most simple expressions to our term type, but
complex features that we do not support (say, nested pattern matching) get put
aside in the \coqinline{OCamlExpression} constructor.

The \define{extracter} acts as an inverse of the embedder, mapping \Chick{}
constructs back to their source language counterpart.  Constructs that have
been stashed in one of our constructors for unsupported language features
are simply restored as is, and \Chick{} constructs are mapped back to the
source language construct they came from.

Unfortunately, we sometimes need to map different source language abstract
syntax trees to the same \Chick{} abstract syntax tree.  For instance,
using \OCaml{} as an example again, the following two functions:

\begin{minted}{ocaml}
let f x y = 42
let g x = fun y -> 42
let h = fun x -> fun y -> 42
\end{minted}

map to similar \Chick{} terms:

\begin{minted}{coq}
Definition f : _ := λ x, λ y, 42.
Definition g : _ := λ x, λ y, 42.
Definition h : _ := λ x, λ y, 42.
\end{minted}

In order to try and preserve surface-level syntax as much as possible, we would
like to attach metadata about such source language syntactic choices in \Chick{}
terms.  This goals seems at odds with our intent of making \Coop{} extensible,
but we can actually achieve this somewhat easily by using the so-called ``Trees
that grow'' technique employed by~\mycite{najd2017trees} for the Glasgow
\Haskell{} Compiler (GHC).  Using generalized algebraic data types (GADTs), and
type families, this technique allows us to build abstract syntax trees for
\Chick{} that can be decorated with arbitrary metadata, and use type-level
computations to determine what this metadata may be for different constructors
in different contexts.

For instance, we can solve the previous problem by attaching metadata about the
number of function arguments that are marked as parameters, as opposed to being
abstracted over.

\begin{minted}{haskell}
-- We declare the existence of a family of types for storing the
-- metadata attached to a Definition.
type family DefinitionMetadata ξ

-- We attach this metadata to the Definition contructor.
data Vernacular ξ
  = Definition (DefinitionMetadata ξ) (DefinitionData ξ)
  | …

-- We instantiate the family for the OCaml language with an
-- integer representing the number of explicit parameters.
type instance (DefinitionMetadata OCaml) = Int
\end{minted}

With this information attached to our \mintinline{haskell}{Definition} nodes,
the extracter can pick the appropriate form, so as to preserve the original
syntactic choice.

\begin{minted}{haskell}
extracterVernacularOCaml ∷ Vernacular 'OCaml → StructureItem
extracterVernacularOCaml (Definition metadata data) =
  -- Here, metadata ∷ Int, tells us how many parameters should
  -- be before the equal sign, vs. bound by a lambda
\end{minted}

Sometimes, \Chick{} might need to come up with a new datum, for which there is
no sensible metadata it could attach to it.  For such cases, using an optional
type for the metadata (say, \mintinline{haskell}{Maybe Int} instead of
\mintinline{haskell}{Int} here) allows us to avoid this issue.

Overall, this technique gives us great flexibility, as different languages will
need different metadata for the same concepts.  The language of choice is
carried throughout the \Chick{} pipeline as a type-level tag, using \Haskell{}'s
\mintinline{haskell}{DataKinds} extension.  This also means that, anywhere in
the pipeline, we can readily choose to inspect the tag and do something specific
based on it.  For now, we only use this technique to pick the appropriate parser
and pretty-printer, but it could also be used to change parts of the repair
algorithm if some language needs specific, distinct support.
