\section{Optics to repair unknown language constructs}\label{coop-optics}

There is a big problem with the approach described in
Section~\ref{embedding-extracting}.  Consider the following \OCaml{} code:

\begin{minted}{ocaml}
type a = … (* some type definition *)

module SomeModule =
  struct
    type b = … (* some other type definition, depends on a *)
    let  f = … (* some function, depends on a *)
  end
\end{minted}

Our current implementation of \Chick{} does not have a notion of modules.
Therefore, \Coop{} has to stash away the whole module declaration in its
\coqinline{OCamlStructureItem}, and will provide no repair for it.  This is
unfortunate, because the module contains data type declarations, and function
declarations, all of which we know how to repair, and could attempt to, were
they visible.

We can find a solution to this problem using functional lenses.  \define{Lenses}
are an abstraction mechanism encompassing a pair of a getter and a setter for
some value within some datum.  They are often described in the simplified
version:

\begin{minted}{haskell}
data Lens s a = Lens
  { get ∷ s → a
  , set ∷ s → a → s
  }
\end{minted}

\noindent where \mintinline{haskell}{s} stands for the type of the store, or the
outer structure, and \mintinline{haskell}{a} stands for the type of the value
under focus.\footnote{Actual implementations of lenses are usually more complex,
with two additional parameters allowing the setter to output at a different
store type. Some instances also add a \mintinline{haskell}{Functor} constraint
to allow getting and setting in a functorial context.}

Lenses are useful because they are first-class, composable values that allow
modifying parts of a datum while ignoring the rest of it.  This matches quite
well with our goal: we wanted to zoom in on the inner parts of the module,
getting and setting the components of the module we understand, ignoring the
rest.  However, lenses only have one focus, whereas our data could contain
an arbitrary amount of components we might want to focus upon.

\define{Traversals} are a generalization of lenses with multiple foci.  They are
also an abstraction of a getter and a setter, except that they may operate on
multiple instances of the focus type within a datum.  In order to allow maximum
flexibility, traversals process the foci within an applicative functor.  Uniform
traversals over parameterized data structures can be automatically derived using
extensions like \Haskell{}'s \mintinline{haskell}{TemplateHaskell}, but in our
case, we will need to write our traversals manually.  This allows us to
specifically choose the order of the traversal, as well as selectively choose
whether some parts of data types should be traversed or not.  Let us see how we
can define such a traversal over \OCaml{} abstract syntax trees.  For this, we
will use type classes to capture data types that contain structures we are
interested in.  For the \OCaml{} language, we are interested in values of type
\mintinline{haskell}{Structure} (which correspond to our \coqinline{Vernacular}
data type), and values of type \mintinline{haskell}{CoreType} and
\mintinline{haskell}{Expression} (which correspond to our \coqinline{Term} data
type).

\begin{minted}{haskell}
class HasStructure t where
  structure ∷ Traversal' t Structure

class HasCoreType t where
  coreType ∷ Traversal' t CoreType

class HasExpression t where
  expression ∷ Traversal' t Expression
\end{minted}

We can then instantiate those classes for all of \OCaml{}'s abstract syntax tree
constructs, for instance:

\begin{minted}{haskell}
instance HasStructure ModuleExprDesc where
  structure f = case f of
    -- if something does not contain a Structure,
    -- we can skip it entirely:
    PmodIdent     i → PmodIdent <$> pure i
    -- if something contains a Structure right here,
    -- we can apply `f`
    PmodStructure s → PmodStructure <$> f s
    -- if something may contain Structures recursively,
    -- we keep traversing it
    PmodApply m1 m2 → PmodApply
                      <$> traverseOf structure f m1
                      <*> traverseOf structure f m2
    …
\end{minted}

\noindent Writing those instances is entirely systematic, so we believe it can
be automated using \Haskell{}'s \mintinline{haskell}{TemplateHaskell}
facilities, though we have not yet written such automation.

\subsection{TODO}

Depending on how the abstract syntax tree of the source language has been
defined, there are actually two distinct ways of building those traversals.
This corresponds to the following two ways of defining a data type of commands
over a datatype of expressions:

\begin{minted}{haskell}
-- Variant 1: concrete representation
data Language
  = DoSomething Language.Term
  | …

-- Variant 2: functorial representation
data LanguageF e
  = DoSomething e
  | …

type Language = LanguageF Language.Term
\end{minted}

With the \define{concrete representation}, the constructor
\mintinline{haskell}{DoSomething} may only ever contain values of type
\mintinline{haskell}{Language.Term}.  It is more rigid than the
\define{functorial representation}, wherein the type of expressions is
abstracted over.  In the functorial setting, one can instantiate the
\mintinline{haskell}{LanguageF} functor with different types, yielding a family
of concrete types.

With the functorial approach, \Coop{} can replace the source language's
expression type with its own type, essentially storing \Chick{} abstract syntax
trees within the source language's abstract syntax tree:

\begin{minted}{haskell}
data Vernacular f
  = …
  | ForeignLanguage (LanguageF f)
\end{minted}

In this case, \Chick{} can, given a traversal for \mintinline{haskell}{LanguageF
f}, transform all occurrences of \mintinline{haskell}{f} into
\mintinline{haskell}{Chick.Term}, and repair those using the applicative
functor.  While it is convenient to store our expressions within the original
abstract syntax tree, we are forced to replace all occurrences so that the types
match up.  Once all of this is done, we have the following endpoints:

\begin{minted}{haskell}
embed ∷ Vernacular Language.Term → Vernacular Chick.Term
embed = …

chick ∷ Vernacular Chick.Term → Vernacular Chick.Term
chick = …

extract ∷ Vernacular Chick.Term → Vernacular Language.Term
extract = …
\end{minted}

\noindent
and we can simply pipeline them together to obtain the resulting abstract syntax
tree, that must finally be processed by our pretty-printer (as described in
Section~\ref{coop-diff-pretty-printer}).

In the case of the concrete representation, because we cannot store \Chick{}
terms within the foreign language's abstract syntax tree, we must proceed with a
level of indirection: we can use an \define{indexed} traversal to extract a list
of all the values being traversed, turn those values into \Chick{} terms, repair
those \Chick{} terms, turn the repaired terms back into source language terms,
and finally reinsert those repaired terms in the proper locations using the
indexed traversal.  The process looks like the following partial code:

\begin{minted}{haskell}
itraverseLanguage ∷
  IndexedTraversal' Int Language.AST Language.Term
itraverseLanguage = …

getTerms ∷ Language.AST → [Language.Term]
getTerms = toListOf itraverseLanguage

-- From there, we do the following:
-- 1. map `embed`  to get a  [Chick.Term]    (unrepaired)
-- 2. map `repair` to get a  [Chick.Term]    (repaired)
-- 3. map `extract` to get a [Language.Term] (repaired)

insertRepairedTerms ∷
  [Language.Term] → Language.AST → Language.AST
insertRepairedTerms l =
  imapOf itraverseLanguage (λ index _ → l ‼ index)
\end{minted}

The indexed traversal allows us to know where to insert the repaired terms in
the original abstract syntax tree.

\subsection{Scope-aware traversals}

There remains yet another trouble with our traversals.  Consider the following
\OCaml{} code:

\begin{minted}{ocaml}
module SomeModule (A : ModuleTypeForA) =
  struct
    (* ... *)
  end
\end{minted}

Thanks to our traversals, we might be able to repair terms that appear within
this \OCaml{} functor\footnotemark{}.  Unfortunately, while glossing over the
\mintinline{ocaml}{module} construct, we also gloss over a binding construct,
introducing a binder \mintinline{ocaml}{A}.  This might create problems when
this module appears in an ambient context where some other variable
\mintinline{ocaml}{A} is bound.  In particular, if our repair algorithm is
trying to repair instances of the variable \mintinline{ocaml}{A}, it will,
erroneously, attempt to repair the local references to \mintinline{ocaml}{A},
mistaking them for references the outer \mintinline{ocaml}{A}.

\footnotetext{Note that \OCaml{} uses the term \emph{functor} to describe about
parameterized \emph{modules}, while, so far, this dissertation has been using
the term functor to describe functorial parameterized \emph{data types}.}

In order to avoid this issue, we can change our traversals so that they
accumulate bound variables as they traverse the source language abstract syntax
tree, and turn our focal points into a pair of the construct under focus, and
the list of binders it lives under.  With this information, we can easily alter
the term so that \Chick{} knows that those variables have been rebound.  This
can be achieved in two ways:

\begin{itemize}

  \item either by over-populating, before running the repair algorithm, its
local context with those binders (bound to an unknown type, represented as a
hole), and populating the local context diff with as many
$\MathMod{\MathSame{}}{…}$,

  \item or, by adding extraneous $\lambda$-abstractions to the term, and adding
as many extraneous $\MathModLam{\MathSame{}}{…}$ to its diff, and running the
repair algorithm in the current global environment and local context.  This
requires a post-processing pass to remove those extraneous lambdas from the
repaired term, before injecting it back into its abstract syntax tree.

\end{itemize}
