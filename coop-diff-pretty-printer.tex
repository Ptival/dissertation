\section{Diff-aware pretty-printing}\label{coop-diff-pretty-printer}

So far, we have explained how we can embed the constructs and binding structure
from the abstract syntax tree of some arbitrary source language, repair those
constructs as part of our core language, and then extract the repaired
constructs back into an abstract syntax tree for the original language.
However, we need to repair the concrete syntax of the original program, not just
the abstract syntax tree.

While we could just call a naive pretty-printer with the final abstract syntax
tree, there is a high chance that doing so would interfere with the layout of
the original code, unless we somehow stored enough information to reproduce the
exact original layout.  In order to minimize our impact on the concrete syntax
of the program, we can instead try to only modify those parts of the concrete
syntax that need being altered.

To do so, we need the abstract syntax tree of the source language to contain
location information about the provenance of its nodes.  This requirement is not
hard to satisfy: we expect most languages to have parsers that produce such
information, if only to be able to indicate error locations.  Now, when we
obtain the repaired source abstract syntax tree, we can simply follow a
simultaneous recursive descent through the original and repaired trees.  At any
point where they differ, we can:

\begin{itemize}

  \item obtain the source span for the text to be replaced from the original
tree,

  \item obtain the replacement text by pretty-printing the node from the
repaired tree.

\end{itemize}

\noindent
Then, we can graft the pretty-printed repaired term in place of the original
text span.  If the language is indentation-sensitive, we might need to be more
clever, figuring out the level of indentation, and hanging the replacement
text at that depth.

We can also come up with extra quality of life improvements, such as retaining
and lifting the comments from the deleted span, so that no comment is lost in
the repair.  Repairs can also be processed sequentially, with a human in the
loop, so that repair suggestions are reviewed by the user before being applied,
for instance, by using a merging tool such as the ones provided for version
control systems.
