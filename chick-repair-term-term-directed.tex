\subsubsection*{Term-directed term repair algorithm}

\newcommand{\RepairTermTwoVar}{
  \inferrule*
  [lab=\UTRVar]
  {
    {\turnstile
      { \dcontext{E}{\delta_E}{\Gamma}{\delta_\Gamma} }
      { \declDiff{v}{\delta_v}{\delta_{\tau_v}} }
    }
    \and
    {\turnstile
      { \dcontext{E}{\delta_E}{\Gamma}{\delta_\Gamma} }
      { \repairArgs{ [] }{\tau_v}{\delta_{\tau_v}}{\delta_v}{\delta}
      }
    }
  }
  {\turnstile
    { \dcontext{E}{\delta_E}{\Gamma}{\delta_\Gamma} }
    { \repairTermWithoutType{v}{\delta} }
  }
}

\newcommand{\RepairTermTwoAppVar}{
  \inferrule*
  [lab=\UTRAppVar]
  {
    {\turnstile
      { \dcontext{E}{\delta_E}{\Gamma}{\delta_\Gamma} }
      { \declDiff{f}{\delta_f}{\delta_{\tau_f}} }
    }
    \and
    {\turnstile
      { \dcontext{E}{\delta_E}{\Gamma}{\delta_\Gamma} }
      { \repairArgs{\range{n}{a_{i}}}{\tau_f}{\delta_{\tau_f}}
        {\delta_f}{\delta}
      }
    }
  }
  {\turnstile
    { \dcontext{E}{\delta_E}{\Gamma}{\delta_\Gamma} }
    { \repairTermWithoutType{ f\ \range{n}{a_{i}} }{\delta} }
  }
}

\newcommand{\RepairTermTwoAppTerm}{
  \inferrule*
  [lab=\UTRAppTerm]
  {
    {\turnstile
      { \dcontext{E}{\delta_E}{\Gamma}{\delta_\Gamma} }
      { \repairTermWithoutType{t}{\delta_{t}}
      }
    }
    \and
    {\turnstile
      { \dcontext{E}{\delta_E}{\Gamma}{\delta_\Gamma} }
      { \repairTermWithoutType{a}{\delta_{a}}
      }
    }
  }
  {\turnstile
    { \dcontext{E}{\delta_E}{\Gamma}{\delta_\Gamma} }
    { \repairTermWithoutType{ t a }{\MathModApp{\delta_{t}}{\delta_{a}}} }
  }
}

\newcommand{\RepairTermTwoPi}{
  \inferrule*
  [lab=\UTRPi]
  {
    {\turnstile
      {\dcontext{E}{\delta_E}{\Gamma}{\delta_\Gamma}}
      {\freshOne{x}{\MathSame}{\delta_x}}
    }
    \and
    {\turnstile
      { \dcontext { e } { \delta_E }
        { \Gamma }
        { \delta_\Gamma }
      }
      { \repair { \tau_1 } { \delta_{\tau_1} } { \MathType } { \MathSame } }
    }
    \\\\\\
    {\turnstile
      { \dcontext { e } { \delta_E }
        { \MathCons{\MathLocalAssum{x}{\tau_1}}{\Gamma} }
        { \MathMod{\MathLocalAssum{\delta_x}{\delta_{\tau_1}}}{\delta_\Gamma} }
      }
      { \repair { \tau_2 } { \delta_{\tau_2} } { \MathType } { \MathSame } }
    }
  }
  {\turnstile
    { \dcontext{E}{\delta_E}{\Gamma}{\delta_\Gamma} }
    { \repairTermWithoutType{ \MathPi{\tau_1}{x}{\tau_2} }
      { \MathModPi{\delta_{\tau_1}}{\delta_x}{\delta_{\tau_2}} }
    }
  }
}

\newcommand{\RepairTermTwoLam}{
  \inferrule*
  [lab=\UTRLam]
  {
    {\turnstile
      {\dcontext{E}{\delta_E}{\Gamma}{\delta_\Gamma}}
      {\freshOne{x}{\MathSame}{\delta_x}}
    }
    \and
    {\turnstile
      { \dcontext { e } { \delta_E }
        { \MathCons{x}{\Gamma} }
        { \MathMod{\delta_{x}}{\delta_{\Gamma}} }
      }
      { \repairTermWithoutType{t}{\delta_{t}} }
    }
  }
  {\turnstile
    { \dcontext{E}{\delta_E}{\Gamma}{\delta_\Gamma} }
    { \repairTermWithoutType{ \MathLam{x}{t} }
      { \MathModLam{\delta_x}{\delta_{t}} }
    }
  }
}

\begin{Rules}
  {fig:repair-term-two-1}
  { Rules for repairing terms, term-directed ($\repairTermThreeOp$) }

  \begin{mathpar}
    \RepairTermTwoVar{}

    \RepairTermTwoAppVar{}

    \RepairTermTwoAppTerm{}

    \RepairTermTwoPi{}

    \RepairTermTwoLam{}
  \end{mathpar}

\end{Rules}


\newcommand{\RepairTermTwoMatch}{
  \inferrule*
  [lab=\UTRMatch]
  {
    {\turnstile
      { \dcontext{E}{\delta_E}{\Gamma}{\delta_\Gamma} }
      { \repairTermWithoutType{t}{\delta_t} }
    }
    \and
    {\turnstile
      { \dcontext{E}{\delta_E}{\Gamma}{\delta_\Gamma} }
      { \repairBranches{t}{[b_1 \ldots b_n]}{\delta_b} }
    }
  }
  {\turnstile
    {\dcontext{E}{\delta_E}{\Gamma}{\delta_\Gamma}}
    {\repairTermWithoutType
      {\MathMatch{t}{b_1 \ldots b_n}}
      {\delta}
    }
  }
}

\newcommand{\RepairTermTwoAnnot}{
  \inferrule*
  [lab=\UTRAnnot]
  {
    {\turnstile
      { \dcontext{E}{\delta_E}{\Gamma}{\delta_\Gamma} }
      { \repair{\tau}{\delta_\tau}{\coqinline{Type}}{\MathSame{}} }
    }
    \and
    {\turnstile
      { \dcontext{E}{\delta_E}{\Gamma}{\delta_\Gamma} }
      { \repair{t}{\delta_t}{\tau}{\delta_\tau} }
    }
  }
  {\turnstile
    { \dcontext{E}{\delta_E}{\Gamma}{\delta_\Gamma} }
    { \repairTermWithoutType{\MathAnnot{t}{\tau}}{\MathAnnot{\delta_t}{\delta_\tau}} }
  }
}

\newcommand{\RepairTermTwoOtherwise}{
  \inferrule*
  [lab=\UTROtherwise]
  {  }
  {\turnstile
    {\dcontext{E}{\delta_E}{\Gamma}{\delta_\Gamma}}
    {\repairTermWithoutType{t}{\MathSame}}
  }
}

\begin{Rules}
  {fig:repair-term-two-2}
  { Rules for repairing terms, term-directed ($\repairTermThreeOp$, part 2/2) }

  \begin{mathpar}
    \RepairTermTwoMatch{}

    \RepairTermTwoAnnot{}

    \RepairTermTwoOtherwise{}
  \end{mathpar}

\end{Rules}


$\repairTermThreeOp$ (described in Figures~\ref{fig:repair-term-two-2}
and~\ref{fig:repair-term-two-2}) is an algorithm directed by the syntax of the
term being repaired, and is used when no information is known about its type, or
how that type might have changed.  Let us again go through the rules one by one.

\paragraph{\rulename{\UTRVar}}

$$\RepairTermTwoVar{}$$

This rule applies when repairing a variable.  By using our lookup rules (from
Section~\ref{chick-lookup}), we can hope to obtain a diff for $v$,
$\out{\delta_{v}}$, telling us whether it has been renamed or not.  Naively, we
might want to call this the answer, and move on.  But, our lookup also lets us
know how the type of $v$ has changed, captured in the output
$\out{\delta_{\tau_{v}}}$.  And it could be that the type of $v$ has changed,
for instance, from a constant to a function of some number of arguments!  Were
this the case, repairing an occurrence of $v$ properly would require adding
values for these new arguments.

This is just a special case of repairing a function application, that we will
see in the next rule (Rule \rulename{\UTRAppVar}).  We will solve the problem by
using a helper function, $\repairArgsOp{}$, which repairs an arbitrary function
applied to some number of arguments.  A variable is simply a special case where
the list of arguments is empty, i.e. $[]$.

\paragraph{\rulename{\UTRAppVar}}

$$\RepairTermTwoAppVar{}$$

This rule applies when the term is a function application.  Repairing nested
function applications is not as trivial as it could seem.  Consider some
function $f$, whose type used to be $(A \rightarrow C)$, and undergoes the
transformation:

\noindent%
%
$\delta_{\tau} = \MathModPi{\MathSame}{\MathSame}{(\MathInsPi{B}{\_}{\MathSame})}$

\noindent%
%
yielding the type $(A \rightarrow B \rightarrow C)$.  Suppose the original code
contains a call to $f$, for instance $(f\ a)$.  The repaired function call ought
to be $(f\ a\ (\MathAnnot{\MathHole}{B}))$.  This means that the repair diff
ought to be:

\noindent%
%
$\delta_{t} = (\MathInsApp{(\MathModApp{\MathSame}{\MathSame})}{(\MathAnnot{\MathHole}{B})})$

\noindent%
%
Notice how the outermost term modification in $\delta_{t}$,
i.e. $\MathInsAppOp{}$, corresponds to the innermost type modification in
$\delta_{\tau}$, i.e.  $\MathInsPiOp{}$.  This is not surprising: a function $g$
whose type is $(A \rightarrow (B \rightarrow (C \rightarrow D)))$, and its
application to arguments $(((g\ a)\ b)\ c)$, exhibit the same inversion.  For
this reason, $\repairArgsOp{}$ works by processing the $\Pi$-telescope type diff
(a sequence of nested modification of $\Pi$s, such as $\delta_{\tau}$) from
outside-in, and builds the term applications diff (such as $\delta_{t}$) from
inside-out.

In order to do so, we syntactically extract as many nested applications as
possible, yielding a sequence of arguments of some arbitrary length
$\range{n}{a_{i}}$, and pass this array of arguments to $\repairArgsOp{}$.

We omit the rules for $\repairArgsOp{}$ as they would be cumbersome.  The
algorithm is first directed by the diff of the function's type, performing the
appropriate addition, modification, deletion, and permutation of arguments,
while also repairing existing arguments.  Care is taken in keeping track of the
unprocessed list of arguments, so that when the function's type diff becomes
$\MathSame{}$, we can introduce the appropriate amount of $\MathMod{}{}$ in the
result.  For instance, consider the following change and repair, and their
diffs:

\begin{figure*}[!htp]
\noindent%
\cprotect\fbox{%
  \begin{minipage}[t]{0.50\textwidth-2\fboxrule-2\fboxsep}%
    \begin{minted}[fontsize=\footnotesize,escapeinside=@@,linenos=false]{coq}
Definition f : A → C → D → E → F := @\dummystrut{ }@
…
.

Definition x := f a c d e.      @\dummystrut{ }@
  \end{minted}
\end{minipage}}%
\cprotect\fbox{%
  \begin{minipage}[t]{0.50\textwidth-2\fboxrule-2\fboxsep}%
  \begin{minted}[fontsize=\footnotesize,escapeinside=@@,linenos=false]{coq}
Definition f : A → @\modified{B →}@ C → D → E → F :=
…
.

Definition x := f a @\repaired{(\_ : B)}@ c d e.
  \end{minted}
\end{minipage}}

\vspace{0.2em}%

  \begin{align*}
    \delta_{\tau_{f}} & = \MathModPi{\MathSame{}_{\coqinline{A}}}{\MathSame{}}{\MathInsPi{A}{B}{\MathSame{}_{\coqinline{C → D → E}}}}\\
    \delta_{x} & = \MathModApp{\MathModApp{\MathModApp{\MathInsApp{\MathSame{}}{\coqinline{(_ : B)}}}{\MathSame{}_{\coqinline{c}}}}{\MathSame{}_{\coqinline{d}}}}{\MathSame{}_{\coqinline{e}}}
  \end{align*}

  \caption{Modification and repair of a function and function call, and their respective diff}
  \label{diff-list-vec-nil}

\end{figure*}

In this figure, we annotate the instances of $\MathSame$ with a subscript
indicating the term that they are preserving.  Notice how, in the type, we have
$\MathSame_{\coqinline{C → D → E}}$, but in the diff for \coqinline{x}, we have
no choice but to expand it to a sequence of $\MathModApp{}{}$ operations, one
for each argument, because of the reversed order.  In practice, this can be
achieved in one of two ways:

\begin{itemize}

  \item by pre-processing instances of $\MathSame{}$ in input types, and
expanding them to the diffs that they stand for.  For instance here, we would
expand $\MathSame{}_{\coqinline{C → D → E}}$ into the equivalent, but more verbose
$\MathModPi{\MathSame{}_{\coqinline{C}}}{\MathSame{}}{\MathModPi{\MathSame{}_{\coqinline{D}}}{\MathSame{}}{\MathSame{}_{\coqinline{E}}}}$,

  \item or by keeping track of the number of arguments in the $\repairArgsOp{}$ algorithm,
and processing $\MathSame{}$ as if it was a $\MathModPi{}{}{}$ for as long as we have
arguments remaining (essentially dynamically emulating the previous option).

\end{itemize}

We present the rules following the latter approach.

\paragraph{\rulename{\UTRAppTerm}}

$$\RepairTermTwoAppTerm{}$$

This rule applies when we are trying to repair a function application that does
not fit the previous rule: its leftmost nested node that is not an application
node is not a variable either, but some other term.  An example of such term
is an immediately applied function obtained by pattern matching:

\begin{minted}[linenos=false]{coq}
(match b with true => f | false => g end) x
\end{minted}

\noindent%
%
Because we do not know the type of the function being applied, we cannot repair
the function applications properly.  We simply repair each argument, and leave
the sequence of applications unchanged.  A more interesting repair would consist
in first inferring the type of $t$, then repairing it to obtain a diff for its
type, and then calling $\repairArgsOp{}$.  We have not yet explored this option.

\paragraph{\rulename{\UTRPi}}

$$\RepairTermTwoPi{}$$

This rule applies to repair a dependent function space.  It is a somewhat
straightforward process.  Once again, we need to possibly refresh the binder,
using $\freshOneOp{}$, for the same reasons as explained for Rule
~\rulename{\RModPi}.

\paragraph{\rulename{\UTRMatch}}

$$\RepairTermTwoMatch{}$$

This rule applies to repair a \coqinline{match} construct. The repair process is
quite complex, so we abstract over it with the $\repairBranchesOp{}$ helper
function.  The discriminee, $t$, can be readily repaired.

Subsequently, we need to repair the branches.  There are two complications:

\begin{enumerate}

  \item We need to find what type is being matched, so that we can look up its
constructors and how they have changed.  However, the type never appears
explicitly in the program.

  \item We would like to keep the existing branches in the same order as they
appear in the user's program, which can be different from the order in which the
constructors are declared in the data type.

\end{enumerate}

To resolve the first issue, we use a sequence of heuristics: we can look up the
type of $t$, which might be the inductive type.  However, remember that in a
dependently-typed language, the type of $t$ could be an arbitrary computation,
rather than simply a concrete type constructor applied to its arguments.  Our
second heuristic looks at the constructors in the branches, and looks for an
inductive data type in scope that has these constructors.  This is more involved,
but should cover cases where the type of $t$ is not helpful.

Once we know the inductive type of $t$, we are ready to attack the second
problem.  Let us consider a concrete case to make matters clear.  Consider the
code:

\begin{minted}{coq}
Inductive T : Type := A : T | B : T | C : T.

Definition code (t : T) : … :=
  match t with
  | C => …
  | A => …
  | B => …
  end.
\end{minted}

\noindent%
%
And let's consider the following modification to inductive data type
\coqinline{T}:

\begin{minted}{coq}
Inductive T : Type := B : T | C : T | D : T.
\end{minted}

\noindent%
%
where constructor \coqinline{A} has been removed, and constructor \coqinline{D}
has been added.  We would like the repair to be:

\begin{minted}{coq}
Definition code (t : T) : … :=
  match t with
  | C => …
  (* | A => … *) (* this branch should be deleted *)
  | B => …
  | D => …       (* this branch should be added *)
  end.
\end{minted}

A simple way to obtain this result is to find a permutation from the old list of
branches to the old list of constructors, and a mapping from the old list of
constructors to the new list of constructors, as depicted in
Figure~\ref{chick-repair-match-permutations}.  We first compute the permutation
$p$ that will let us reorder the patterns as they appear in the user's program,
to the order in which they appear in the data type declaration.  Now that they
are in the same order, and we know the inductive data type being modified, we
can also obtain the diff to that inductive data type, and in particular, the
diff to the list of constructors.  From this, it is easy to compute a
diff-continuation, represented in Figure~\ref{chick-repair-match-permutations}
as a list diff operation with ``$…$''.

\begin{figure}[!htp]

  \centering
  \begin{tikzpicture}

    \tikzstyle{box} = [%
    align=right,
    minimum height=1cm,
    minimum width=2.25cm,
    %draw=black,
    ];

    \node[box]                        (branch-C) {\coqinline{| C => …}};
    \node[box,below=0.25 of branch-C] (branch-A) {\coqinline{| A => …}};
    \node[box,below=0.25 of branch-A] (branch-B) {\coqinline{| B => …}};

    \node[box,right=2.00 of branch-C]          (permuted-branch-A) {\coqinline{| A => …}};
    \node[box,below=0.25 of permuted-branch-A] (permuted-branch-B) {\coqinline{| B => …}};
    \node[box,below=0.25 of permuted-branch-B] (permuted-branch-C) {\coqinline{| C => …}};
    \node[box,below=0.25 of permuted-branch-C] (ghost-D)           {};

    \node[box,right=2.00 of permuted-branch-A] (diff-A) {$\MathDrop{…}$};
    \node[box,below=0.25 of diff-A]            (diff-B) {$\MathMod{\delta_{B}}{…}$};
    \node[box,below=0.25 of diff-B]            (diff-C) {$\MathMod{\delta_{C}}{…}$};
    \node[box,below=0.25 of diff-C]            (diff-D) {$\MathIns{\delta_{D}}{…}$};

    \node[box,right=2.00 of diff-A]          (permuted-diff-C) {$\MathMod{\delta_{C}}{…}$};
    \node[box,below=0.25 of permuted-diff-C] (permuted-diff-A) {$\MathDrop{…}$};
    \node[box,below=0.25 of permuted-diff-A] (permuted-diff-B) {$\MathMod{\delta_{B}}{…}$};
    \node[box,below=0.25 of permuted-diff-B] (permuted-diff-D) {$\MathIns{D}{…}$};

    \draw[->,ultra thick,color=purple] (branch-A.east) -- (permuted-branch-A.west);
    \draw[->,ultra thick,color=purple] (branch-B.east) -- (permuted-branch-B.west);
    \draw[->,ultra thick,color=purple] (branch-C.east) -- (permuted-branch-C.west);

    \node[
    purple,
    fit={(branch-C.east) (permuted-branch-C.west)},
    label=$\op{\MathPermute{p}{}}$,
    ] {};

    \draw[->,ultra thick,color=purple] (permuted-branch-A.east) -- (diff-A.west);
    \draw[->,ultra thick,color=purple] (permuted-branch-B.east) -- (diff-B.west);
    \draw[->,ultra thick,color=purple] (permuted-branch-C.east) -- (diff-C.west);
    \draw[->,ultra thick,color=purple]           (ghost-D.east) -- (diff-D.west);

    \node[
    purple,
    fit={(permuted-branch-A.east) (diff-D.west)},
    label=$\repairBranchOp{}$,
    ] {};

    \draw[->,ultra thick,color=purple] (diff-A.east) -- (permuted-diff-A.west);
    \draw[->,ultra thick,color=purple] (diff-B.east) -- (permuted-diff-B.west);
    \draw[->,ultra thick,color=purple] (diff-C.east) -- (permuted-diff-C.west);
    \draw[->,ultra thick,color=purple] (diff-D.east) -- (permuted-diff-D.west);

    \node[
    RoundedRectangle,
    purple,
    fit={(diff-A.east) (permuted-diff-B.west)},
    label=$\op{\MathPermute{p^{-1}}{}}$,
    ] {};

  \end{tikzpicture}

  \caption{Repairing branches of a  \coqinline{match} using
permutations}~\label{chick-repair-match-permutations}

\end{figure}

For instance, since the \coqinline{A} constructor was removed, we can compute
that the \coqinline{A} branch must be removed.  Had the \coqinline{B}
constructor been modified, we could compute a relevant $\delta_{B}$ patch to
repair the branch.  In our simple example where it has not changed, we have
$\delta_{B} = \MathSame{}$.  For new constructors, like \coqinline{D}, we can
just fabricate empty branches (with the proper constructor and arity for each
pattern) and append them last.

We have not yet extended our language to dependent pattern matching of the form:

\begin{minted}{coq}
match … as … in … return … with
| …
end
\end{minted}

\noindent%
%
as found in \Coq{}, but we believe it should not pose a significant challenge:
the \coqinline{as} clause, \coqinline{in} clause, and \coqinline{return} clause,
should be repaired in order, with proper care taken about what variables come in
and out of scope.  They can be repaired independently from the discriminee and
the branches.  When an \coqinline{in} clause is present, it would in fact tell
us the type of the discriminee!  When a \coqinline{return} clause is present,
however, we might be able to use $\repairTermOneOp{}$, rather than
$\repairTermThreeOp{}$, to repair the body of each branch, since we know what
type to expect from the branch!

Our treatment of \coqinline{match} has a great advantage over manual refactoring
in the presence of a wildcard pattern.  A common source of problems when
refactoring a data type definition happens when some code uses wildcard
patterns.  Consider the following code:

\begin{minted}{coq}
Inductive T : Type := A : T | B : T | C : T.

(* then, somewhere far from this definition *)

Definition handle (t : T) : bool :=
  match t with
  | A => true
  | _ => false (* everything else is false! *)
  end.
\end{minted}

\noindent%
%
If the programmer later decides to extend the datatype to the following:

\begin{minted}[escapeinside=@@]{coq}
Inductive T : Type := A : T | B : T | C : T @\modified{| D : T}@.
\end{minted}

\noindent%
%
then the old version of the \coqinline{handle} function will happily type-check,
mapping input \coqinline{D} to output \coqinline{false}, which might not have
been the intent of the programmer.  Pattern matches that does not use wildcard
patterns, on the other hand, will complain that the \coqinline{D} case is not
handled.  For this reason, wildcard patterns are often considered bad practice.
However, with our technique, the same program will naturally be repaired into:

\begin{minted}[escapeinside=@@]{coq}
Inductive T : Type := A : T | B : T | C : T @\modified{| D : T}@.

Definition handle (t : T) : bool :=
  match t with
  | A => true
  @\repaired{| D => (\_ : bool)}@
  | _ => false (* everything else is false! *)
  end.
\end{minted}

\noindent%
%
which will force the user to consider whether they want this branch to return
\coqinline{true} or \coqinline{false}.  In some sense, it makes working with
wildcard patterns safer, as long as all refactoring attempts are carried through
our tool, of course.

We omit the rules for $\repairBranchesOp{}$, but the high-level algorithm
consists in:

\begin{itemize}

  \item heuristically figure out the type of the discriminee $t$,

  \item perform the transformation described in
Figure~\ref{chick-repair-match-permutations}.

\end{itemize}

\paragraph{\rulename{\UTRAnnot}}

$$\RepairTermTwoAnnot{}$$

This rule applied when the term being repaired is a type-annotated term.  Note
that $(t : \tau)$ \emph{is} the term being repaired here, the black colon
operator being part of the syntax of \Chick{} (described in
Section~\ref{chick-syntax-terms}), as opposed to the purple colon operator that
we use in the previous judgment $\repairTermOneOp{}$.  This is quite
straightforward: we first repair the type, then we repair the term.  Note that,
even though we repair the type with $\repairTermOneOp{}$, there are no rules
that apply to a type like \coqinline{Type}, so it will defer back to
$\repairTermThreeOp{}$.  We only choose to present the rule with
$\repairTermOneOp{}$ so that, if someone extended $\repairTermOneOp{}$ to be
more clever in its treatment of \coqinline{Type}, this could benefit from it.

\paragraph{\rulename{\UTROtherwise}}

$$\RepairTermTwoOtherwise{}$$

For all other cases, we do not yet perform any meaningful repair.  The concrete
terms that we do not attempt to repair are:

\begin{itemize}

  \item immediately-applied term abstractions, which are harder than applied
functions because we cannot simply look up the type of the term abstraction,

  \item holes, that can remain the same,

  \item and universes, that can remain the same.

\end{itemize}
