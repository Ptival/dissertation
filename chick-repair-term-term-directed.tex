\subsubsection*{Term-directed term repair algorithm}

\newcommand{\RepairTermTwoVar}{
  \inferrule*
  [lab=\UTRVar]
  {
    {\turnstile
      { \dcontext{E}{\delta_E}{\Gamma}{\delta_\Gamma} }
      { \declDiff{v}{\delta_v}{\delta_{\tau_v}} }
    }
    \and
    {\turnstile
      { \dcontext{E}{\delta_E}{\Gamma}{\delta_\Gamma} }
      { \repairArgs{ [] }{\tau_v}{\delta_{\tau_v}}{\delta_v}{\delta}
      }
    }
  }
  {\turnstile
    { \dcontext{E}{\delta_E}{\Gamma}{\delta_\Gamma} }
    { \repairTermWithoutType{v}{\delta} }
  }
}

\newcommand{\RepairTermTwoApp}{
  \inferrule*
  [lab=\UTRApp]
  {
    {\turnstile
      { \dcontext{E}{\delta_E}{\Gamma}{\delta_\Gamma} }
      { \declDiff{f}{\delta_f}{\delta_{\tau_f}} }
    }
    \and
    {\turnstile
      { \dcontext{E}{\delta_E}{\Gamma}{\delta_\Gamma} }
      { \repairArgs{\range{n}{a_{i}}}{\tau_f}{\delta_{\tau_f}}
        {\delta_f}{\delta}
      }
    }
  }
  {\turnstile
    { \dcontext{E}{\delta_E}{\Gamma}{\delta_\Gamma} }
    { \repairTermWithoutType{ f\ \range{n}{a_{i}} }{\delta} }
  }
}

\newcommand{\RepairTermTwoPi}{
  \inferrule*
  [lab=\UTRPi]
  {
    {\turnstile
      {\dcontext{E}{\delta_E}{\Gamma}{\delta_\Gamma}}
      {\freshOne{x}{\MathSame}{\delta_x}}
    }
    \and
    {\turnstile
      { \dcontext { e } { \delta_E }
        { \Gamma }
        { \delta_\Gamma }
      }
      { \repair { \tau_1 } { \delta_{\tau_1} } { \MathType } { \MathSame } }
    }
    \\\\\\
    {\turnstile
      { \dcontext { e } { \delta_E }
        { \MathCons{\MathLocalAssum{x}{\tau_1}}{\Gamma} }
        { \MathMod{\MathLocalAssum{\delta_x}{\delta_{\tau_1}}}{\delta_\Gamma} }
      }
      { \repair { \tau_2 } { \delta_{\tau_2} } { \MathType } { \MathSame } }
    }
  }
  {\turnstile
    { \dcontext{E}{\delta_E}{\Gamma}{\delta_\Gamma} }
    { \repairTermWithoutType{ \MathPi{\tau_1}{x}{\tau_2} }
      { \MathModPi{\delta_{\tau_1}}{\delta_x}{\delta_{\tau_2}} }
    }
  }
}

\newcommand{\RepairTermTwoMatch}{
  \inferrule*
  [lab=\UTRMatch]
  {
    {\turnstile
      { \dcontext{E}{\delta_E}{\Gamma}{\delta_\Gamma} }
      { \repairTermWithoutType{t}{\delta_t} }
    }
    \and
    {\turnstile
      { \dcontext{E}{\delta_E}{\Gamma}{\delta_\Gamma} }
      { \repairBranches{t}{[b_1 \ldots b_n]}{\delta_b} }
    }
  }
  {\turnstile
    {\dcontext{E}{\delta_E}{\Gamma}{\delta_\Gamma}}
    {\repairTermWithoutType
      {\MathMatch{t}{b_1 \ldots b_n}}
      {\delta}
    }
  }
}

\newcommand{\RepairTermTwoAnnot}{
  \inferrule*
  [lab=\UTRAnnot]
  {
    {\turnstile
      { \dcontext{E}{\delta_E}{\Gamma}{\delta_\Gamma} }
      { \repair{\tau}{\delta_\tau}{\coqinline{Type}}{\MathSame{}} }
    }
    \and
    {\turnstile
      { \dcontext{E}{\delta_E}{\Gamma}{\delta_\Gamma} }
      { \repair{t}{\delta_t}{\tau}{\delta_\tau} }
    }
  }
  {\turnstile
    { \dcontext{E}{\delta_E}{\Gamma}{\delta_\Gamma} }
    { \repairTermWithoutType{\MathAnnot{t}{\tau}}{\MathAnnot{\delta_t}{\delta_\tau}} }
  }
}

\newcommand{\RepairTermTwoOtherwise}{
  \inferrule*
  [lab=\UTROtherwise]
  {  }
  {\turnstile
    {\dcontext{E}{\delta_E}{\Gamma}{\delta_\Gamma}}
    {\repairTermWithoutType{t}{\MathSame}}
  }
}

\begin{Rules}
  {fig:repair-term-3}
  { Rules for repairing terms, term-directed ($\repairTermThreeOp$) }

\begin{mathpar}
  \RepairTermTwoVar{}

  \RepairTermTwoApp{}

  \RepairTermTwoPi{}

  \RepairTermTwoMatch{}

  \RepairTermTwoAnnot{}

  % This is not necessary, as it is encompassed in the otherwise rule
  % {
  %   \inferrule*
  %   [lab=\UTRType]
  %   { u \in \{ \MathProp, \MathSet, \MathType \} }
  %   {\turnstile
  %     { \dcontext{E}{\delta_E}{\Gamma}{\delta_\Gamma} }
  %     { \repairTermWithoutType{u}{\MathSame} }
  %   }
  % }

  \RepairTermTwoOtherwise{}

  \end{mathpar}

\end{Rules}


$\repairTermThreeOp$ (described in Figure~\ref{fig:repair-term-3}) is an
algorithm directed by the syntax of the term being repaired, and is used when no
information is known about its type, or how that type might have changed.  Let
us again go through the rules one by one.

\paragraph{\rulename{\UTRVar}}

$$\RepairTermTwoVar{}$$

This rule applies when repairing a variable.  By using our lookup rules (from
Section~\ref{chick-lookup}), we can hope to obtain a diff for $v$,
$\out{\delta_{v}}$, telling us whether it has been renamed or not.  Naively, we
might want to call this the answer, and move on.  But, our lookup also lets us
know how the type of $v$ has changed, captured in the output
$\out{\delta_{\tau_{v}}}$.  And it could be that the type of $v$ has changed,
for instance, from a constant to a function of some number of arguments!  Were
this the case, repairing an occurrence of $v$ properly would require adding
values for these new arguments.

This is just a special case of repairing a function application, that we will
see in the next rule (Rule \rulename{\UTRApp}).  We will solve the problem by
using a helper function, $\repairArgsOp{}$, which repairs an arbitrary function
applied to some number of arguments.  A variable is simply a special case where
the list of arguments is empty, i.e. $[]$.

\paragraph{\rulename{\UTRApp}}

$$\RepairTermTwoApp{}$$

This rule applies when the term is a function application.  Repairing nested
function applications is not as trivial as it could seem.  Consider some
function $f$, whose type used to be $(A \rightarrow C)$, and undergoes the
transformation:

\noindent%
%
$\delta_{\tau} = \MathModPi{\MathSame}{\MathSame}{(\MathInsPi{B}{\_}{\MathSame})}$

\noindent%
%
yielding the type $(A \rightarrow B \rightarrow C)$.  Suppose the original code
contains a call to $f$, for instance $(f\ a)$.  The repaired function call ought
to be $(f\ a\ (\MathAnnot{\MathHole}{B}))$.  This means that the repair diff
ought to be:

\noindent%
%
$\delta_{t} = (\MathInsApp{(\MathModApp{\MathSame}{\MathSame})}{(\MathAnnot{\MathHole}{B})})$.

\noindent%
%
Notice how the outermost term modification in $\delta_{t}$,
i.e. $\MathInsAppOp{}$, corresponds to the innermost type modification in
$\delta_{\tau}$, i.e.  $\MathInsPiOp{}$.  This is not surprising: a function $g$
whose type is $(A \rightarrow (B \rightarrow (C \rightarrow D)))$, and its
application to arguments $(((g\ a)\ b)\ c)$, exhibit the same inversion.  For
this reason, $\repairArgsOp{}$ works by processing the $\Pi$-telescope type diff
(a sequence of nested modification of $\Pi$s, such as $\delta_{\tau}$) from
outside-in, and builds the term applications diff (such as $\delta_{t}$) from
inside-out.

In order to do so, we syntactically extract as many nested applications as
possible, yielding a sequence of arguments of some arbitrary length
$\range{n}{a_{i}}$, and pass this array of arguments to $\repairArgsOp{}$.

TODO: decide on this

We omit the rules for $\repairArgsOp{}$ as they would be cumbersome and not
enlightening.

\paragraph{\rulename{\UTRPi}}

$$\RepairTermTwoPi{}$$

This rule applies to repair a dependent function space.  It is a somewhat
straightforward process.  Once again, we need to possibly refresh the binder,
using $\freshOneOp{}$, for the same reasons as explained for Rule
~\rulename{\RModPi}.

\paragraph{\rulename{\UTRMatch}}

$$\RepairTermTwoMatch{}$$

This rule applies to repair a \coqinline{match} construct. The repair process is
fairly simple, but cumbersome, so we abstract over it with the
$\repairBranchesOp{}$ helper function.  The discriminee, $t$, is repaired
independently.

Subsequently, each branch can be repaired independently.  Repairing a branch
requires repairing its pattern, its optional guard, and its body.  Repairing the
pattern is the hard part, as we need to recognize what constructor we are
looking at, and what inductive type this constructor belongs to, so that we can
figure out changes to the type of said constructor, and update the pattern
variables accordingly.  We use some heuristics that let us guess the type being
eliminated without user annotation.  Once the pattern is repaired, we can update
the local context and its diff accordingly, and simply repair the optional guard
and the body, who are just terms, using $\repairTermThreeOp{}$.

\TODO{Mention how it's cool that we add branches automatically, so that we don't
accidentally capture new cases in wildcard patterns!}

We have not yet extended it to dependent pattern matching, of the form:

\noindent%
%
\coqinline{match … as … in … return … with … end}

\noindent%
%
as found in \Coq{}, but we believe it should not pose a significant challenge:
the \coqinline{as} clause, \coqinline{in} clause, and \coqinline{return} clause,
should be repaired in order, with proper care taken about what variables come in
and out of scope.  They can be repaired independently from the discriminee and
the branches.  When a \coqinline{return} clause is present, however, we might be
able to use $\repairTermOneOp{}$, rather than $\repairTermThreeOp{}$, to repair
the body of each branch, since we know what type to expect from the branch!

\TODO{Decide whether to type up rules for $\repairBranchesOp{}$}

\paragraph{\rulename{\UTRAnnot}}

$$\RepairTermTwoAnnot{}$$

WRITEME

\paragraph{\rulename{\UTROtherwise}}

$$\RepairTermTwoOtherwise{}$$

WRITEME








Also of note, in Rule \rulename{\UTRAnnot}, we actually retrieve type
information from the annotation, and can repair the annotated term with more
information.  In practice however, we don't expect much code to contain type
annotations.  Finally, Rule \rulename{\UTROtherwise} captures terms such as
holes, and universes, which we repair by doing nothing.
