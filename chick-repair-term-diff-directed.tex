\subsubsection*{Diff-directed term repair algorithm}

The algorithm for $\repairTermOneOp$ is directed by the syntax of the diff for
the type of the term being repaired.

Rules \rulename{\RSamePi} and \rulename{\RSame} apply when the type has not
been modified.  The first one turns the $\MathSame$ diff for a $\Pi$-type into a
$\MathModPiOp$, to defer the handling of the $\lambda$'s binder, and the
populating of the local context, to the rule \rulename{\RModPi} described
further below.  The second one deals with all other types.  Naively, one might
expect the output to simply be $\MathSame$, however, this would not account for
changes to definitions in the global environment and local context.  The rule
therefore defers this work to the type-directed term repair algorithm
$\repairTermTwoOp$.

The rule that handles the modification of a dependent $\Pi$-type for an explicit
$\lambda$ function, Rule \rulename{\RModPi}, requires a lot of care.  A simple
version of it would not alter the binder $x$ in the term abstraction.
Unfortunately, there is a potential for a binder conflict.  Consider the term
$(\MathLam{a}{b\ a})$ defined in a global environment where $b$ is bound, and
suppose the user renamed that global $b$ into $a$.  The algorithm will
eventually need to replace the occurrence of $b$ in the term abstraction with an
$a$, but in order to do so without capture, we had better $\alpha$-rename the
abstraction's binder with a variable that is fresh in the repaired program.  We
use $\freshOneOp$ again to pick a suitable binder.

A similar issue arises when we want to repair the body of the abstraction: we
want to instantiate the term abstraction's binder $x$ and the type abstraction's
binder $\chi$ with the same variable.  Unfortunately, we can't always choose $x$
because the body of the type abstraction $\tau_2$ might depend on an existing
$x$ in either the global environment or local context, neither can we always
choose $\chi$ because the body of the term abstraction $t$ might depend on such
a $\chi$ too.  The $\freshTwoOp$ helper picks such a suitably fresh variable
$z$.  This $z$ is only necessary for the algorithm, and will never be shown to
the programmer.

Now armed with $\delta_x$ and $z$, we can introduce $(z : \tau_1)$ in the local
context, and have it not need be renamed in the diffed context.  In this
enriched context, we can repair the body of the abstraction, $t$, after having
performed the substitution of $z$ for $x$.  We must perform similar
substitutions in $\tau_2$ and $\delta_{\tau_2}$.  We obtain the diff
$\delta_\tau$, which repairs the body of the abstraction when its binder is
named $z$.  However, in the original program, we will use $\delta_x$ to repair
the binder.  We must therefore substitute back instances of $\MathReplace{x}$
where we find free instances of $\MathReplace{x}$ in $\delta_\tau$.  The same
considerations must be taken into account when a $\Pi$ is inserted
(\rulename{\RInsPi}).

To deal properly with $\Pi$-permutations (\rulename{\RPermutePis}), we must be
able to syntactically extract as many $\lambda$s as the length of the
permutation.  When we can do so, the repair proceeds recursively after having
permuted both the term and the type abstractions.  When the original type $\tau$
gets replaced by an arbitrary type $\tau'$ (\rulename{\RReplace}), we have no
better option than inserting a typed hole.  The final rule
(\rulename{\RepairTermPrefix-Otherwise-\GenericRepairPrefix}) triggers if none
of the previous rules have, and dispatches the repair to the type-directed term
repair algorithm $\repairTermTwoOp$.

\begin{Rules}{fig:repair-term}{ Rules for repairing terms, diff-directed ($\repairTermOneOp$) }

\begin{mathpar}
  {
    \inferrule*
    [lab=\RSamePi]
    {
      {\turnstile
        { \dcontext{E}{\delta_E}{\Gamma}{\delta_\Gamma} }
        { \repair
          {\MathLam{x}{t}}
          {\delta}
          {\MathPi{\tau_1}{\chi}{\tau_2}}
          {\MathModPi{\MathSame}{\MathSame}{\MathSame}}
        }
      }
    }
    {\turnstile
      { \dcontext{E}{\delta_E}{\Gamma}{\delta_\Gamma} }
      { \repair
        {\MathLam{x}{t}}
        {\delta}
        {\MathPi{\tau_1}{\chi}{\tau_2}}
        {\MathSame}
      }
    }
  }

  {
    \inferrule*
    [lab=\RSame]
    {\turnstile
      { \dcontext{E}{\delta_E}{\Gamma}{\delta_\Gamma} }
      { \repairTermWithoutType{t}{\delta_{t}}}
    }
    {\turnstile
      { \dcontext{E}{\delta_E}{\Gamma}{\delta_\Gamma} }
      { \repair{t}{\delta_{t}}{\tau}{\MathSame} }
    }
  }

  \newcommand{\RepairTermDiffModPi}{
  \inferrule*
  [lab=\RModPi]
  {
    {\turnstile%
      {\dcontext{E}{\delta_E}{\Gamma}{\delta_\Gamma}}
      {\freshOne{x}{\MathSame}{\delta_x}}
    }
    \and
    {\turnstile%
      {\dcontext{E}{\delta_E}{\Gamma}{\delta_\Gamma}}
      {\freshTwo{\blackdiff{x}{\delta_x}}{\blackdiff{\chi}{\delta_\chi}}{z}}
    }
    \\\\\\
    {\MathPatches{x}{\delta_x}{x'}}
    \and
    {\MathPatches{\chi}{\delta_\chi}{\chi'}}
    \\\\\\
    {\turnstile%
      { \dcontext{E}{\delta_E}
        { \MathCons{\MathLocalAssum{z}{\tau_1}}{\Gamma} }
        { \MathMod{\MathLocalAssum{\MathSame}{\delta_{\tau_1}}}{\delta_\Gamma} }
      }
      { \repair%
        {t[\subst{x}{z}]}
        {\delta_{t}}
        {\tau_2[\subst{\chi}{z}]}
        {\delta_{\tau_2}[\subst{\MathReplace{\chi'}}{\MathReplace{z}}]}
      }
    }
  }
  {\turnstile%
    { \dcontext{E}{\delta_E}{\Gamma}{\delta_\Gamma} }
    {\repair%
      { \MathLam{x}{t} }
      { \MathModLam{\delta_x}{\delta_t[\subst{\MathReplace{z}}{\MathReplace{x'}}]} }
      { \MathPi{\tau_1}{\chi}{\tau_2} }
      { \MathModPi{\delta_{\tau_1}}{\delta_\chi}{\delta_{\tau_2}} }
    }
  }
}

  \RepairTermDiffModPi{}

  {
    \inferrule*
    [lab=\RInsPi]
    {
      {\turnstile
        {\dcontext{E}{\delta_E}{\Gamma}{\delta_\Gamma}}
        {\freshOne{\chi}{\MathSame}{\delta_\chi}}
      }
      \and
      {\MathPatches{\chi}{\delta_\chi}{x}}
      \and
      {\MathPatches{\tau}{\delta_{\tau_1}}{\tau_1}}
      \\\\\\
      \turnstile
      {\dcontext{E}{\delta_E}{\Gamma}
        {\MathIns{\MathLocalAssum{x}{\tau_1}}{\delta_\Gamma}}
      }
      { \repair{t}{\delta_{t}}{\tau}
        {\delta_{\tau_2}[\subst{\MathReplace{\chi}}{\MathReplace{x}}]}
      }
    }
    {
      \turnstile
      { \dcontext{E}{\delta_E}{\Gamma}{\delta_\Gamma} }
      { \repair
        {t}
        { \MathInsLam{x}{\delta_{t}} }
        {\tau}
        { \MathInsPi{\delta_{\tau_1}}{\chi}{\delta_{\tau_2}} }
      }
    }
  }

  {
    \inferrule*
    [lab=\RDropPi]
    {
      {\turnstile
        {\dcontext{E}{\delta_E}{\Gamma}{\delta_\Gamma}}
        {\freshTwo{\blackdiff{x}{\MathSame}}{\blackdiff{\chi}{\MathSame}}{z}}
      }
      \\
      {\turnstile
        {\dcontext
          {E}{\delta_E}
          {\MathCons{\MathLocalAssum{z}{\tau_1}}{\Gamma}}
          {\MathDrop{\delta_\Gamma}}
        }
        {\repair
          {t[\subst{x}{z}]}
          {\delta_{t}}
          {\tau[\subst{\chi}{z}]}
          {\delta_{\tau_2}}
        }
      }
    }
    {
      \turnstile
      { \dcontext{E}{\delta_E}{\Gamma}{\delta_\Gamma} }
      { \repair
        {\MathLam{x}{t}}
        {\MathDropLam{\delta_{t}}}
        {\MathPi{\tau_1}{\chi}{\tau_2}}
        {\MathDropPi{\delta_{\tau_2}}}
      }
    }
  }

  {
    \inferrule*
    [lab=\RPermutePis]
    {\turnstile
      { \dcontext{E}{\delta_E}{\Gamma}{\delta_\Gamma} }
      {\repair
        { \MathLam{x_{p(1)} \ldots x_{p(|p|)}}{t} }
        { \delta }
        { \mkMathPiRaw{(\chi_{p(1)} : \tau_{p(1)}) \ldots (\chi_{p(|p|)} : \tau_{p(|p|)})}{\tau}{}{} }
        { \dtau{} }
      }
    }
    {\turnstile
      { \dcontext{E}{\delta_E}{\Gamma}{\delta_\Gamma} }
      {\repair
        { \MathLam{x_1 \ldots x_{|p|}}{t} }
        { \MathPermuteLams{p}{\delta} }
        { \mkMathPiRaw{(\chi_{1} : \tau_i) \ldots (\chi_{|p|} : \tau_{|p|})}{\tau}{} }
        { \MathPermutePis{p}{\dtau{}} }
      }
    }
  }

  {
    \inferrule*
    [lab=\RReplace]
    { }
    {\turnstile
      { \dcontext{E}{\delta_E}{\Gamma}{\delta_\Gamma} }
      { \repair
        {t}
        {\MathReplace{\MathAnnot{\MathHole}{\tau'}}}
        {\tau}
        {\MathReplace{\tau'}}
      }
    }
  }

  {
    \inferrule*
    [lab=\RepairTermPrefix-Otherwise-\GenericRepairPrefix]
    {\turnstile
      { \dcontext{E}{\delta_E}{\Gamma}{\delta_\Gamma} }
      { \genericrepair
        { t }
        { \delta_t }
        { \tau }
      }
    }
    {\turnstile
      { \dcontext{E}{\delta_E}{\Gamma}{\delta_\Gamma} }
      {\repair
        { t }
        { \delta_t }
        { \tau }
        { \delta_\tau }
      }
    }
  }

\end{mathpar}

\end{Rules}
