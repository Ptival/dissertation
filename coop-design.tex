\section{Design of \Coop{}}\label{coop-design}

Unfortunately, testing \Chick{} on real-world \Coq{} programs is extremely
complicated.  For one, the language described by \Chick{} is a very small subset
of Coq's \Gallina{} and \Vernacular{} languages, ignoring not only many of the
core constructs of those languages, but also all of \Ltac{}.  Even ignoring this
lack of support, parsing \Coq{} code correctly is almost always not possible
outside of \Coq{} itself, because the language supports an extensible syntax,
adding almost arbitrary notations to its core syntax, and does not (yet) expose
this information in a convenient way to external tools.

Even assuming the problem of parsing solved, \Chick{} as described in
Section~\ref{chick-syntax} has no way of representing all the features of \Coq{}
that we do not explicitly support.  In order to build a robust tool, that can
withstand future evolution of those languages, we need a way to account for
language constructs that we do not know.  What we would like is the ability to
gloss over syntactic constructs that we do not know, and provide repairs for the
rest of the program, to the best of our ability.  This is unsafe by nature,
since the syntactic constructs we gloss over can change the global environment
in ways we do not account for, but any help attempt is better than none, since
our tool only claims to cut some of the tedious work for its users.

We present the workflow that we envision for \Coop{}, an extension of \Chick{}
to support repairing other programming languages, in
Figure~\ref{fig:coop-workflow}.  Starting with an original program and a
partially refactored program, in some source language, we can use a parser for
the source language to obtain abstract syntax trees (\define{AST}) for those.
Now, in order to embed these into \Chick{}, we need some program we call an
\define{embedder}, and discuss in Section~\ref{coop-embedding-extracting}.  Now
we can run the original \Chick{} pipeline, described in
Section~\ref{chick-design}, until we obtained a repaired abstract syntax tree,
still in the \Chick{} language.  We can now run an \define{extracter}, which
performs the inverse operation to our embedder.  We obtain a repaired abstract
syntax tree, now in the syntax of the source language.  We can then obtained the
repaired program, using the surface syntax of the source language, by calling a
pretty-printer.  We use a diff-pretty-printer, described in
Section~\ref{coop-diff-pretty-printer}, so as to obtain a program syntactically
as close as possible to the original.

\begin{figure}[!htp]

  \centering
  \begin{tikzpicture}[node distance=0.8cm]

    \tikzstyle{box} = [%
    align=center,
    draw=black,
    font=\bfseries,
    inner sep=2ex,
    line width=3pt,
    rectangle,
    %ultra thick,
    ];

    \tikzstyle{doc}=[%
    align=center,
    draw=black,
    shape=document,
    inner sep=2ex,
    shape=document,
    thick,
    ]

    %\draw [dotted,draw=black] (0,0) grid (15,22) rectangle (0,0);
    \node at (0,0)  (SW) { };
    \node at (15,0) (SE) { };

    \node[doc] at (4,21) (original-1) {Original program};
    \node[doc,right=of original-1] (modified-1) {Partially refactored program};
    \path (original-1) -- node (mid-1) {} (modified-1);

    \node[box,below=1.25cm of mid-1,minimum width=11cm] (parser) { Source language parser };

    \node[doc,below=3cm of original-1] (original-2) {Original AST};
    \node[doc,below=3cm of modified-1] (modified-2)
    {Partially refactored AST};

    \node[box,below=3cm of parser,minimum width=11cm] (embedder) { Source language embedder };

    \draw[dashed, ultra thick] (embedder.west) -- (embedder.west -| SW);
    \draw[dashed, ultra thick] (embedder.east) -- (embedder.east -| SE)
    node[above left,font=\bfseries]{\makecell[r]{Source\\language}}
    node[below left,font=\bfseries]{Chick}
    ;

    \node[doc,below=3cm of original-2] (original-3) {Original AST};
    \node[doc,below=3cm of modified-2] (modified-3)
    {Partially refactored AST};

    \node[box,below=3cm of embedder,minimum width=11cm] (chick)
    { Chick (as described in Figure~\ref{chick-workflow}) };

    \node[doc,below=of chick]      (repaired-1) { Repaired AST };
    \node[box,below=of repaired-1,minimum width=11cm] (extracter)
    { Source language extracter };

    \draw[dashed, ultra thick] (extracter.west) -- (extracter.west -| SW);
    \draw[dashed, ultra thick] (extracter.east) -- (extracter.east -| SE)
    node[above left,font=\bfseries]{Chick}
    node[below left,font=\bfseries]{\makecell[r]{Source\\language}}
    ;

    \node[doc,below=of extracter]  (repaired-2) { Repaired AST };
    \node[box,below=of repaired-2,minimum width=11cm] (printer)
    { Source language diff-pretty-printer };
    \node[doc,below=of printer]    (repaired-3) { Repaired program };

    \draw[->,ultra thick] (original-1.south) -- (original-1.south |- parser.north);
    \draw[->,ultra thick] (modified-1.south) -- (modified-1.south |- parser.north);

    \draw[->,ultra thick] (parser.south -| original-1.south) -- (original-2.north);
    \draw[->,ultra thick] (parser.south -| modified-1.south) -- (modified-2.north);

    \draw[->,ultra thick] (original-2.south) -- (original-2.south |- embedder.north);
    \draw[->,ultra thick] (modified-2.south) -- (modified-2.south |- embedder.north);

    \draw[->,ultra thick] (embedder.south -| original-2.south) -- (original-3.north);
    \draw[->,ultra thick] (embedder.south -| modified-2.south) -- (modified-3.north);

    \draw[->,ultra thick] (original-3.south) -- (original-3 |- chick.north);
    \draw[->,ultra thick] (modified-3.south) -- (modified-3 |- chick.north);

    \draw[->,ultra thick] (chick.south)      -- (repaired-1.north);
    \draw[->,ultra thick] (repaired-1.south) -- (extracter.north);
    \draw[->,ultra thick] (extracter.south)  -- (repaired-2.north);
    \draw[->,ultra thick] (repaired-2.south) -- (printer.north);
    \draw[->,ultra thick] (printer.south)    -- (repaired-3.north);

    \draw[->,rounded corners,ultra thick]
    (modified-2.east)
    -| ($(embedder.east)+(1,0)$)
    |- (printer.east);

    % \draw[->,rounded corners,ultra thick]
    % (original.east)
    % -- (guess.south |- original.east)
    % -- (guess.south)
    % ;

    % \draw[->,ultra thick] (modified.east) -- (guess.west);
    % \draw[->,ultra thick] (guess.east)    -- (guessed.west);

    % \draw[->,rounded corners,ultra thick]
    % (original.east)
    % -- (repair.south |- original.east)
    % -- (repair.south)
    % ;

    % \draw[->,ultra thick] (guessed.east) -- (repair.west);
    % \draw[->,ultra thick] (repair.east)  -- (repaired.west);

    % \draw[->,ultra thick] (original.east)  -- (patch.west);
    % \draw[->,ultra thick] (repaired.south) -- (patch.north);
    % \draw[->,ultra thick] (patch.south)    -- (patched.north);

  \end{tikzpicture}

  \caption{\Coop{}'s workflow}~\label{fig:coop-workflow}

\end{figure}
