\newcommand{\coqinlineDEBUG}[1]{#1}

%%%%% General-purpose text macros %%%%%
\newcommand*\circled[1]{\tikz[baseline=(C.base)]\node[draw,circle,inner sep=1.2pt,line width=0.2mm,](C) {\small #1};\!}

\definecolor{colorblind-green}{HTML}{AEBD25}
\definecolor{colorblind-red}{HTML}{830024}

\newcommand*\OK{\textcolor{colorblind-green}{\ding{51}}}
\newcommand*\KO{\textcolor{colorblind-red}{\ding{55}}}
\newcommand*\YN{$\sim$} % mnemonic: Yes/No

\newcommand{\participant}{\begin{tikz}\node[graduate,monitor,minimum size=0.1cm] {};\end{tikz}}
\newcommand{\ptcpone}{  \participant{} $\times 1$}
\newcommand{\ptcptwo}{  \participant{} $\times 2$}
\newcommand{\ptcpthree}{\participant{} $\times 3$}
\newcommand{\ptcpfour}{ \participant{} $\times 4$}
\newcommand{\ptcpfive}{ \participant{} $\times 5$}

\newcommand{\define}[1]{\emph{#1}}
\newcommand{\mycite}[1]{\citeauthor{#1}~\cite{#1}}

\newcommand{\Language}[1]{\emph{#1}}
\newcommand{\Tool}[1]{\emph{#1}}
\newcommand{\Gallina}{\Language{Gallina}}
\newcommand{\Ltac}{\Language{Ltac}}
\newcommand{\Vernacular}{\Language{Vernacular}}

\newcommand{\Agda}{\Language{Agda}}
\newcommand{\Chick}{\Language{Chick}}
\newcommand{\Coop}{\Language{Coop}}
\newcommand{\Coq}{\Language{Coq}}
\newcommand{\CoqIDE}{\Tool{CoqIDE}}
\newcommand{\Erlang}{\Language{Erlang}}
\newcommand{\Haskell}{\Language{Haskell}}
\newcommand{\Idris}{\Language{Idris}}
\newcommand{\JavaScript}{\Language{JavaScript}}
\newcommand{\OCaml}{\Language{OCaml}}
\newcommand{\PeaCoq}{\Language{PeaCoq}}
\newcommand{\RxJS}{\Language{RxJS}}
\newcommand{\SerAPI}{\Tool{SerAPI}}
\newcommand{\Snap}{\Language{Snap}}
\newcommand{\TypeScript}{\Language{TypeScript}}

\newcommand{\OperatorColor}{purple}
\newcommand{\Operator}[1]{\textcolor{\OperatorColor}{\ #1\ }}
\newcommand{\Entails}{\Operator{\vdash}}
\newcommand{\HasType}{\Operator{:}}

%\newmintinline{coq}{fontsize=\small}
% \newcommand{\coqinline}[1]{%
%   %\colorbox{monokaibg}{%
%   \parbox[c][0.9em]{\widthof{\mycoq{#1}}}{\mycoq{#1}}%
%   %}%
% }

% need this because minted will change height of text... huh
\newcommand{\dummystrut}[1]{\colorbox{white}{\strut #1}}
\newcommand{\modified}[1]{\colorbox{orange!30}{\strut #1}}
\newcommand{\repaired}[1]{\colorbox{teal!30}{\strut #1}}

\newcommand{\rulename}[1]{$\LeftTirNameStyle{#1}$}

\newcommand{\RmApp}{Rm-App}
\newcommand{\RmPi}{Rm-Pi}
\newcommand{\InsApp}{Ins-App}

%%%%% Math-mode macros %%%%%

\newcommand{\opcolor}{purple}
\newcommand{\out}[1]{ \boxed{ \textcolor{teal}{#1} } }
\newcommand{\MathPatches}[3]{%
#1 \overset{#2}{\mathbin{\textcolor{\opcolor}{\rightsquigarrow}}} \out{#3}%
}

\newcommand{\App}{\$}
\newcommand{\Mod}{Mod}
\newcommand{\Drop}{Drop}
\newcommand{\Ins}{Ins}
\newcommand{\Lam}{\lambda}
\newcommand{\Permute}{Permute}

\newcommand{\oMod}{\overset{\mathtt{\Mod}}}
\newcommand{\oDrop}{\overset{\mathtt{\Drop}}}
\newcommand{\oIns}{\overset{\mathtt{\Ins}}}

\newcommand{\Cons}{::}
\newcommand{\permute}[2]{%
\overline{#2}^{\overset{#1}{\rightleftarrows}}%
}
\newcommand{\permuteOp}[2]{%
\overset{\overset{#1}{\rightleftarrows}}{#2}%
}

\newcommand{\mkMathPiRaw}[4]{#3{\Pi} #4{#1} \rightarrow #2}
\newcommand{\mkMathPi}[5]{\mkMathPiRaw{(#2 : #1)}{#3}{#4}{#5}}


\newcommand{\MathCons}[2]{#1 \Cons #2}
\newcommand{\MathDrop}[1]{\oDrop{\Cons} #1}
\newcommand{\MathDropApp}[1]{\oDrop{\App} #1}
\newcommand{\MathDropLam}[1]{\oDrop{\Lam} #1}
\newcommand{\MathDropPi}[1]{\MathDropPiOp #1}
\newcommand{\MathDropPiOp}{\oDrop{\Pi}}
\newcommand{\MathHole}{\texttt{\_}}
\newcommand{\MathIns}[2]{#1 \oIns{\Cons} #2}
\newcommand{\MathInsApp}[2]{\mkMathApp{#1}{#2}{\oIns}}
\newcommand{\MathInsAppOp}{\oIns{\App}}
\newcommand{\MathInsLam}[2]{\mkMathLam{#1}{#2}{\oIns}{}}
\newcommand{\MathInsLamOp}{\oIns{\Lam}}
\newcommand{\MathInsPi}[3]{\mkMathPi {#1}{#2}{#3}{\oIns}{}}
\newcommand{\MathInsPiOp}{\oIns{\Pi}}
\newcommand{\MathLam}[2]{\mkMathLam{#1}{#2}{}{}}
\newcommand{\MathLams}[3]{\mkMathLam{#1}{#2}{}{\BarCount{#3}}}
\newcommand{\MathMod}[2]{#1 \oMod{\Cons} #2}
\newcommand{\MathModAppOp}{\oMod{\App}}
\newcommand{\MathModLamOp}{\oMod{\Lam}}
\newcommand{\MathModPiOp}{\oMod{\Pi}}
\newcommand{\MathPermute}[2]{\permuteOp{#1}{\Cons} #2}
\newcommand{\MathPermuteLams}[2]{\permuteOp{#1}{\lambda} #2}
\newcommand{\MathPermutePis}[2]{\permuteOp{#1}{\Pi} #2}
\newcommand{\MathPi}[3]{\mkMathPi{#1}{#2}{#3}{}{}}
\newcommand{\MathPis}[4]{\mkMathPi{#1}{#2}{#3}{}{\BarCount{#4}}}
\newcommand{\MathReplaceOp}{\mathds{K}}
\newcommand{\MathReplace}[1]{\MathReplaceOp{}({#1})}
\newcommand{\MathSame}{\mathds{1}}

\newcommand{\mkMathApp}[3]{#1 #3{ \$ } #2}

%%% REPAIR %%%

\newcommand{\op}[1]{\textcolor{\opcolor}{#1}}

\newcommand{\boundOp}{\op{\text{Bound}}}
\newcommand{\diffExpandOp}{\op{\Delta_{\text{Expand}}}}
\newcommand{\freshOneOp}{\op{\text{Fresh}_1}}
\newcommand{\freshTwoOp}{\op{\text{Fresh}_2}}
\newcommand{\repairIndOp}{\op{R_{\text{Inductive}}}}
\newcommand{\repairProgOp}{\op{R_{\text{Program}}}}
\newcommand{\repairTermOneOp}{\op{R_{\text{Term}_1}}}
\newcommand{\repairTermTwoOp}{\op{R_{\text{Term}_2}}}
\newcommand{\repairTermThreeOp}{\op{R_{\text{Term}_2}}}
\newcommand{\repairVernacOp}{\op{R_{\text{Vernacular}}}}
\newcommand{\vernacEnvOp}{\op{E_{V}}}
\newcommand{\repairBranchOp}{\op{R_{\text{Branch}}}}
\newcommand{\repairBranchesOp}{\op{R_{\text{Branches}}}}

\newcommand{\diffExpand}[3]{
  \op{\diffExpandOp{}(} \diff{#1}{#2} \op{) = } \out{#3}
}

\newcommand{\repairParametersOp}{\op{R_{\text{Parameters}}}}
\newcommand{\repairIndicesOp}{\op{R_{\text{Indices}}}}
\newcommand{\repairConstructorsOp}{\op{R_{\text{Constructors}}}}

\newcommand{\repairParameters}[3]{
  \mksimplerepair{\repairParametersOp}{\diff{#1}{#2}}{#3}
}
\newcommand{\repairIndices}[3]{
  \mksimplerepair{\repairIndicesOp}{\diff{#1}{#2}}{#3}
}

\newcommand{\repairConstructors}[3]{
  \mksimplerepair{\repairConstructorsOp}{\diff{#1}{#2}}{#3}
}

\newcommand{\blackbrackets}[1]{\left[ #1 \right]}
\newcommand{\blackdiff}[2]{\blackbrackets{\genfrac{}{}{0pt}{}{#1}{#2}}}
\newcommand{\bound}[1]{\boundOp\op{(}#1\op{)}}
\newcommand{\brackets}[1]{%
  \color{\opcolor} \left[ \normalcolor #1 \color{\opcolor} \right] \normalcolor%
}
\newcommand{\context}[2]{#1 \op{,} #2}
\newcommand{\dcontext}[4]{\context{\diff{#1}{#2}}{\diff{#3}{#4}}}
\newcommand{\denv}[2]{\diff{#1}{#2}}
\newcommand{\diff}[2]{\brackets{\genfrac{}{}{0pt}{}{#1}{#2}}}
\newcommand{\dtau}[1]{\delta_{\tau_{#1}}}
\newcommand{\equals}[2]{#1 \mathbin{\textcolor{\opcolor}{=}} #2}
\newcommand{\freshOne}[3]{\freshOneOp\op{(}\diff{#1}{#2}\op{) =}\ \out{#3}}
\newcommand{\freshTwo}[3]{\freshTwoOp\op{(}#1\op{,}#2\op{) =}\ \out{#3}}
\newcommand{\genericrepair}[3]{\mkrepair{\repairTermTwoOp}{#1}{#2}{#3}{\op{?}}}
\newcommand{\nturnstile}[2]{#1 \mathbin{\textcolor{\opcolor}{\nvdash}} #2}
\newcommand{\qmark}{\textcolor{\opcolor}{?}}
\newcommand{\squiggly}[2]{#1 \mathbin{\textcolor{\opcolor}{\rightsquigarrow}} #2}
\newcommand{\subst}[2]{#1 \leftarrow #2}
\newcommand{\turnstile}[2]{#1\ \mathbin{\textcolor{\opcolor}{\vdash}}\ #2}
\newcommand{\repair}[4]{ \mkrepair{\repairTermOneOp}{#1}{#2}{#3}{#4} }
\newcommand{\repairBranches}[3]{\mksimplerepair{\repairBranchesOp}{#1\op{,}\ #2}{#3}}

\newcommand{\repairInd}[6]{
  \mksimplerepair%
  {\repairIndOp}
  {#1\op{,}\ #2\op{,}\ #3\op{,}\ #4\op{,}\ #5}
  %{\diff{#1}{#2}\op{,}\ \diff{#3}{#4}\op{,}\ \diff{#5}{#6}\op{,}\ \diff{#7}{#8}}
  {#6}
}

\newcommand{\repairProg}[3]{
  \repairProgOp\op{(}\blackdiff{#1}{#2}\op{)}\ \op{=}\ \out{#3}
}
\newcommand{\repairTerm}[4]{\mkrepair{\repairTermOneOp}{#1}{#2}{#3}{#4}}
\newcommand{\repairTermWithoutType}[2]{\mksimplerepair{\repairTermThreeOp}{#1}{#2}}
\newcommand{\repairVernac}[3]{
  \repairVernacOp\op{(}\blackdiff{#1}{#2}\op{)}\ \op{=}\ \out{#3}
}
\newcommand{\repairVernacTwoLines}[2]{
  \repairVernacOp\op{(}\blackdiff{#1}{#2}\op{)}\ \op{=}
}

\newcommand{\DefinitionText}{\text{Definition}}
\newcommand{\InductiveText}{\text{Inductive}}
\newcommand{\ConstructorText}{\text{Constructor}}
% See below
\newcommand{\GlobalDefinitionText}{\text{Definition}}
\newcommand{\GlobalInductiveText}{\text{Inductive}}
% I don't remember if this difference was very important, it just seems confusing
% \newcommand{\GlobalDefinitionText}{\text{Global\-Definition}}
% \newcommand{\GlobalInductiveText}{\text{Global\-Inductive}}

\newcommand{\Definition}[4]{ \DefinitionText(\{ #1, #2, #3, #4 \}) }
\newcommand{\Inductive}[5]{ \InductiveText(\{ #1, #2, #3, #4, #5 \}) }
\newcommand{\Constructor}[3]{ \ConstructorText(\{ #1, #2, #3 \}) }
\newcommand{\GlobalDefinition}[3]{ \GlobalDefinitionText(\{ #1, #2, #3 \}) }
\newcommand{\GlobalInductive}[5]{ \GlobalInductiveText(\{ #1, #2, #3, #4, #5 \}) }

\newcommand{\ModifyDefinition}[4]{ \delta_{\DefinitionText}(\{ #1, #2, #3, #4 \}) }
\newcommand{\ModifyInductive}[5]{ \delta_{\InductiveText}(\{ #1, #2, #3, #4, #5 \}) }
\newcommand{\ModifyGlobalDefinition}[3]{ \delta_{\GlobalDefinitionText}(\{ #1, #2, #3 \}) }
\newcommand{\ModifyGlobalInductive}[5]{ \delta_{\GlobalInductiveText}(\{ #1, #2, #3, #4, #5 \}) }

\newcommand{\ModifyConstructor}[3]{ \delta_{\ConstructorText}(\{ #1, #2, #3 \}) }

\newcommand{\GlobalDefinitionAnon}{\GlobalDefinitionText(\{ \ldots \})}
\newcommand{\GlobalInductiveAnon}{\GlobalInductiveText(\{ \ldots \})}

\newcommand{\DefinitionAnon}{\DefinitionText(\{ \ldots \})}
\newcommand{\InductiveAnon}{\InductiveText(\{ \ldots \})}

\newcommand{\RepairProg}[1]{Repair-Program-#1}
\newcommand{\RepairVernac}[1]{Repair-Vernacular-#1}
\newcommand{\RepairParameters}[1]{Repair-Parameters-#1}
\newcommand{\RepairIndices}[1]{Repair-Indices-#1}

\newcommand{\RProgDrop}{\RepairProg{Drop}}
\newcommand{\RProgIns}{\RepairProg{Insert}}
\newcommand{\RProgMod}{\RepairProg{Modify}}
\newcommand{\RProgPermute}{\RepairProg{Permute}}
\newcommand{\RProgReplace}{\RepairProg{Replace}}
\newcommand{\RProgSameCons}{\RepairProg{Same-Cons}}

\newcommand{\scalefactor}{1}
\newcommand{\invscalefactor}{1}

\NewEnviron
    {Rules}[2]
    {
      \begin{figure*}
        %\scalebox{\invscalefactor}{
          %\begin{minipage}{\scalefactor\textwidth}
            %\begin{mathpar}
              \BODY
            %\end{mathpar}
          %\end{minipage}
        %}
        \caption{#2}
        \label{#1}
      \end{figure*}
    }

\newcommand{\MathProp}{\mathtt{Prop}}
\newcommand{\MathSet} {\mathtt{Set}}
\newcommand{\MathType}{\mathtt{Type}}

\newcommand{\hasType}[2]{ #1\ \op{:}\ #2 }

\newcommand{\mkrepair}[5]{
  #1
  \op{(}
  \hasType
  { #2 }
  { \diff{ #4 }{ #5 } }
  \op{) =\ }
  \out{ #3 }
}

\newcommand{\MathLocalAssum}[2]{ ( #1 : #2 ) }
\newcommand{\MathLocalDef}[3]{ ( #1 : #2 \coloneqq #3 ) }

\newcommand{\mksimplerepair}[3]{
  #1
  \op{(}
  #2
  \op{) =\ }
  \out{ #3 }
}

\newcommand{\RModPi}{\RepairTerm{Mod-$\Pi$}}
\newcommand{\RInsPi}{\RepairTerm{Ins-$\Pi$}}
\newcommand{\RDropPi}{\RepairTerm{Drop-$\Pi$}}
\newcommand{\RPermutePis}{\RepairTerm{Permute-$\Pi$s}}
\newcommand{\RReplace}{\RepairTerm{Replace}}
\newcommand{\RSame}{\RepairTerm{Same-Other}}
\newcommand{\RSamePi}{\RepairTerm{Same-$\Pi$}}

\newcommand{\RepairTermPrefix}{Repair-Term-1}
\newcommand{\GenericRepairPrefix}{Repair-Term-2}
\newcommand{\UnknownTypeRepairPrefix}{Repair-Term-2}

\newcommand{\RepairTerm}[1]{\RepairTermPrefix-#1}
\newcommand{\GenericRepair}[1]{\GenericRepairPrefix-#1}
\newcommand{\UnknownTypeRepair}[1]{\UnknownTypeRepairPrefix-#1}

\newcommand{\mkMathLam}[4]{#3{\lambda} #4{#1} \rightarrow #2}

\newcommand{\MathModApp}[2]{ \mkMathApp{#1}{#2}{\oMod}{} }
\newcommand{\MathModLam}[2]{ \mkMathLam{#1}{#2}{\oMod}{} }
\newcommand{\MathModPi} [3]{ \mkMathPi{#1}{#2}{#3}{\oMod}{} }

\newcommand{\MathAnnot}[2]{#1 : #2}

\newcommand{\UTRAppVar}{\UnknownTypeRepair{Application}-Variable}
\newcommand{\UTRAppTerm}{\UnknownTypeRepair{Application}-Term}
\newcommand{\UTRLam}{\UnknownTypeRepair{$\lambda$}}
\newcommand{\UTRMatch}{\UnknownTypeRepair{Match}}
\newcommand{\UTRPi}{\UnknownTypeRepair{$\Pi$}}
\newcommand{\UTRType}{\UnknownTypeRepair{Universe}}
\newcommand{\UTRVar}{\UnknownTypeRepair{Variable}}
\newcommand{\UTRAnnot}{\UnknownTypeRepair{Annotation}}
\newcommand{\UTRHole}{\UnknownTypeRepair{Hole}}
\newcommand{\UTROtherwise}{\UnknownTypeRepair{Otherwise}}

\newcommand{\repairArgsOp}{\op{R_{\text{Arguments}}}}

\newcommand{\repairArgs}[5]{
  \op{\repairArgsOp(}#1\op{,} #2\op{,} #3\op{,} #4\op{) =}\ \out{#5}
}

\newcommand{\declDiff}[3]{
  \hasType
      { \diff{ #1 }{ \out{#2} } }
      { \diff{ \op{?} }{ \out{#3} } }
}

\newcommand{\MathMatch}[2]{ \coqinline{match}\ #1\ \coqinline{with}\ #2 }

%%% For `deriving`

\newcommand{\MathMkElimTypeOp}{\op{\text{EliminatorType}}}
\newcommand{\MathMkΔElimTypeOp}{\op{\text{ΔEliminatorType}}}
\newcommand{\MathMkΔElimType}[2]{\op{\text{ΔEliminatorType}(} #1 \op{,\ } #2 \op{)}}
\newcommand{\MathMkElimType}[1]{\op{\text{EliminatorType}(} #1 \op{)}}

%%% For `appendix-chick`
\newcommand{\Lookup}[1]{Lookup-{#1}}
\newcommand{\LookupCtxt}{\Lookup{Context}}
\newcommand{\LookupEnv}{\Lookup{Environment}}

\newcommand{\LC}[1]{Lookup-Context-{#1}}
\newcommand{\LCSame}{\LC{Same}}
\newcommand{\LCIns}{\LC{\Ins}}
\newcommand{\LCMod}[1]{\LC{\Mod{#1}}}
\newcommand{\LCDrop}{\LC{\Drop}}
\newcommand{\LCPermute}{\LC{\Permute}}

\newcommand{\GE}[1]{Lookup-Env-{#1}}
\newcommand{\GESame}{\GE{Same}}
\newcommand{\GEDrop}{\GE{\Drop}}
\newcommand{\GEIns}{\GE{\Ins}}
\newcommand{\GEMod}[1]{\GE{\Mod-{#1}}}
\newcommand{\GEPermute}{\GE{\Permute}}

\newcommand{\deltaInductiveTypeOp}{\op{\delta_{\text{InductiveType}}}}
\newcommand{\deltaInductiveType}[5]{
  \deltaInductiveTypeOp\op{(} #1\op{,}\ #2\op{,}\ #3\op{,}\ #4\op{) =\ } \out{#5}
}

\newcommand{\constructorTypeOp}{\op{\text{ConstructorType}}}
\newcommand{\constructorType}[3]{
  \constructorTypeOp\op{(} #1 \op{,\ }#2 \op{) =\ }\out{#3}
}

\newcommand{\deltaConstructorTypeOp}{\op{\delta_{\text{ConstructorType}}}}
\newcommand{\deltaConstructorType}[5]{
  \deltaConstructorTypeOp\op{(} \diff{#1}{#2} \op{,} \diff{#3}{#4} \op{) =\ }\out{#5}
}

\newcommand{\deltaEliminatorTypeOp}{\op{\delta_{\text{EliminatorType}}}}
\newcommand{\deltaEliminatorNameOp}{\op{\delta_{\text{EliminatorName}}}}
\newcommand{\deltaEliminatorType}[4]{
  \deltaEliminatorTypeOp\op{(} \diff{#1}{#2} \op{,} #3 \op{) =\ } \out{#4}
}
\newcommand{\deltaEliminatorName}[4]{
  \deltaEliminatorNameOp\op{(} \diff{#1}{#2} \op{,} #3 \op{) =\ } \out{#4}
}

\newcommand{\easy}{ \faStar \faStarO \faStarO }
\newcommand{\medium}{ \faStar \faStar \faStarO }
\newcommand{\hard}{ \faStar \faStar \faStar }

\newcommand{\MatchingToDiff}{\op{\text{MD}}}

\newcommand{\InductiveType}[5]{
  \op{\text{InductiveType}(} #1\op{,}\ #2\op{,}\ #3\op{,}\ #4\op{)} = \out{#5}
}

\newcommand{\nind}{n_{\text{ind}}}
\newcommand{\pind}{\overline{p_{\text{ind}}}}
\newcommand{\pindprime}{\overline{p'_{\text{ind}}}}
\newcommand{\iind}{\overline{i_{\text{ind}}}}
\newcommand{\uind}{u_{\text{ind}}}
\newcommand{\cind}{\overline{c_{\text{ind}}}}

\newcommand{\dnind}{\delta{n_{\text{ind}}}}
\newcommand{\dpind}{\delta{\overline{p_{\text{ind}}}}}
\newcommand{\dpindprime}{\delta{\overline{p'_{\text{ind}}}}}
\newcommand{\diind}{\delta{\overline{i_{\text{ind}}}}}
\newcommand{\duind}{\delta{u_{\text{ind}}}}
\newcommand{\dcind}{\delta{\overline{c_{\text{ind}}}}}

\newcommand{\nctor}{n_{\text{ctor}}}
\newcommand{\pctor}{\overline{p_{\text{ctor}}}}
\newcommand{\ictor}{\overline{i_{\text{ctor}}}}

\newcommand{\dnctor}{\delta_{n_{\text{ctor}}}}
\newcommand{\dpctor}{\delta_{\overline{p_{\text{ctor}}}}}
\newcommand{\dictor}{\delta_{\overline{i_{\text{ctor}}}}}

\newcommand{\uelim}{u_\text{elim}}

%\newcommand{\MathLabel}[1]{\normalfont \coqinline[fontsize=\scriptsize]{#1}}
\newcommand{\MathLabel}[1]{\normalfont \text{#1}}

% {\normalfont ...} is the same as \textnormal{...}
\newcommand{\DiffType}[1]{\Delta_{\MathLabel{#1}}}
\newcommand{\DeltaAtomic}{\DiffType{Atomic}}
\newcommand{\DeltaBinder}{\DiffType{Binder}}
\newcommand{\DeltaInductive}{\DiffType{Inductive}}
\newcommand{\DeltaList}{\DiffType{List}}
\newcommand{\DeltaTerm}{\DiffType{Term}}
\newcommand{\DeltaProgram}{\DiffType{Program}}

\newcommand{\MathSameLabel}[1]{\MathSame{}_{\MathLabel{#1}}}

\newcommand{\MathSameBinder}[1]{\MathSameLabel{Binder}}
\newcommand{\MathSameInductive}{\MathSameLabel{Inductive}}
\newcommand{\MathSameList}{\MathSameLabel{List}}
\newcommand{\MathSameProgram}[1]{\MathSameLabel{Program}}
\newcommand{\MathSameTerm}[1]{\MathSameLabel{Term}}
\newcommand{\MathSameUniverse}{\MathSameLabel{Universe}}
\newcommand{\MathSameVariable}[1]{\MathSameLabel{Variable}}
\newcommand{\MathSameVernacular}[1]{\MathSameLabel{Vernacular}}

\newcommand{\deltaType}[1]{\delta_{\MathLabel{#1}}}
\newcommand{\deltaConstructor}{\deltaType{Constructor}}
\newcommand{\deltaInductive}{\deltaType{Inductive}}

\newcommand{\range}[2]{\overline{#2}^{i \in \{ 1, …, #1 \}}}
\newcommand{\permutationRange}[1]{\range{|p|}{#1}}
%\newcommand{\permutationRange}[1]{\overline{#1}^{i \in \{ 1, …, |p| \}}}
\newcommand{\deprecateOp}[1]{\op{\text{Deprecate}}}
\newcommand{\deprecate}[1]{\op{\deprecateOp{}(} #1 \op{)}}

\newcommand{\TODO}[1]{\noindent\colorbox{yellow}{#1}}
