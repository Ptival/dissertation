\section{Design of \Chick{}}\label{chick-design}

\begin{figure}[!htp]

  \centering
  \begin{tikzpicture}[node distance=0.8cm]

    \tikzstyle{box} = [%
    align=center,
    draw=black,
    font=\bfseries,
    inner sep=2ex,
    line width=3pt,
    rectangle,
    %ultra thick,
    ];

    \tikzstyle{doc}=[%
    align=center,
    draw=black,
    shape=document,
    inner sep=2ex,
    shape=document,
    thick,
    ]

    \node[doc]                           (modified) {Partially\\refactored\\program};
    \node[doc,below=of modified]         (original) {Original\\program};
    \node[box,right=of modified]         (guess)    {Guess\\diff};
    \node[doc,right=of guess]            (guessed)  {Partial\\refactoring\\diff};
    \node[box,right=of guessed]          (repair)   {Repair};
    \node[doc,right=of repair]           (repaired) {Completed\\refactoring\\diff};
    \node[box] at (repaired |- original) (patch)    {Patch};
    \node[doc,below=of patch]            (patched)  {Refactored\\program};

    \draw[->,rounded corners,ultra thick]
    (original.east)
    -- (guess.south |- original.east)
    -- (guess.south)
    ;

    \draw[->,ultra thick] (modified.east) -- (guess.west);
    \draw[->,ultra thick] (guess.east)    -- (guessed.west);

    \draw[->,rounded corners,ultra thick]
    (original.east)
    -- (repair.south |- original.east)
    -- (repair.south)
    ;

    \draw[->,ultra thick] (guessed.east) -- (repair.west);
    \draw[->,ultra thick] (repair.east)  -- (repaired.west);

    \draw[->,ultra thick] (original.east)  -- (patch.west);
    \draw[->,ultra thick] (repaired.south) -- (patch.north);
    \draw[->,ultra thick] (patch.south)    -- (patched.north);

  \end{tikzpicture}

  \caption{\Chick{}'s workflow}~\label{chick-workflow}

\end{figure}

In order to help programmers in dependently-typed languages refactor their
programs, we built a prototype tool called \Chick{}.  The high-level workflow of
\Chick{} is given in Figure~\ref{chick-workflow}.  The main idea is to take the
original, existing program prior to the refactoring attempt, and the modified
program where a partial refactoring has been performed, and automatically finish
propagating changes through the rest of the program.  However, we do not perform
this operation all at once.  Instead, our workflow consists in three phases.

First, we analyze the original program against the modified program to build a
list of all the differences between the two.  We will describe the data types
used to represent those differences in Section~\ref{chick-diffs}, but will defer
explanations about how this guess is performed until Section~\ref{chick-guess}.
We try to capture the intent of the programmer in this list of differences; as
we will discuss, there can be several ways of interpreting the same change, and
the repair performed will depend on which interpretation is chosen.

Second, this list of differences is used to compute a repair of the program.
Here again, we do not directly compute a repaired program (a value of type
\coqinline{program}), but instead compute a value of type $\DeltaProgram$ (as
described in Section~\ref{chick-diffs}) describing how the program must be
modified in order to be repaired.  We do so because the diff contains more
information about the changes than the repaired program, as we will demonstrate
in Section~\ref{chick-diffs-importance}.

Third and finally, the repaired program is obtained by simply applying the
repaired diff to the original program.

\begin{figure*}[!htp]

  \noindent%
  \begin{minipage}[t]{0.50\textwidth}
    \begin{minted}[fontsize=\footnotesize,linenos=false]{coq}
Inductive nat : Set :=
| O : nat
| S : ∀ (n : nat), nat.

Inductive list (A : Type)
: Type :=
| nil : list A
| cons : A →
    list A → list A.

Definition a_list
: list nat :=
  cons nat (S (S (S O)))
    (nil nat).

Definition length :
  ∀ (T : Type),
    list T → nat :=
  λ T l, list_rect T (λ _, nat) O
    (λ _ _ lt, S lt) l.

Fixpoint map :
  ∀ (A B : Type), (A → B) →
    list A →
    list B :=
  λ _ B f l,
    match l with
    | nil _      => nil B
    | cons _ h t =>
      cons B (f h)
           (map A B f t)
    end.
  \end{minted}
\end{minipage}%
\begin{minipage}[t]{0.50\textwidth}
  \begin{minted}[fontsize=\footnotesize,escapeinside=@@,linenos=false]{coq}
Inductive nat : Set :=
| O : nat
| S : ∀ (n : nat), nat.

Inductive @\modified{vec}@ (A : Type)
: @\modified{nat →}@ Type :=
| @\modified{vnil}@ : @\modified{vec}@ A @\modified{O}@
| @\modified{vcons}@ : A → @\modified{∀ (n : nat),}@
    @\modified{vec}@ A @\modified{n}@ → @\modified{vec}@ A @\modified{(S n)}@.

Definition a_list
: @\repaired{vec}@ nat @\repaired{(\_ : nat)}@ :=
  @\repaired{vcons}@ nat @\repaired{(\_ : nat)}@ (S (S (S O)))
    (@\repaired{vnil}@ nat).

Definition length :
  ∀ (T : Type),
    @\repaired{vec}@ T @\repaired{(\_ : nat)}@ → nat :=
  λ T l, @\repaired{vec\_rect}@ T (λ _, nat) O
    (λ _ _ lt, S lt) l.

Fixpoint map :
  ∀ (A B : Type), (A → B) →
    @\repaired{vec}@ A @\repaired{(\_ : nat)}@ →
    @\repaired{vec}@ B @\repaired{(\_ : nat)}@ :=
  λ _ B f l,
    match l with
    | @\repaired{vnil}@ _         => @\repaired{vnil}@ B
    | @\repaired{vcons}@ _ @\repaired{\_}@ h t =>
      @\repaired{vcons}@ B @\repaired{\_}@ (f h)
            (map A B f t)
    end.
  \end{minted}
\end{minipage}

\caption%
[Running example of a \Chick{} repair]
{Running example of a \Chick{} repair.  The user-provided partial refactoring is highlighted in orange.  The computed repairs are highlighted in teal.}~\label{listtovec}

\end{figure*}

Figure~\ref{listtovec} gives a concrete example of the kind of repair we want to
propagate.  We will use it as our running example in the following sections.  On
the left-hand side is an original program, prior to any attempt at refactoring
it.  In this example, the user defined a data type \coqinline{list}, and a
couple of definitions and functions over this data type.  Let us now assume they
would like to replace this type list with a richer type of length-indexed lists.
We chose this example because it brings up many of the feature we would like to
have.  On the right-hand side of the same figure, the orange part is the partial
refactoring the user has manually done.  The green parts on the right-hand side
constitute the rest of the refactoring, as automatically produced by our tool.

In the \coqinline{a_list} definition, the type must be renamed, and a type index
must be added.  While the repair algorithm knows that an index must be inserted,
it does not know what value to use: here, because the length is static, the
right value would be \coqinline{S O}, but in more complex cases (as in the next
function, \coqinline{length}), the value can be a new variable, or the result of
an arbitrary computation, so guessing what it should be is a hard problem.
Instead, we simply insert a typed hole \coqinline{(_ : nat)}.  The user might
need to eventually provide a concrete value of type \coqinline{nat}, though,
there are cases where the type system will infer such value without any user
assistance.  The term of the definition was also updated, to rename the
constructor and add a type hole for its new argument.  While this change seems
trivial, it relies on our ability to match constructors from the old inductive
definition to constructors in the new inductive definition.  In order to guess
that \coqinline{nil} and \coqinline{vnil} are related, and that \coqinline{cons}
and \coqinline{vcons} are related, our analysis must evaluate the most likely
pairs of matching constructors.  Since a user might also have removed a
constructor, and created an entirely different one, we also want to assess our
confidence in matching two constructors.

The definition of \coqinline{length} is updated in a similar way, but notice
that the function call being updated is a call to the \define{eliminator}
\coqinline{vec_rect}.  We will describe eliminators later, in
Section~\ref{chick-deriving}, but for now, all we need to highlight is that this
eliminator is never explicitly defined in the user code: it is a function that
is automatically generated when the \coqinline{vec} definition is accepted.  Our
tool supports patching such functions, as long as we describe to it how those
functions change when the inductive definition they are generated from changes.

Finally, in the definition of \coqinline{map}, the patterns of a
\coqinline{match} construct are updated.  This will require some care since it
may introduce or remove binders, but for this example, our repair algorithm took
the conservative approach of not binding the new arguments, using a wildcard
\coqinline{_} pattern.
