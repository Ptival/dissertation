\subsection{Repairing programs}~\label{repair-program}

The repair algorithm for programs $\repairProgOp$ is described formally in
Figure~\ref{fig:repair-program} using the following judgment:

{
  \[
    {\turnstile
      {\denv{E}{\delta_E}}
      {\repairProg{p}{\delta_p}{\delta_p'}}
    }
  \]
}

\paragraph{Notation} This judgment can be read as follows:
\begin{itemize}
\item \textbf{Top}: If the original program $p$ type-checked in the original environment $E$,
\item \textbf{Bottom left}: and the environment underwent modification $\delta_{E}$,
\item \textbf{Bottom center}: and the program underwent modification $\delta_{p}$,
\item \textbf{Right}: then $\repairProgOp$ proposes to repair the program with modification $\out{\delta'_{p}}$.
\end{itemize}

To be more precise, $E$ is the environment in which the original program $p$ was
defined: a list of term definitions, with type $\GlobalDefinitionText$, and of
inductive data type definitions, with type $\GlobalInductiveText$ (as presented
in Section~\ref{chick-syntax-programs}).  Each vernacular command is
type-checked in such a global typing environment, and upon being executed,
populates it with some additional definitions for the subsequent commands.

$\delta_E$ is a diff indicating how the global environment has been changed by
the time $p$ is reached by the repair algorithm: it accounts for whether some
definitions have been added, removed, or modified, in the prefix of the already
repaired program preceding $p$.  For instance, in our motivating example, by the
time we reach the inductive definition being modified (\coqinline{list},
becoming \coqinline{vec}), $E$ would contain the inductive data type definition
for the type \coqinline{nat}, and $\delta_E$ would keep it intact.  After this
definition has been processed, $E$ would contain, in order, \coqinline{nat}, and
\coqinline{list}, and $\delta_E$ would still keep the former intact, but would
register the diff turning \coqinline{list} into \coqinline{vec}.

\begin{Rules}{fig:repair-program}%
  { Rules for repairing programs ($\repairProgOp$) }

  \begin{mathpar}
    \RuleRProgSameNil{}

    {
      \inferrule*
          [lab=\RProgSameCons]
          {
            {\turnstile%
              { \denv{E}{\delta_E} }
              { \repairProg%
                {\MathCons{v}{p}}
                {\MathMod{\MathSameVernacular{}}{\MathSameProgram{}}}
                {\delta}
              }
            }
          }
          {\turnstile%
            { \denv{E}{\delta_E} }
            { \repairProg{\MathCons{v}{p}}{\MathSameProgram{}}{\delta} }
          }
    }

    {
      \inferrule*
          [lab=\RProgReplace]
          { }
          {\turnstile%
            { \denv{E}{\delta_E} }
            { \repairProg{p}{\MathReplace{q}}{\MathReplace{q}} }
          }
    }

    {
      \inferrule*
          [lab=\RProgIns]
          {\turnstile%
            { \denv%
              {E}
              {\MathIns{v}{\delta_E}}
            }
            { \repairProg{p}{\delta_{p}}{\delta} }
          }
          {\turnstile%
            { \denv{E}{\delta_E} }
            { \repairProg{p}{\MathIns{v}{\delta_{p}}}{\delta} }
          }
    }

    \RuleRProgModify{}

    {
      \inferrule*
          [lab=\RProgDrop]
          {
            {\turnstile%
              { \denv{v \Cons{} E}{\MathDrop{\delta_E}} }
              { \repairProg{p}{\delta_p}{\delta_p'} }
            }
          }
          {\turnstile%
            { \denv{E}{\delta_E} }
            { \repairProg{\MathCons{v}{p}}{\MathDrop{\delta_p}}{\MathDrop{\delta_p'}} }
          }
    }

    {
      \inferrule*
          [lab=\RProgPermute]
          {
            { \MathPatches{q}{\MathPermute{p}{}}{q'} }
            \and
            {\turnstile%
              { \denv{E}{\delta_E} }
              { \repairProg{q'}{\delta}{\delta'} }
            }
          }
          {\turnstile%
            { \denv{E}{\delta_E} }
            { \repairProg{q}{\MathPermute{p}{\delta}}{\MathPermute{p}{\delta'}} }
          }
    }

    \end{mathpar}

\end{Rules}


The repair algorithm for programs takes as input the original program $p$, and
the user-provided modification $\delta_p$.  From those, it outputs a repaired
modification $\out{\delta_p'}$, that propagates the changes introduced by
$\delta_E$ and $\delta_p$ to the rest of the program $p$.  The algorithm
essentially folds over the sequence of vernacular commands that make up the
program, propagating changes from previous commands to subsequent ones.

Rule \rulename{\RProgMod} does the bulk of the work, dispatching the repair of
each vernacular command to the repair algorithm for vernacular commands
$\repairVernacOp$ (described in Section~\ref{repair-vernacular}).  First, the
head vernacular command $v$ is repaired (using $\repairVernacOp$), returning a
repaired diff $\out{\delta_v'}$.  Then, we want to repair the rest of the
program, but the changes made to $v$ may affect the global environment for the
subsequent commands.  For instance, in our motivating example, the vernacular
command defining the inductive type \coqinline{list} gets modified to define the
inductive type \coqinline{vec} instead: when we repair the rest of the program,
we must remember that the original program was defined in a global environment
where \coqinline{list} was defined, and that the updated program must replace it
with \coqinline{vec}.  We therefore repair the rest of the program $p$ in the
appropriate updated environment and its diff.

\rulename{\RProgSameCons} simply defers to \rulename{\RProgMod}: even though it
processes unchanged commands, they might need repairs to account for changes in
their dependencies, as accounted for in the global environment diff.

\TODO{Describe the remaining rules}

\paragraph{Note} While our presentation makes it look like vernacular commands
each add exactly one element to the global environment, it is not always as
simple: inductive data type definitions add \begin{enumerate*} \item the
inductive type (e.g. \coqinline{list}), \item all its constructors
(e.g. \coqinline{nil}, \coqinline{cons}), and \item all its eliminators
(e.g. \coqinline{list_rec}, \coqinline{list_ind}, as described later in
Section~\ref{chick-deriving}) \end{enumerate*}.  For simplifying diff
operations, the whole $\InductiveText$ effectively acts as a placeholder, in our
formalism, for all of those at once.  The lookup rules, given in
Appendix~\ref{appendix-chick}, encapsulate the complexity of accounting for all
the definitions arising from one inductive data type definition.
