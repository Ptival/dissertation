\subsection{Describing program modifications with diffed data structures}\label{chick-design-diffs}

We define a family of data types that allows us to capture changes made to
programs from the language presented in
section~\ref{chick-design-syntax-programs}.  We will refer to these descriptions
of changes as \textit{diffs}.  We will usually denote a \textit{diff type}
$\delta_{\tau}$ if it corresponds to values of the diffed type $\tau$.  For
instance, the type $\delta_{\text{universe}}$ would capture how values of type
\coqinline{universe} may have been modified).

Importantly, note that we are not talking about values and types as they appear
in the user's program, but rather, about the internal values and types of the
meta-language.  For instance, if a user changes a \coqinline{bool} value in
their program from \coqinline{true} to \coqinline{false}, we will capture this
change using $\delta_{\text{term}}$, the diff type for $\synt{term}$; whereas
values of the diff type $\delta_{\text{bool}}$ would describe changes made to
boolean flags in the meta-language.

When the user makes a change to their program, we will attempt to guess the
structure of their changes, as a value of one of the program diff type
$\delta_{\text{program}}$.  To illustrate the kind of changes we are interested
in describing, let us go back to our motivating example in
Figure~\ref{listtovec}, where we can observe the following user-provided
modifications (dashed blue outline):

\begin{enumerate}

\item they renamed \coqinline{list} into \coqinline{vec},

\item added an index of type \coqinline{nat},

\item renamed constructor \coqinline{nil} into \coqinline{vnil},

\item instantiated the index for the first constructor with \coqinline{O},

\item renamed constructor \coqinline{cons} into \coqinline{vcons},

\item added a parameter \coqinline{n} of type \coqinline{nat} to the second
  constructor,

\item updated the recursive occurrence's name,

\item updated the recursive occurrence's index,

\item and instantiated the index for the second constructor with \coqinline{(S n)}.

\end{enumerate}

Intuitively, we want to capture changes like insertion, modifications,
deletions, and permutations, at all syntactic levels of the meta-language
(within terms, within inductive declarations).  We will use the same
descriptions to capture user-provided and repair-generated modifications.

Every diff type is accompanied by a patching function, which, given an element
of the diffed type and an element of the diff type produces, when successful, a
patched element.  This operation is partial for many diff types, because they
often capture modifications that only make sense for certain constructors of the
diffed type: for instance, a diff that says the head of a list has been modified
does not apply to the empty list.  We will overload the notation
$\MathPatches{x}{\delta_x}{x'}$ to indicate that $x'$ is the (optional) value
obtained when (successfully) patching $x$ according to the diff $\delta_x$,
using the relevant patching function for that diff type.  In all of our
notations, we use a black frame and a teal highlight to indicate that a value is
an output.

\subsection{Atomic diff}

There are many data types for which we will only capture changes at an atomic
granularity: either the value is the same in the new program, or it has been
replaced with a different, unrelated value.  For instance, a binder can either
have the same name, or have been renamed.  Similarly, the only possible change
to the recursive flag of a definition is to be atomically changed to a different
value.  The parameterized $\delta_{atomic}$ diff type captures such cases for a
given type $\tau$:

\begin{grammar}
<$\delta_{atomic}\ \tau$> ::= \ %trick LaTeX
\alt $\MathSame$                        \hfill (unchanged)
% \alt $\MathReplace{\langle\tau\rangle}$ \hfill (replaced)
\end{grammar}
%
with the following semantics:

\begin{mathpar}

  {
    \inferrule*
    [right=Identity]
    {  }
    {\MathPatches{x}{\MathSame}{x}}
  }

  {
    \inferrule*
    [right=Replace]
    {  }
    {\MathPatches{x}{\MathReplace{y}}{y}}
  }

\end{mathpar}

\subsection{List diff} \label{list-diff}

\subsubsection{Syntax}

For lists, we provide a rich selection of diff operations.  The aim is not to
have a canonical representation, but rather to capture closely the intent of
the user modifications:

\begin{grammar}
<$\delta_{\text{list}} \ \tau \ \delta_{\tau}$> ::= \ %trick LaTeX
\alt \synt{$\delta_{\text{atomic}}\ \tau$} \hfill (atomic modification)
\alt \begin{tabular}{p{0.7cm} >{\centering}p{0.7cm} l}$\langle\tau\rangle$
       & $\MathIns{}{}$
       & \synt{$\delta_{list}\ \tau\ \delta_{\tau}$} \\\end{tabular} \hfill
     (insert a head)
\alt \begin{tabular}{p{0.7cm} >{\centering}p{0.7cm} l}$\langle\delta_\tau\rangle$
       & $\MathMod{}{}$
       & \synt{$\delta_{list}\ \tau\ \delta_{\tau}$} \\\end{tabular}
     \hfill (modify and keep the head)
%\alt \begin{tabular}{p{0.3cm} >{\centering}p{0.6cm} l}\quad & $\MathKeep{}$
%       & \synt{$\delta_{list}\ \tau\ \delta_{\tau}$} \\\end{tabular}
%     \hfill (keep the head)
\alt \begin{tabular}{p{0.7cm} >{\centering}p{0.7cm} l}\quad
       & $\MathDrop{}$
       & \synt{$\delta_{list}\ \tau\ \delta_{\tau}$} \\\end{tabular}
     \hfill (drop the head)
\alt \begin{tabular}{p{0.7cm} >{\centering}p{0.7cm} l}\quad
       & $\MathPermute{p}{}$
       & \synt{$\delta_{\text{list}}\ \tau\ \delta_{\tau}$} \\\end{tabular}
     \hfill (permute according to a permutation $p$)
\end{grammar}
%
with the following semantics:

\begin{mathpar}
  {
    \inferrule*
    [right=Insert]
    {\MathPatches{l}{\delta_{l}}{l'}}
    {\MathPatches{l}{\MathIns{h}{\delta_{l}}}{(h :: l')}}
  }

  {
    \inferrule*
    [right=Modify]
    {\MathPatches{h}{\delta_{h}}{h'} \quad \MathPatches{t}{\delta_{t}}{t'}}
    {\MathPatches{(h :: t)}{\MathMod{\delta_{h}}{\delta_{t}}}{(h' :: t')}}
  }
\\
  % {
  %   \inferrule*
  %   [right=Keep]
  %   {\MathPatches{t}{\delta_{t}}{t'}}
  %   {\MathPatches{(h :: t)}{\MathKeep{\delta_{t}}}{(h :: t')}}
  % }
  {
    \inferrule*
    [right=Drop]
    {\MathPatches{t}{\delta_{t}}{t'}}
    {\MathPatches{(h :: t)}{\MathDrop{\delta_{t}}}{t'}}
  }

  {
    \inferrule*
    [right=Permute]
    {\MathPatches{(h_{p(1)} \Cons \ldots \Cons h_{p(|p|)} \Cons t)}{\delta}{l}}
    {\MathPatches{(h_1 \Cons \ldots \Cons h_{|p|} \Cons t)}{\MathPermute{p}{\delta}}{l}}
  }

\end{mathpar}

Note that we defined the semantics of $\MathModPiOp$ so that it both modifies
and keeps the head: the recursive occurrence in rule~\rulename{Modify},
$\delta_t$, therefore applies to the tail $t$ and not the whole list after the
head has been repaired.  On the other hand, the recursive occurrence in
rule~\rulename{Permute}, $\delta$, targets the entire list after the permutation
is performed, not solely the tail: this is necessary because we will want to
perform modifications of elements after having shuffled them around.

\subsection{Term diff}

The diffs for terms include atomic changes, as well as insertion, modification,
deletion, and permutation of most constructors.  We illustrate a couple of
these:

\begin{grammar}
<$\delta_{term}$> ::= \ %trick LaTeX
\alt \synt{$\delta_{atomic}\ t$} \hfill (atomic modification)
\alt \synt{$\delta_{term}$} $\oIns{ \$ }$ \synt{$\delta_{term}$} \hfill
(insert application)
\alt $\oIns{\lambda}$ \synt{v}, \synt{$\delta_{term}$} \hfill (insert value
abstraction)
\alt $\oIns{\Pi}$ (\synt{v} : \synt{$\delta_{term}$}),
\synt{$\delta_{term}$} \hfill (insert type abstraction)
\alt ... \hfill (other insertions)
\alt ... \hfill (removals/modifications/permutations)
\end{grammar}

\noindent Note that we use an infix dollar sign ($\$$) as a symbol for function
application in our diffs, even though we use an infix space for function
application in our terms, which is not ideal but should prevent confusion.  For
our purpose, we biased the diffs on binary operations in the least surprising
way: deleting a function application keeps its left child, i.e. removes the
function call and keeps the function (see rule~\rulename{\RmApp}), while
deleting a \coqinline{Pi} keeps its right child, i.e. removes the value being
quantified but keeps the return type (see rule~\rulename{\RmPi}).  For
insertion, we allow maximal flexibility by passing the entire old term to all
recursive occurrences: for instance, the diff
$(\MathInsApp{\MathSame}{\MathSame})$ turns any term $t$ into the
self-application $(t\ t)$ (see rule~\rulename{\InsApp}).  However, it is often
the case that only one recursive occurrence will use the original term, while
the other ones will replace it: for instance, the diff
$(\MathInsApp{\MathSame}{\MathReplace{x}})$ turns any term $t$ into the
application $(t\ x)$.

\begin{mathpar}
  {
    \inferrule*
    [right=\RmApp]
    {\MathPatches{f}{\delta}{f'}}
    {\MathPatches{f\ x}{\MathDropApp{\delta}}{f'}}
  }

  {
    \inferrule*
    [right=\RmPi]
    {\MathPatches{\tau_{2}}{\delta}{\tau_{2}'}}
    {\MathPatches{\MathPi{\tau_{1}}{x}{\tau_{2}}}{\MathDropPi{\delta}}{\tau_{2}'}}
  }

\\

  {
    \inferrule*
    [right=\InsApp]
    {\MathPatches{t}{\delta_1}{t_1} \quad \MathPatches{t}{\delta_2}{t_2}}
    {\MathPatches{t}{\MathInsApp{\delta_1}{\delta_2}}{t_1\ t_2}}
  }

\end{mathpar}

\subsection{Other diff types}

We also need diffs for many other internal data types.  Diff types for tuples
are derived from diff types of their constituents in a straightforward way.
Inductive data type definitions, as well as constructor definitions, behave
essentially like a tuple of all their arguments, so their diff type is derived
accordingly.
