\subsection{Repairing vernacular commands} \label{repair-vernacular}

We formally describe the repair algorithm for vernacular commands
$\repairVernacOp$ in Figure~\ref{fig:repair-vernac} using the following
judgment:

{
  \[
    {\turnstile
      {\denv{E}{\delta_E}}
      {\repairVernac{v}{\delta_v}{\delta_v'}}
    }
  \]
}

\noindent which can be read analogously to $\repairProgOp$ as described in
Section~\ref{repair-program}.

\begin{Rules}
  {fig:repair-vernac}
  { Rules for repairing vernacular commands ($\repairVernacOp$) }

  \begin{mathpar}
    {
      \inferrule*
      [lab=\RepairVernac{Definition}]
      {
        {\MathPatches{\tau}{\delta_\tau}{\tau'}}
        \and
        {\turnstile
          { \dcontext{E}{\delta_E}{[]}{\MathSame} }
          { \repairTerm{\tau'}{\delta_\tau'}{\MathType}{\MathSame} }
        }
        \and
        {\MathPatches{\tau'}{\delta_\tau'}{\tau''}}
        \\\\\\
        {\turnstile
          {\denv{E}{\delta_E}}
          {\freshOne{n}{\delta_n}{\delta_{n}'}}
        }
        \and
        {\MathPatches{n}{\delta_{n}'}{n'}}
        \and
        {
          \Gamma \op{=}
          \begin{cases}
            []                                      & \text{if } k = \text{Definition}\\
            \MathCons{\MathLocalAssum{n}{\tau}}{[]} & \text{if } k = \text{Fixpoint}
          \end{cases}
        }
        \\\\\\
        {
          \delta_\Gamma \op{=}
          \begin{cases}
            \MathIns{\MathLocalAssum{n'}{\tau''}}{\MathSame} & \text{when } k = \text{Definition}\\
                                                             & \phantom{\text{when }} \delta_k = \MathReplace{\text{Fixpoint}}\\
            \MathDrop{\MathSame}                             & \text{when } k = \text{Fixpoint}\\
                                                             & \phantom{\text{when }} \delta_k = \MathReplace{\text{Definition}}\\
            \MathSame                                        & \text{otherwise}
          \end{cases}
        }
        \\\\\\
        {\MathPatches{t}{\delta_t[\subst{\delta_n}{\delta_{n}'}]}{t'}}
        \and
        {\turnstile
          {\dcontext{E}{\delta_E}{\Gamma}{\delta_\Gamma}}
          {\repairTerm{t'}{\delta_{t}'}{\tau}{\delta_\tau'}}
        }
      }
      {\turnstile
        { \denv{E}{\delta_E} }
        { \repairVernac
          {\Definition{k}{n}{\tau}{t}}
          {\ModifyDefinition{\delta_k}{\delta_n}{\delta_\tau}{\delta_t}}
          {\ModifyDefinition{\delta_k}{\delta_n'}{\delta_\tau'}{\delta_t'}}
        }
      }
    }

    {
      \inferrule*
      [lab=\RepairVernac{Inductive}]
      {
        I = \Inductive{\nind{}}{\pind{}}{\iind{}}{\uind{}}{\cind{}}
        \\\\\\
        \delta_{I} = \ModifyInductive{\dnind{}}{\dpind{}}{\diind{}}{\duind{}}{\dcind{}}
        \\\\\\
        {
          \turnstile%
          {\denv{E}{\delta_E}}
          {\freshOne{\nind{}}{\dnind{}}{\dnind{}'}}
        }
        \\\\\\
        {\turnstile%
          {
            \denv%
            {E}
            {\delta_E}
          }
          {
            \repairInd%
            {\diff{\nind{}}{\dnind{}}}
            {\diff{\pind{}}{\dpind{}}}
            {\diff{\iind{}}{\diind{}}}
            {\diff{\uind{}}{\duind{}}}
            {\diff{\cind{}}{\dcind{}}}
            {(\dpind{}', \diind{}', \dcind{}')}
          }
        }
      }
      {
        \turnstile%
        { \denv{E}{\delta_E} }
        { \repairVernac{I}{\delta_{I}}
          {
            \ModifyInductive{\dnind{}'}{\dpind{}'}{\diind{}'}{\duind{}}{\dcind{}'}
          }
        }
      }
    }
  \end{mathpar}

\end{Rules}


Rule \rulename{\RepairVernac{Def}} shows how we attempt to repair a vernacular
term definition.  It is quite involved because we account for the possibility of
a non-recursive definition becoming recursive and vice-versa.  We want to
account for the user-provided modifications $\delta_k$, $\delta_n$,
$\delta_\tau$, and $\delta_t$, but we might need to repair some of them.  We
first repair the type $\tau$, and obtain its repaired diff $\delta_\tau'$.  Now,
the user has modified the name of the definition according to $\delta_n$, but
there are two situations where we must intervene:

\begin{enumerate}

\item If $\delta_n$ is $\MathSame$, the user intends to keep the name $n$ for
this definition.  However, they could have introduced another name $n$ in the
global environment, which would be accounted for in $\delta_E$.  In this case,
we need to come up with a fresh name that is free in the new environment.

\item If $\delta_n$ is $\MathReplace{m}$, the user intends to rename this
definition from $n$ into $m$.  Again, even though this is unlikely in practice,
it could be a problem if the new global environment contains an $m$ already.  In
this case, we also need to come up with a fresh name that is free in the new
environment.

\end{enumerate}

\noindent $\freshOneOp$ takes care of these two situations: it takes as input a
name $n$, and a desired modification for it $\delta_n$, and returns a suitable
modification $\delta_n'$, that first tries to preserve $\delta_n$ when possible,
otherwise tries to preserve $n$ when possible, or finally, comes up with a fresh
name when neither of these conditions can be met.

Finally, in order to repair the body of the definition, we need to build an
initial local context (as described in Section~\ref{chick-lookup}).  This
context is empty for non-recursive definitions, but should contain a
self-reference for recursive definitions.  The side conditions defining $\Gamma$
and $\delta_\Gamma$ make sure that the local context is appropriately set in all
possible combinations of the recursive flag before and after the
user-modification.
