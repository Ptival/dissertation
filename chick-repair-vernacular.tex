\subsection{Repairing vernacular commands} \label{repair-vernacular}

We formally describe the repair algorithm for vernacular commands
$\repairVernacOp$ in Figure~\ref{fig:repair-vernac} using the following
judgment:

{
  \[
    {\turnstile
      {\denv{E}{\delta_E}}
      {\repairVernac{v}{\delta_v}{\delta_v'}}
    }
  \]
}

\noindent The rule \rulename{R-Vernac-Def} shows how we attempt to repair a
vernacular term definition.  It is quite involved because we account for the
possibility of a non-recursive definition becoming recursive and vice-versa.  We
want to account for the user-provided modifications $\delta_k$, $\delta_n$,
$\delta_\tau$, and $\delta_t$, but we might need to repair some of them.  We
first repair the type $\tau$, and obtain its repaired diff $\delta_\tau'$.  Now,
the user has modified the name of the definition according to $\delta_n$, but
there are two situations where we must intervene:

\begin{enumerate}

\item If $\delta_n$ is $\MathSame$, the user intends to keep the name $n$ for
this definition.  However, they could have introduced in the global environment,
which would be accounted for in $\delta_E$. In this case, we need to come up
with a fresh name that is free in the new environment.

\item If $\delta_n$ is $\MathReplace{m}$, the user intends to rename this
definition from $n$ into $m$.  Again, this is a problem if the new global
environment contains an $m$ already.  In this case, we also need to come up with
a fresh name that is free in the new environment.

\end{enumerate}

\noindent $\freshOneOp$ takes care of these two situations: it takes as input a
name $n$, and a desired modification for it $\delta_n$, and returns a suitable
modification $\delta_n'$, that tries to preserve $\delta_n$ when it is possible,
to preserve $n$ otherwise when it is possible, or comes up with a fresh name
when neither of these conditions can be met.

Finally, in order to repair the body of the definition, we need to build an
initial local context (as described in Section~\ref{chick-lookup-rules}).  This
context is empty for non-recursive definitions, but should contain a
self-reference for recursive definitions.  The side conditions defining $\Gamma$
and $\delta_\Gamma$ make sure that the local context is appropriately set in all
possible combinations of the recursive flag before and after the
user-modification.

We omit most of the details of repairing an inductive data type definition
behind the symbol $\repairIndOp$.  It essentially needs to go recursively in all
parameters and indices of both the inductive type itself and all of its
constructors and repair them.  Each subsequent parameter must be repaired in a
local context where its predecessor parameter has been accounted, so as to react
to possible modifications: for instance, if the first parameter must be renamed,
and the second parameter's type refers to the first parameter's name, the type
will need to be repaired.  Indices also must repaired in the context of the
repaired parameters.  Constructors are repaired one by one, in isolation, but
the repairing of their parameters.  Constructor indices are simply
instantiations of the inductive indices with terms: they are simply repaired in
isolation.

\begin{Rules}
  {fig:repair-vernac}
  { Rules for repairing vernacular commands ($\repairVernacOp$) }

  \begin{mathpar}
    {
      \inferrule*
      [lab=\RepairVernac{Definition}]
      {
        {\MathPatches{\tau}{\delta_\tau}{\tau'}}
        \and
        {\turnstile
          { \dcontext{E}{\delta_E}{[]}{\MathSame} }
          { \repairTerm{\tau'}{\delta_\tau'}{\MathType}{\MathSame} }
        }
        \and
        {\MathPatches{\tau'}{\delta_\tau'}{\tau''}}
        \\\\\\
        {\turnstile
          {\denv{E}{\delta_E}}
          {\freshOne{n}{\delta_n}{\delta_{n}'}}
        }
        \and
        {\MathPatches{n}{\delta_{n}'}{n'}}
        \and
        {
          \Gamma \op{=}
          \begin{cases}
            []                         & \text{if } k = \text{Definition}\\
            [\MathLocalAssum{n}{\tau}] & \text{if } k = \text{Fixpoint}
          \end{cases}
        }
        \\\\\\
        {
          \delta_\Gamma \op{=}
          \begin{cases}
            \MathIns{\MathLocalAssum{n'}{\tau''}}{\MathSame} & \text{when } k = \text{Definition}\\
                                                             & \phantom{\text{when }} \delta_k = \MathReplace{\text{Fixpoint}}\\
            \MathDrop{\MathSame}                             & \text{when } k = \text{Fixpoint}\\
                                                             & \phantom{\text{when }} \delta_k = \MathReplace{\text{Definition}}\\
            \MathSame                                        & \text{otherwise}
          \end{cases}
        }
        \\\\\\
        {\MathPatches{t}{\delta_t[\subst{\delta_n}{\delta_{n}'}]}{t'}}
        \and
        {\turnstile
          {\dcontext{E}{\delta_E}{\Gamma}{\delta_\Gamma}}
          {\repairTerm{t'}{\delta_{t}'}{\tau}{\delta_\tau'}}
        }
      }
      {\turnstile
        { \denv{E}{\delta_E} }
        { \repairVernac
          {\Definition{k}{n}{\tau}{t}}
          {\ModifyDefinition{\delta_k}{\delta_n}{\delta_\tau}{\delta_t}}
          {\ModifyDefinition{\delta_k}{\delta_n'}{\delta_\tau'}{\delta_t'}}
        }
      }
    }

    {
      \inferrule*
      [lab=\RepairVernac{Inductive}]
      {
        {\turnstile
          {\denv{E}{\delta_E}}
          {\freshOne{n}{\delta_n}{\delta_{n}'}}
        }
        \and
        {\turnstile
          {\denv{E}{\delta_E}}
          {\repairInd
            {\overrightarrow{p}}
            {\delta_{\overrightarrow{p}}}
            {\overrightarrow{i}}
            {\delta_{\overrightarrow{i}}}
            {\overrightarrow{c}}
            {\delta_{\overrightarrow{c}}}
            {(\delta_{\overrightarrow{p}}', \delta_{\overrightarrow{i}}',
              \delta_{\overrightarrow{c}}')}
          }
        }
      }
      {\turnstile
        { \denv{E}{\delta_E} }
        { \repairVernac
          {\Inductive
            {n}
            {\overrightarrow{p}}
            {\overrightarrow{i}}
            {u}
            {\overrightarrow{c}}
          }
          {\ModifyInductive
            {\delta_n}
            {\delta_{\overrightarrow{p}}}
            {\delta_{\overrightarrow{i}}}
            {\delta_u}
            {\delta_{\overrightarrow{c}}}
          }
          {\ModifyInductive
            {\delta_n'}
            {\delta_{\overrightarrow{p}}'}
            {\delta_{\overrightarrow{i}}'}
            {\delta_u}
            {\delta_{\overrightarrow{c}}'}
          }
        }
      }
    }
  \end{mathpar}

\end{Rules}
